\ifx\mainclass\undefined
\documentclass[cn,11pt,chinese,black,simple]{../elegantbook}
\usepackage{array}
\newcommand{\pp}[1]{\partial #1}
\newcommand{\dd}[1]{\mathrm{d}#1}
\newcommand{\abs}[1]{\left| #1 \right|}
\newcommand{\degr}[1]{#1^{\circ}}
\newcommand{\aint}{\int_{-\infty}^{+\infty} }
% \renewcommand{\dfrac}[2]{\dfrac{#1}{#2}}

\newcommand{\qfig}[2]{\begin{figure}[!htb]
    \centering
    \includegraphics[width=0.6\textwidth]{#1}
    \caption{#2}
\end{figure}}
\usepackage{circuitikz}

\renewcommand\arraystretch{1.5}
\begin{document}
\fi 

% Start Here
\chapter{噪声}


臊声限制了一个电路能够正确处理的最小信号电平, 现今的模拟电路没计者经常考虑
噪声的问題, 因为噪声与功耗、速度和线性度之间是互相制约的。

\section{噪声的统计特性}

噪声是一个随机过程。
在时域中的瞬时值是不可
预测的。
噪声的平均功率是可以
预测的。
允许我们用统计模型研究噪声。

\subsection{平均功率}

\(R_L\) 负载上消耗的平均功率为

$$\begin{array}{l}
    P_{a v}=\lim _{T \rightarrow \infty} \dfrac{1}{T} \int_{-T / 2}^{+T / 2} \dfrac{x^{2}(t)}{R_{L}} d t \\
    
\end{array}$$

简化之后

\[P_{a v}=\lim _{T \rightarrow \infty} \dfrac{1}{T} \int_{-T / 2}^{+T / 2} x^{2}(t) d t\]

实际功率同样可以由 \(\dfrac{P_{av}}{R_L}\) 得到。

\subsection{功率谱密度}

频谱描述的是噪声的频率分布,也叫功率谱密度(PSD),说明每个频率上信号的功率
噪声波形x(t)的功率谱密度 \(S_{X}(f)\) 定义为:
\(x(t)\) 在 \(f\) 附近的1Hz带宽中携带的平均功率。

与 \(P_{a v}\) 一样, \(S_{X}(f)\) 通常省略 \(R_{L},\) 用 \(V^{2} / H z\)表示,而不是\(W / H z_{\circ}\) 。
对 \(S_{X}(f)\) 求平方根也很常见,用 \(V / \sqrt{H z}\) 表示结果。
常见的噪声频谱的类型是“白噪声" 一在所有频率的值相同。
严格地说,白噪声并不存在,因为噪声所携带的总功率不可能
是无限的。
在一段频带内平坦的噪声频谱通常被称为白噪声。

在系统函数为 \(H(s)\) 的系统中 输入的噪声谱引 \(S_X(f)\) 起的输出噪声谱 \(S_Y(f)\) 满足 

\[S_Y(f) = S_X(f) \abs{H(f)}^2\] 

其中 \(H(f) = H(s)\left|_{s = j\omega = j 2 \pi f}\right.\)

对于实数信号,其噪声谱是频域的偶函数,因此可以分为双边谱与单边谱。

\subsection{噪声的概率分布}

虽然噪声的瞬时值是不可预测的,但是通过足够长时间的观
察噪声波形来构造幅值的“分布”是可能的。

中心极限定理指出,如果加入许多具有任意概率密度函数的
独立随机过程,总和的概率密度函数近似于高斯分布。

本课程中用频谱分析更多。

\section{噪声源特性}

\subsection{相干性}

对于给定的两种噪声 \(x_1(t)\) , \(x_2(t)\) 其噪声之和会体现两者的相干性。

\[\begin{aligned}
    P_{a v}=& \lim _{T \rightarrow \infty} \dfrac{1}{T} \int_{-T / 2}^{+T / 2}\left[x_{1}(t)+x_{2}(t)\right]^{2} d t \\
    =& \lim _{T \rightarrow \infty} \dfrac{1}{T} \int_{-T / 2}^{+T / 2} x_{1}^{2}(t) d t+\lim _{T \rightarrow \infty} \dfrac{1}{T} \int_{-T / 2}^{+T / 2} x_{2}^{2}(t) d t \\
    &+\lim _{T \rightarrow \infty} \dfrac{1}{T} \int_{-T / 2}^{+T / 2} 2 x_{1}(t) x_{2}(t) d t \\
    =& P_{a v 1}+P_{a v 2}+\lim _{T \rightarrow \infty} \dfrac{1}{T} \int_{-T / 2}^{+T / 2} 2 x_{1}(t) x_{2}(t) d t
\end{aligned}\]

独立器件产生的噪声波形是不相关的。

\subsection{信噪比}

信噪比 (SNR) 即信号与噪声噪声之比,被噪声干扰的信号,信噪比要足够高才能被解答。

\[SNR = \dfrac{P_{sig}}{P_{noise}}\]

根据噪声频谱密度 

\[P_{noise} = \aint S_{noise}(f) \dd{f} \]

频谱越宽,总的噪声功率越大,因此对于宽带放大器,特定频带的信号就会被整个频带的噪声破坏。所以电路的带宽必须限制在最小可接受值,以使总集成噪声功率
最小化。

\section{器件噪声分析}

整体流程可以分为记录器件的输入噪声频谱,分析到输出的传递函数,利用 \(S_Y(f) = S_X(f) \abs{H(f)}^2\) 分析输出噪声频谱,对输出求和得到总的输出频谱,对频率积分获得总噪声功率。

\subsection{电阻}

导体中电子的随机运动尽管平均电流为零, 但是它会引起导体两端电压的波动。电阻的热噪声可以用单边谱密度的串联电压源来模拟

\[S_v(f) = \overline{V_R^2 }= 4 k T R\]

并联电流源 

\[\overline{I_n^2} = \dfrac{4 k T}{R}\]

\(k\) 是玻尔兹曼常数。

\qfig{f49.png}{电阻噪声模型-电压源}

\qfig{f50.png}{电阻噪声模型-电流源}

\subsection{MOS 管}

\subsubsection{沟道噪声}

MOS 晶体管热噪声,主要是沟道噪声。对于\textbf{饱和}工作的\textbf{长沟道}MOS器件,沟道噪声可以通过一个连
接在漏极和源极之间的电流源来模拟

\[\overline{I_n ^2} = 4 k T \gamma g_m \]

此处的 \(\gamma\) 不是体效应系数, 对于长沟道晶体管是 \(\dfrac{2}{3}\) ,对于
亚微米MOS管更高。

\qfig{f51.png}{MOS 管噪声模型}

沟道电阻 \(r_o\) 不产生噪声,因为它不是一个实际的电阻。

\subsubsection{栅极电阻噪声}

对于宽晶体管,源极和漏极电阻可忽略不计,而栅极分布电阻较为显著。

\[\overline{V^2_{nR_G}} = 4 k T \dfrac{R_G}{3}\]

\qfig{f52.png}{栅极电阻噪声}

\qfig{f53.png}{栅极分布电阻噪声模型}

为了使得栅极电阻噪声可以忽略不计,其影响应远小于沟道噪声

\[4 k T \dfrac{R_{G}}{3}\left(g_{m} r_{O}\right)^{2} \ll\left(4 k T \gamma g_{m}\right) r_{O}^{2} \Rightarrow \dfrac{R_{G}}{3} \ll \dfrac{\gamma}{g_{m}}\]

上式通常可以成立。

\subsubsection{闪烁噪声}

在栅极氧化物和硅衬底之间的界面上出现许多“悬空”键,
产生额外的能态。
在界面上移动的载流子被随机捕获,随后被这种能量态释放,
在漏电流中引入“闪烁”噪声,模拟成与栅极串联的电压源,在饱和区大致为

\[\overline{V_n^2} = \dfrac{K}{C_{ox} WL} \dfrac{1}{f}\]

\(K\)是一个与工艺相关的常数,其数量级为 \(10^{-25} V^{2} F\)。可以看出噪声谱密度与频率成反比,即捕获和释放现象多发生在低频,也称为“ \(1/f\) ”噪声。 一般来说,PMOS器件比NMOS晶体管的 \(1/f\) 噪声小。

在低频,闪烁噪声功率接近无穷大,但是这个周期会相当的长,这么慢的速度无法和热漂移以及老化区分。

沟道热噪声与闪烁噪声谱密度的交点称为转角频率

\[\begin{array}{c}
    \begin{aligned}
        
    4 k T \gamma g_{m}&=\dfrac{K}{C_{o x} W L} \cdot \dfrac{1}{f_{C}} \cdot g_{m}^{2} \\
    f_{C}&=\dfrac{K}{\gamma C_{o x} W L} g_{m} \dfrac{1}{4 k T}
    \end{aligned}
\end{array}\]

闪烁噪声可以通过栅极的跨导变换转换为一个并联在源漏之间的电流源,并且栅极接地。

\section{电路噪声分析}

为了计算出输出噪声,将输入设为零,并根据电路中所
有的噪声源计算出输出的总噪声。

\subsection{输入参考噪声}

对噪声进行归一化,研究在特定增益下的噪声。

\[\overline{V_{n,out}^2} = A_v^2 \overline{V_{n,in}^2}\]

等效电路如图

\qfig{f54.png}{噪声处理的等效电路}

\subsection{有限输入阻抗的参考噪声}

在存在有限的输入阻抗时,使用单一电压源模拟会出现输出噪声随阻抗上升而下降的现象。理论可以证明再引入一个电流源是必须的,并且可以表示任何线性二端口网络的噪声。对于同样的输出噪声,分别计算输入参考电压与电流即可。

当输入阻抗很小时,输入参考噪声电流\(\overline{I_{n,in}}\)的作用才比较重要。

\qfig{f55.png}{完备的噪声模型}

\subsection{辅助定理}

低频时等效的条件

\[\overline{V_n^2} = \dfrac{\overline{I_n^2}}{g_m^2}\]

\qfig{f56.png}{等效电路}

\section{单级放大器的噪声}

\subsection{共源级}


\subsubsection{简单共源级}



\qfig{f57.png}{简单共源级}


\[\begin{aligned}
    \overline{V_{n, i n}^{2}} &=\dfrac{\overline{V_{n, o u t}^{2}}}{A_{v}^{2}} \\
    &=\left(4 k T \gamma g_{m}+\dfrac{K}{C_{o x} W L} \cdot \dfrac{1}{f} \cdot g_{m}^{2}+\dfrac{4 k T}{R_{D}}\right) R_{D}^{2} \dfrac{1}{g_{m}^{2} R_{D}^{2}} \\
    &=4 k T \dfrac{\gamma}{g_{m}}+\dfrac{K}{C_{o x} W L} \cdot \dfrac{1}{f}+\dfrac{4 k T}{g_{m}^{2} R_{D}}
\end{aligned}\]

\[\overline{I_{n,in}^2} = (4 k T \dfrac{\gamma}{g_{m}}+\dfrac{K}{C_{o x} W L} \cdot \dfrac{1}{f}+\dfrac{4 k T}{g_{m}^{2} R_{D}}) g_m^2\]

若是作为放大器,使得输出噪声最小,\(g_m\) 尽量大;若是作为电流源,\(g_m\) 尽量小。

\subsubsection{电流源负载的共源级热噪声}

\[\overline{V_{n,  {out}}^{2}}=4 k T\left(\gamma g_{m 1}+\gamma g_{m 2}\right)\left(r_{O 1} \| r_{O 2}\right)^{2}\]

那么为了作为放大器,\(g_{m1}\) 最大化, \(g_{m2}\) 最小化。

\[\begin{aligned}
    \overline{V_{n, in}^{2}} &=4 k T\left(\gamma g_{m 1}+\gamma g_{m 2}\right) \dfrac{1}{g_{m 1}^{2}} \\
    &=4 k T \gamma\left(\dfrac{1}{g_{m 1}}+\dfrac{g_{m 2}}{g_{m 1}^{2}}\right)
\end{aligned}\]

总输出噪声的功率为

\[
    \begin{aligned}
\overline{V_{n, \text {out}, \text {tot}}^{2}}&=\int_{0}^{\infty} 4 k T \gamma\left(g_{m 1}+g_{m 2}\right)\left(r_{O 1} \| r_{O 2}\right)^{2} \dfrac{d f}{1+\left(r_{O 1} \| r_{O 2}\right)^{2} C_{L}^{2}(2 \pi f)^{2}}\\
 &= \gamma\left(g_{m 1}+g_{m 2}\right)\left(r_{O 1} \| r_{O 2}\right) \dfrac{k T}{C_{L}}
    \end{aligned}
\]

\qfig{f58.png}{电流源负载的共源级热噪声}

当输入的信号摆幅为 \(V_m\) 时, 信噪比为

\[\begin{aligned}
    SNR _{\text {out }} &=\left[\dfrac{g_{m 1}\left(r_{O 1} \| r_{O 2}\right) V_{m}}{\sqrt{2}}\right]^{2} \cdot \dfrac{1}{\gamma\left(g_{m 1}+g_{m 2}\right)\left(r_{O 1} \| r_{O 2}\right)\left(k T / C_{L}\right)} \\
    &=\dfrac{C_{L}}{2 \gamma k T} \cdot \dfrac{g_{m 1}^{2}\left(r_{O 1} \| r_{O 2}\right)}{g_{m 1}+g_{m 2}} V_{m}^{2}
\end{aligned}\]

为了降低噪声影响需要增大电容,但是会导致带宽下降。

\subsection{共栅级热噪声}

以下讨论忽略沟道调制效应。

显然输出噪声仅由 \(R_D\) 提供,

\[\overline{V_{n,out}^2} = 4 k T R_D\]

那么 

\[\overline{I_{n,in}^2} = \dfrac{4 k T}{R_D}\]


\qfig{f59.png}{共栅级电路-输入参考电流}

显然

\[\left(4 k T \gamma g_{m}+\dfrac{4 k T}{R_{D}}\right) R_{D}^{2}=\overline{V_{n, i n}^{2}}\left(g_{m}+g_{m b}\right)^{2} R_{D}^{2}\]

\[\overline{V_{n, i n}^{2}}=\dfrac{4 k T\left(\gamma g_{m}+1 / R_{D}\right)}{\left(g_{m}+g_{m b}\right)^{2}}\]

\qfig{f60.png}{共栅级电路-输入参考电压}


\subsection{源随器热噪声}

由于源跟随器的输入阻抗比较高,在大小适中的驱动源阻抗
下,可以忽略输入参考噪声电流。


将 \(M_1\) 使用定理变换到输入,仅求解 \(M_2\)

\[\left.\overline{V_{n, \text {out}}^{2}}\right|_{M 2}=\overline{I_{n 2}^{2}}\left(\dfrac{1}{g_{m 1}}\left\|\dfrac{1}{g_{m b 1}}\right\| r_{O 1} \| r_{O 2}\right)^{2}\]

增益为

\[A_{v}=\dfrac{\dfrac{1}{g_{m b 1}}\left\|r_{O 1}\right\| r_{O 2}}{\dfrac{1}{g_{m b 1}}\left\|r_{O 1}\right\| r_{O 2}+\dfrac{1}{g_{m 1}}}\]

得到

\[\begin{aligned}
    \overline{V_{n, i n}^{2}} &=\overline{V_{n 1}^{2}}+\dfrac{\left.\overline{V_{n, o u t}^{2}}\right|_{M 2}}{A_{v}^{2}} \\
    &=4 k T \gamma\left(\dfrac{1}{g_{m 1}}+\dfrac{g_{m 2}}{g_{m 1}^{2}}\right)
\end{aligned}\]

\qfig{f61.png}{源随器的噪声}


源跟随器为输入信号添加噪声,并提供小于1的电压增益,
信噪比性能较差。
在低噪放大器中通常不使用源跟随器。

% End Here

\ifx\mainclass\undefined
\end{document}
\fi 