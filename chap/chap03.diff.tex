\ifx\mainclass\undefined
\documentclass[cn,11pt,chinese,black,simple]{../elegantbook}
\usepackage{array}
\newcommand{\ccr}[1]{\makecell{{\color{#1}\rule{1cm}{1cm}}}}
\newcommand{\pp}[1]{\partial #1}
\newcommand{\dd}[1]{\mathrm{d}#1}
\newcommand{\abs}[1]{\left| #1 \right|}
\newcommand{\degr}[1]{#1^{\circ}}
\newcommand{\aint}{\int_{-\infty}^{+\infty} }
% \renewcommand{\dfrac}[2]{\dfrac{#1}{#2}}

\newcommand{\qfig}[2]{\begin{figure}[!htb]
    \centering
    \includegraphics[width=0.6\textwidth]{#1}
    \caption{#2}
\end{figure}}
\usepackage{circuitikz}

\renewcommand\arraystretch{1.5}
\begin{document}
\fi 

% Start Here
\chapter{差动放大器}



\section{基本概念}

\textbf{共模电平}:差动信号的中间电平。

\section{直流分析}

\subsection{定量电压电流分析}

处于饱和区的晶体管,从输入端到电流源的过驱动电压满足:

\[V_{override} = V_{GS} - V_{TH} = \sqrt{\frac{2 I_D}{\mu_n C_{ox}(W/L)}}\]

并且我们还要建立两路输入电压的不同引起的输出电压不同,为此,从电流入手:

\[V_{in1}-V_{in2} = \sqrt{\frac{2 I_{D1}}{\mu_n C_{ox} (W/L)}} - \sqrt{\frac{2 I_{D2}}{\mu_n C_{ox} (W/L)}}\]

最终化简为

\begin{equation}\label{eq:diffvdiifi}
    \begin{aligned}
        \Delta I_D &= I_{D1} - I_{D2} \\
        &= \frac{1}{2} \mu_n C_{ox} \frac{W}{L} (V_{in1} - V_{in2}) \sqrt{\frac{4 I_{SS}}{\mu_n C_{ox} (W/L)} - (V_{in1} - V_{in2})^2}\\
        &= \frac{1}{2} \mu_n C_{ox} \frac{W}{L} (\Delta V_{in}) \sqrt{\frac{4 I_{SS}}{\mu_n C_{ox} (W/L)} - (\Delta V_{in})^2}
    \end{aligned}
\end{equation}


\qfig{f4.png}{理想电流源的基本差动对}

\subsection{差动信号增益}

根据式\ref{eq:diffvdiifi}求斜率表达式

\begin{equation}
    \begin{aligned}
        G_m = \frac{\pp \Delta I_D}{\pp \Delta V_{in}} = \frac{1}{2} \mu_n C_{ox} \frac{W}{L} \displaystyle\frac{\displaystyle\frac{4 I_{SS}}{\mu_n C_{ox} W/L} - 2\Delta V_{in}^2}{\sqrt{\displaystyle\frac{4 I_{SS}}{\mu_n C_{ox} W/L}-\Delta V_{in}^2}}
    \end{aligned}
\end{equation}


根据其图像,差动信号在差动信号直流电平相等时,可以得到最大的小信号增益。在非理想电流源中,直流电位差别较大的时候,存在电流完全被一路抽取的情况。

并且在 \(\Delta V_{in} = \sqrt{2 I_{SS} / (\mu_n C_{ox} W/L)}\) 增益下降为 \(0\) 。


\subsection{共模电平选取}

差动放大器的共模电平应保证放大管以及电流源均工作在饱和区,以获得最大且最稳定的小信号增益。

\qfig{f5.png}{非理想电流源的基本差动对}

\section{交流分析}

\subsection{叠加原理}

差动输入分别考虑为两个,通过线性电路狄叠加原理分别得到 \((V_{o1}-V_{o2}) = K V_{in1}\) , \((V_{o2}-V_{o1}) = K V_{in2} = K (- V_{in1})\) 之后

\[\frac{(V_{o1}-V_{o2})_{total}}{V_{in1}-V_{in2}} = K \]

\subsection{半边电路法}

电路完全对称,D1和D2代表任何三端有源器件。 假设 \(Vin1\) 和
\(Vin2\) 差动变化,如果电路保持线性,则 \(V_p\) 不变,在交流通路中接地。

\qfig{f13.png}{辅助定理}




\section{共模分析}

研究共模 (Common Mode) 响应的原因:
实际电路中共模信号的位置会影响整个电路的静态工作点,进而改变小信号的增益与电路摆幅。并且电路往往不会完全对称,共模输入也会自然地导致差动地输出。

差动对的共模响应取决于\textbf{尾电流源的输出阻抗}和\textbf{电路的不对称性},并表现为两个方面的影响: 对称电路的输出共模电平变化以及输入
共模电压变化在输出端产生差模分量。在模拟电路
中, 后者的影响要比前者严重得多。有鉴于此, 研究
共模响应时通常应当考虑不匹配。

响应可以通过 MOS 管并联求解,并联视为宽度加倍。增益为

\[A_{v,CM} = - \frac{R_D/2}{1/(2 g_m)+R_SS}\]

\qfig{f7.png}{共模响应- MOS 并联}

\section{失配分析}

电路的不对称既来自负载电阻也来自输入晶体管,通常后者产生的失配要大
得多。

仅考虑晶体管失配,即\(g_{m1}\) ,\(g_{m2}\) 不相等。即\textbf{共模输入}由于电路不对称引起的\textbf{差动输出},增益为

\[A_{CM-DM} = \frac{\Delta V_X - \Delta V_Y}{\Delta V_{in,CM}} = - \frac{(g_{m1}-g_{m2})R_D}{1+(g_{m1}+g_{m2})R_{SS}}\]


\qfig{f8.png}{失配分析的电路样例}

\section{共模响应总结}

无失配情况:尾电流源的有限阻抗引起共模电平变化。

失配情况:共模信号会产生差动输出。

高频情况:尾电流源的寄生电容阻抗的降低会引起共模电平的剧烈变化。

\qfig{f9.png}{高频情况下尾电流源的寄生电容}

\section{差动放大器评估}

为了合理地比较各种差动电路, 由共模变化而产生的不期望的差动成分必须用放大后所
需要的差动输出归一化。定义共模抑制比 CMRR (common-mode rejection ratio) ,可以理解为放大效果被失配稀释的程度。

\[CMRR = \left|\frac{A_{DM}}{A_{CM-DM}} \right|\]




\section{典型差动对}

\subsection{带源极负反馈的差动对}

增加了线性放大区输入的范围,但是源极负载消耗电压裕度,摆幅减小。

\qfig{f6.png}{带源极负反馈的差动对}

\subsection{带有负载的差动对}

差动对的负载往往是二极管接法或电流源接法的 MOS 管。

二极管接法:

\begin{equation*}
    \begin{aligned}
        A_v &= - g_{m1} \left(r_{o1} \| r_{o3} \| \frac{1}{g_{m3}}\right)
        \\& \approx - \frac{g_{m1}}{g_{m3}} = - \sqrt{\frac{\mu_n(W/L)_N}{\mu_p(W/L)_P}}
    \end{aligned}
\end{equation*}

电流源接法:

\[A_v = -g_{m,N} (r_{o,N} \| r_{o,P}) \]

\qfig{f10.png}{二极管接法的差动对负载}


\qfig{f11.png}{电流源接法的差动对负载}


\subsection{一道典型增益的求解}

在这里,可以将横跨电阻 \(R_1\) 看作是两个 \( 0.5R_1\) 的串联,便于将其使用半边电路对称处理。在直流条件下,两个 \( 0.5R_1\) 连接点的电平是固定的,因此在半边电路小信号中, 将之当作大小为 \( 0.5R_1\) 的接地电阻即可。

由辅助定理,尾电流源接地。


\qfig{f12.png}{带有横跨电阻的差动放大器}

\subsection{一道典型共模抑制比的求解}

求解本题目的方法具有普适性。

\begin{exercise}
    Due to a manufacturing defect, a large parasitic resistance has appeared between the drain and source terminals of M1. Assuming \(\lambda\) = \(\gamma\) = 0, calculate the small-signal gain, common-mode gain, and CMRR. 
    \qfig{f14}{题图}
\end{exercise}

\begin{solution}
    设电流源的漏极电平为 \(V_X\) ,输出端两端分别为 \(V_{out,1}\), \(V_{out,2}\) 。

    由于无沟道调制与体效应,小信号中总电流为 \(0\),即输出端两端反向,那么 \(V_{out,1} = - V_{out,2} = \displaystyle\frac{1}{2} V_o\) 。

    求解 \(A_{CM-DM}\) 时, 设\(V_{in,1} = V_{in,2} = V_{in}\)。

    求解 \(A_{DM}\) 时,设\(V_{in,1} = - V_{in,2} = \displaystyle \frac{1}{2} V_{in}\)。

    联立两条支路的基尔霍夫方程即可,不再赘述。
\end{solution}




% End Here

\ifx\mainclass\undefined
\end{document}
\fi 