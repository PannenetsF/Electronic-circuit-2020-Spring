\ifx\mainclass\undefined
\documentclass[cn,11pt,chinese,black,simple]{../elegantbook}
\usepackage{array}

\newcommand{\pp}[1]{\partial #1}
\newcommand{\dd}[1]{\mathrm{d}#1}
\newcommand{\abs}[1]{\left| #1 \right|}
\newcommand{\degr}[1]{#1^{\circ}}
\newcommand{\aint}{\int_{-\infty}^{+\infty} }
% \renewcommand{\dfrac}[2]{\dfrac{#1}{#2}}

\newcommand{\qfig}[2]{\begin{figure}[!htb]
    \centering
    \includegraphics[width=0.6\textwidth]{#1}
    \caption{#2}
\end{figure}}
\usepackage{circuitikz}

\renewcommand\arraystretch{1.5}
\begin{document}
\fi 

% Start Here
\chapter{频率响应}

本章研究的核心问题是放大器对不同频率的信号的放大效果,电容的存在使得增益等指标成为频率的函数,这就是\textbf{频率响应}。


\qfig{f48.png}{MOS 管完整等效电路}

\section{波特图}

横轴使用对数刻度 \(\lg f\) 或者 \(\lg \omega\),幅频纵轴为 \(20 \lg \abs{A_v} \) 单位为 \(dB\) ,相频纵轴仍为角度。并且曲线折线化,容忍误差。

以低通电路为例,有

\[\dot{U}_o = \frac{1 / (j \omega C)}{R + (1 + j \omega C)} U_i = \frac{1}{1 + j \omega R C} \dot{U}_i\]

那么 \[\abs{A_v} = \sqrt{\dfrac{1}{1 + \omega^2R^2C^2}} = \sqrt{\dfrac{1}{1 + (f / f_p)^2}}\] 
\[\varphi = -\arctan(\omega RC) = -\arctan(\dfrac{f}{f_p}) \] 

折线化的来源 



\begin{equation*}
    20 \lg \abs{A_v} \approx \left\{
    \begin{aligned}
        0, &f \ll f_p\\
        -20\lg \dfrac{f}{f_p}, &f \gg f_p
    \end{aligned}
    \right.
\end{equation*}


\begin{equation*}
    -\arctan \frac{f}{f_p} \approx \left\{
    \begin{aligned}
        &\degr{0}, &f \ll 0.1 f_p\\
        &\degr{-45} - \lg \frac{f}{f_p}, &0.1  f_p <f < 10 f_p\\
        &\degr{-90}, &f \gg 10 f_p
    \end{aligned}
    \right.
\end{equation*}

波特图的方式幅频特性的横轴均用对数刻度,拓宽视野;纵轴用对数,将乘除运算换算为加减运算,方便处理级联等问题;误差较小,近似程度可以接受。

本课程中采用拉普拉斯变换,需满足以下\textbf{标准形式}

\[A_v = K \frac{1 + \dfrac{s}{\omega_z}}{\left(1 + \dfrac{s}{\omega_{p1}}\right)\left(1 + \dfrac{s}{\omega_{p2}}\right)}\]

转化为实频域


\[A_v = K \frac{1 + \dfrac{j \omega}{\omega_z}}{\left(1 + \dfrac{j \omega}{\omega_{p1}}\right)\left(1 + \dfrac{j \omega}{\omega_{p2}}\right)}\]

之后转化为对数形式的加减法

\[20 \lg \abs{A_v} = 20 \lg K + 20 \lg \sqrt{1 + \left(\dfrac{j \omega}{\omega_z}\right)^2} - 20 \lg \sqrt{1 + \left(\dfrac{j \omega}{\omega_{p1}}\right)^2} - 20 \lg \sqrt{1 + \left(\dfrac{j \omega}{\omega_{p2}}\right)^2}\] 

\[\varphi = \arctan\dfrac{\omega}{\omega_z} - \arctan\dfrac{\omega}{\omega_{p1}} - \arctan\dfrac{\omega}{\omega_{p2}}  \]


\begin{definition}[频响零极点]
    定义以上的标准形式中的 \(\omega_z\) 为零点频率,\(\omega_p\) 为极点频率
\end{definition}

每经过一个极点,增益下降的速度增加 20 dB/10倍频,
相位在\(\pm 10\)倍频的范围内下降。

每经过一个零点,增益上升的速度增加 20 dB/10倍频,
相位在\(\pm 10\)倍频的范围内上升 \(\degr{90}\) 。

\section{密勒定理}

\begin{theorem}[密勒定理]
    如果一个电路可以完成如图~\ref{fig:ch5:1} 的变换,那么有
    \[\frac{V_X-V_Y}{Z} = \frac{V_X}{Z_1} \Rightarrow Z_1 = \frac{Z}{1 - \dfrac{V_Y}{V_X}} = \frac{Z}{1 - A_v}\]

    \[\frac{V_Y-V_X}{Z} = \frac{V_Y}{Z_2} \Rightarrow Z_2 = \frac{Z}{1 - \dfrac{V_X}{V_Y}} = \frac{Z}{1 - A_v^{-1}}\]
\end{theorem}

\begin{figure}[htb]
    \centering
    \includegraphics[width=0.6\textwidth]{f31.png}
    \caption{密勒等效变换}
    \label{fig:ch5:1}
\end{figure}

但是遗憾的是,密勒定理没有给出成立条件,在单信号通路时,通常不成立,阻抗与信号主通路并联时,通常成立。

\qfig{f32.png}{阻抗与主信号通路并联}

\section{密勒近似的评估}

严格求解各个频率下的系统特性使代数式变得十分复杂。然而, 在许多情况下, 我们采用低频的放大性能便能深人了解
电路的特性。密勒近似就是用低频的增益替代实际的增益,来近似了解电路的特性。实际上密勒近似的结果与实际存在一定的偏差,但是极大简化了运算过程。


\section{结点与极点}

放大器进行级联时,对于每一级的输入输出关系都有 

\[V_{out} = V_{in} \frac{\dfrac{1}{sC}}{R + \dfrac{1}{sC}} A_v = \frac{A_v}{1 + R s C}V_{in} \]

那么对整个放大器链有

\[\frac{V_{out}}{V_{in}}(s) =  \frac{A_1}{1 + R_S s C_{in}} \frac{A_2}{1 + R_1 s C_N}  \frac{1}{1 + R_2 s C_P} \]

那么可以说\textbf{一个结点贡献一个极点},极点频率为 \(\dfrac{1}{RC}\)虽在存在耦合时计算会十分复杂且不满足这个结论,但是仍然可以进行一定的估算,并且极大简化问题。值得注意的是,在实际使用时每一级放大器后面跟随的是其输出电阻,而不一定需要保持串联的形式,常见的输出电阻如沟道电阻就是和输出并联的。



\qfig{f33.png}{放大器的级联}

\qfig{f34.png}{一种耦合的例子}


\section{特征频率}

MOS 管的特征频率指的是源极漏极接地时,输入电流等于输出电流时的频率。
输入电流 \(I_{in} = I_G = \dfrac{V_{in}}{\dfrac{1}{s (C_{GS} + C_{GD})}}\) 。
输出电流 \(I_{out} = g_m V_{in}\) 。

解得 \[f_T = \frac{g_m}{2 \pi (C_{GS} + C_{GD})}\]

\qfig{f35.png}{MOS 小信号电路}


\section{共源级频响}

\subsection{密勒近似}

增益为 \(A_v = - g_m R_D\) 

输入极点 \(\omega_{in} = \frac{1}{R_S [C_{GS} + (1 - A_v) C_{GD}]}\)

输出极点 \(\omega_{out} = \frac{1}{R_D [C_{DB} + (1 - A_v^{-1}) C_{GD}]}\)

估算传输函数 \[\frac{V_{out}}{V_{in}}(s) = \frac{-g_m R_D}{(1 + \dfrac{s}{\omega_{in}}) (1 + \dfrac{s}{\omega_{out}})}\]

\qfig{f36.png}{共源级密勒近似}

输入阻抗

\[Z_{in} = \frac{1}{s C_{GS}} \| \frac{1/ (s C_{GD})}{1 + g_m R_D}\]

\subsection{精确解}

回路方程的解

\[\frac{V_{out}}{V_{in}}(s) = \frac{(s C_{GD} - g_m)R_D}{R_S R_D \xi^2 + [R_S(1 + g_m R_D) C_{GD} + R_S C_{GS} +R_D(C_{GD} + C_{DB} )]s + 1}\]

\[\xi = C_{DS} C_{GD} + C_{GS} C_{DB} + C_{GD} C_{DB}\]

采用\textbf{主极点近似法},当分母很复杂时,靠直觉假定,两个极点相距很远。

分母改写为 

\[D = (1 + \frac{s}{\omega_{p1}})(1 + \frac{s}{\omega_{p2}})\]

极点间隔很大,一次项系数为 \(\dfrac{1}{\omega_{p1}}\) 。再通过二次项系数解出另一个极点。

零点为 \(s_z = \dfrac{g_m}{C_{GD}}\)

增益在高频区间(零点之后)有反弹 \(C_{GD}\) 提供前馈通路,传导高频信号到输出端。

\qfig{f37.png}{共源级频响波特图}

输入阻抗

\[Z_{in} = \frac{1 + R_D (C_{GD} + C_{DB })s}{s C_{GD} (1 + g_m R_D + s C_{DB} R_D)} \| \frac{1}{s C_{GS}}\]



\subsection{对比评估}

估算无法估出零点,估算得到的极点也不准确。

主极点:

\[\omega_{p1} = \frac{1}{R_S(1 + g_m R_D) C_{GD} + R_S C_{GS} + R_D(C_{GD} + C_{DB})}\]

\[\omega_{in} = \frac{1}{R_S(1 + g_m R_D) C_{GD} + R_S C_{GS}}\]

只要 \( R_D(C_{GD} + C_{DB})\) 足够小,误差就很小。

次极点

\[\omega_{p2} = \frac{R_S(1 + g_m R_D)C_{GD} + R_S C_{DS} + R_D(C_{GD} + C_{DB})}{R_S R_D (C_{GS} C_{GD} + C_{GS} C_{DB} + C_{GD} C_{DB})}\]

\[\omega_{out} = \frac{1}{R_D(C_{GD} + C_{DB})}\]

只要 \(C_{GD}\) 够大,误差就很小。

在低频情况下,输入阻抗无穷大,均符合预期。

\section{源随器频响}

\subsection{精确解}

X点与Y点的相互作用极强,不宜用密勒近似。实际上,若 \(\lambda = \gamma = 0\),则X到Y的增益为 \(1\)。 密勒近似出现分母为 \(0\) 的情况。

\qfig{f38.png}{源随器等效电路}

\[\frac{V_{out}}{V_{in}} = \frac{s C_{DS} + g_m}{R_S (C_L C_{DS} + C_L C_{GD} + C_{GS} C_{GD})s^2 + (g_m R_S C_{GD} + C_L + C_{GS}) s + g_m }\]

主极点近似

\[\omega_{p1} = \frac{g_m}{g_m R_S C_{GD} + C_L +C_{DS}}\]

\[\omega_{p2} = \frac{1}{\omega_{p1}} \frac{g_m}{R_S (C_L C_{GS} + C_L C_{GD} + C_{GS} C_{GD})}\]

若 \(R_S = 0\) 那么输出极点消失,主极点近似得到的是精确解。

\[s_z = -\frac{-g_m}{C_{GS}}\]

\subsection{输入阻抗}

\[Z_{in} = (\frac{1}{s C_{DS}} + (1 + \frac{g_m}{s C_{GS}})\frac{1}{g_{mb} + s C_L}) \| \frac{1}{s C_{GD}} \]

若 \(C_L = 0\) 且 \(g_{mb} = 0\) 那么 \(C_{DS}\)  被自举,输出端感受不到输入的变化,输入阻抗无穷大。

\subsection{输出阻抗}

\[Z_{out} = \frac{1 + s R_s C_{GS}}{g_m + s C_{GS}} \| \frac{1}{s (C_L + C_{SB}) } = R_2 + R_1 \| (s L)\]

(仅第一部分)低频近似为 \(\frac{1}{g_m}\) 

(仅第一部分)高频近似为 \(R_S\)

其中  \(R_2 = \dfrac{1}{g_m}\) \(R_1 + R_2 = R_S\)

\[L = \frac{C_{GS}}{g_m} (R_S - \frac{1}{g_m})\]


在较高频率下,呈现负电阻效应。输出阻抗包含电
感成分。

\section{共栅级频响}

\subsection{无沟道调制效应}

此时晶体管两端无阻抗耦合,无需使用密勒定理。

\[\omega_{in} = \dfrac{1}{C_S (R_S \| \dfrac{1}{g_m + g_{mb}})}\]

\[\omega_{out} = \frac{1}{C_D R_D}\]

增益为 

\[\frac{V_{out}}{V_{in}}(s) = \frac{(g_m + g_{mb}) R_D}{1 + (g_m + g_{mb} R_S)} \dfrac{1}{(1 + \dfrac{s}{\omega_{in}}) (1 + \dfrac{s}{\omega_{out}})} \]

电容无密勒乘积项,可达到宽带。输入阻抗较低,级联时,容易引起电压信号损失。

输入阻抗 
\[Z_{in} = R_S + (\frac{1}{g_m +g_{mb}}) \| \frac{1}{s C_s}\]

\qfig{f39.png}{共栅级等效电路 \(\lambda = 0\)}

\subsection{沟道调制效应}

为简化计算,负载电阻用电流源代替。

\[\frac{V_{out}}{V_{in}} = \frac{1 + g_m r_o}{r_o C_L C_{in} R_s s^2 + [r_o C_l + R_S C_{in} + R_S C_L (1 + g_m r_o)]s + 1}\]

\[Z_{in} =  R_S +  \frac{1/(s C_L) + r_o}{1 + g_m r_o} \| \frac{1}{s C_{in}}\]

输入阻抗较低,且随频率升高而降低。

\qfig{f40.png}{共栅级等效电路 \(\lambda \neq 0\)}

\subsection{栅极电阻}

从源极看过去的阻抗为 \[Z = \frac{1 + s C_{GS}R_G}{g_m + s C_{GS}}\]

易知 \(I_{out}\) 和右侧的电流成正比,那么存在一个系数 \(\dfrac{Z}{Z + R_s}\)

那么 \[\omega_p = \frac{1+ g_m R_S}{(R_G + R_S)C_{GS}}\]

为获得宽带,应减小栅极电阻。

\qfig{f41.png}{共栅级等效电路 栅极电阻)}

\section{共源共栅级频响}

\subsection{极点}

输入的密勒电容是 \[C_X = (1 + \frac{g_{m1}}{g_{m2} + g_{mb2}})C_{GD1} + C_{DB1} + C_{SB2} + C_{GS2}\]

那么 \(A\) 节点 \[\omega_{p,A} = \frac{1}{R_S\left[C_{GS1}+(1 + \dfrac{g_{m1}}{g_{m2} + g_{mb2}})C_{GD1}\right]}\]

\(X\) 节点

\[\omega_{p.X} = \frac{g_{m2} + g_{mb2}}{2 C_{GD1} + C_{DB1} + C_{SB2} + C_{GS2}}\]


\(Y\) 节点无密勒乘积项

\[\omega_{p,Y} = \frac{1}{R_D (C_{DB2} + C_L + C_{GD2})}\]

一般情况下选取 \(\omega_{p,X}\) 离原点最远。

\qfig{f42.png}{共源共栅级等效电路}

\subsection{简化模型:电流源负载}

\[\frac{V_{out}}{I_{in}} = -\frac{1 + g_{m2} r_{o2}}{s C_X} \frac{1}{1 + (1 + g_{m2} r_{o2} ) \dfrac{C_Y}{C_X}+ s C_Y r_{o2} } \approx - \frac{g_[m2]}{s C_X} \frac{1}{g_{m2} \dfrac{C_Y}{C_X} + s C_Y}  \]

那么结点 \(X\) 的极点为 \(\dfrac{g_m2}{C_X}\) 与之前的结果一致。

暂时忽略 \(C_Y\) ,
输出阻抗为 \[Z_{out} = r_{o2} + (1 + g_{m2} r_{02}) \left(r_{o1} \| \dfrac{1}{s C_X}\right)\]

在高频时,输出阻抗会下降,即使忽略 \(C_Y\) 。


\qfig{f43.png}{简化模型}

\section{差动对频响}

对于差分信号,其频率响应与共源放大器一致(半边电路法)

\[V_{DM-OUT}(s) = \frac{V_{in}(s)}{2} H(s) - \left[-\frac{V_{in}(s)}{2}\right] H(s) = V_{in} (s) H(s)\]

\qfig{f44.png}{差动对等效电路}

共模增益带入之前的增益即可 

\begin{equation*}
    \begin{aligned}
        A_{CM-DM} &= \frac{\Delta g_m (R_D \| \dfrac{1}{s C_L}) }{1 + (g_{m1} + g_{m2})\left(r_{o3} \| \dfrac{1}{s C_P}\right)} \\
        &= \frac{\Delta g_m R_D}{1 + (g_{m1} + g_{m2})r_{o3}} \frac{1 + s C_P r_{o3}}{(1 + s R_D C_L)\left[1 + s \dfrac{C_P r_{o3}}{1 + (g_{m1} + g_{m2})r_{o3}}\right]}
    \end{aligned}
\end{equation*}

共模抑制比同样带入

\begin{equation*}
    \begin{aligned}
        CMRR &= \frac{g_{m1} + g_{m2} + 4 g_{m1} g_{m2} (r_{o3} \| \dfrac{1}{s C_P})}{2 \Delta g_m} \\
        &\approx \frac{g_m}{\Delta g_m} \frac{1 + 2 g_m r_{o3} + s C_P r_{o3}}{1 + s C_P r_[o3]} \\
        &\approx \frac{2 g_m^2 r_{o3}}{\Delta g_m } \frac{1 + s \dfrac{C_P}{2 g_m}}{1 + s C_[ r_{o3}]} 
    \end{aligned}
\end{equation*}

共模抑制比在高频下降,模抑制比的零点实际是 \(A_{Dm}\) 的零点


\qfig{f45.png}{共模等效电路}

\section{OTA 频响}

\qfig{f46.png}{OTA 等效电路}

系统函数为

\[\frac{g_{m N} r_{O N}\left(2 g_{m} p+C_{E} s\right) r_{O P}}{2 r_{O P} r_{O N} C_{E} C_{L} s^{2}+\left[\left(2 r_{O N}+r_{O P}\right) C_{E}+r_{O P}\left(1+2 g_{m P} r_{O N}\right) C_{L}\right] s+2 g_{m P}\left(r_{O N}+r_{O P}\right)}\]

主极点为 

\[\omega_{p 1} \approx \frac{2 g_{m P}\left(r_{O N}+r_{O P}\right)}{\left(2 r_{O N}+r_{O P}\right) C_{E}+r_{O P}\left(1+2 g_{m} P r_{O N}\right) C_{L}}\]

忽略第一项分母并且假定增益极大 \(2 g_{m} p r_{O N} \gg 1\)

\[\omega_{p 1} \approx \frac{1}{\left(r_{O N} \| r_{O P}\right) C_{L}}\]

那么标准形式为 
\[\begin{aligned}
    \frac{V_{o u t}}{V_{i n}} &=\frac{A_{0}}{1+s / \omega_{p 1}}\left(\frac{1}{1+s / \omega_{p 2}}+1\right) \\
    &=\frac{A_{0}\left(2+s / \omega_{p 2}\right)}{\left(1+s / \omega_{p 1}\right)\left(1+s / \omega_{p 2}\right)}
    \end{aligned}\]

\section{增益带宽积}

增益与带宽通常存在折中。
通常用增益带宽积(GBW)来综合考量。

\subsection{单极点电路}

极点和 \(-3 dB\) 带宽可以认为相等。

\[GBW = \abs{A_{v0}} \omega_{p,1}\]

\subsection{多极点电路}



对于两级结构有 

\[\frac{V_{o u t}}{V_{i n}}=\frac{A_{0}^{2}}{\left(1+\frac{s}{\omega_{p}}\right)^{2}}\]

即 

\[\frac{A_{0}^{2}}{1+\frac{\omega_{-3 d B}^{2}}{\omega_{p}^{2}}}=\frac{A_{0}^{2}}{\sqrt{2}}\]

解得

\[\begin{aligned}
    \omega_{-3 d B} &=\sqrt{\sqrt{2}-1 \omega_{p}} \\
    & \approx 0.64 \omega_{p}
    \end{aligned}\]

因此

\[\mathrm{GBW}=\sqrt{\sqrt{2}-1} A_{0}^{2} \omega_{p}\]

对于 N 级联使 \(-3 dB\) 带宽减小,使增益带宽积增大

\[\omega_{-3 d B}=\sqrt{\sqrt[N]{2}-1 \omega_{p}}\]

\[\mathrm{GBW} = \omega_{-3 d B} A_0^N\]



\qfig{f47.png}{多级共源共栅结构}

% End Here

\ifx\mainclass\undefined
\end{document}
\fi 