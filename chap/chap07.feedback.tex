\ifx\mainclass\undefined
\documentclass[cn,11pt,chinese,black,simple]{../elegantbook}
\usepackage{array}
\newcommand{\pp}[1]{\partial #1}
\newcommand{\dd}[1]{\mathrm{d}#1}
\newcommand{\abs}[1]{\left| #1 \right|}
\newcommand{\degr}[1]{#1^{\circ}}
\newcommand{\aint}{\int_{-\infty}^{+\infty} }
% \renewcommand{\dfrac}[2]{\dfrac{#1}{#2}}

\newcommand{\qfig}[2]{\begin{figure}[!htb]
    \centering
    \includegraphics[width=0.6\textwidth]{#1}
    \caption{#2}
\end{figure}}
\usepackage{circuitikz}

\renewcommand\arraystretch{1.5}
\begin{document}
\fi 

% Start Here
\chapter{负反馈}

\section{负反馈系统}

\subsection{系统组成}

一个典型的负反馈系统作用时使净输入量减小,或者使输出量
的变化减弱,通常包括以下部分,


\begin{enumerate}
    \item \(H(s)\) :前馈网络,通常是一个放大器。
    \item \(G(s)\) :反馈网络,通常是一个与频率无关的函数。
    \item \(X(s) - G(s) Y(s)\) :反馈误差
    \item \(\dfrac{Y(s)}{X(s)}\) :闭环传输函数
    \item \(H(s)\) :开环传输函数
\end{enumerate}

满足方程

\[\begin{array}{c}
    \begin{aligned}
    Y(s)&=H(s)[X(s)-G(s) Y(s)] \\
    \frac{Y(s)}{X(s)}&=\frac{H(s)}{1+G(s) H(s)}
    \end{aligned}
\end{array}\]

\qfig{f62.png}{典型负反馈系统-复频域}

设计时通常满足以下原则 
\begin{itemize}
    \item 反馈网络 \(G(s)\) 与频率无关,不移相
    \item 深度负反馈 \begin{itemize}
        \item 反馈误差也就是前馈网络的输入极小,可以认为虚地
        \item \(X(s)\) 与 \(Y(s)\) 的比例几乎只与反馈有关
        \item \(G(s) H(s) \gg 1\) 
    \end{itemize}
\end{itemize}

以简单的放大器构成负反馈系统,那么有

\qfig{f63.png}{简单负反馈放大器}

参照以上的讨论,有

\begin{enumerate}
    \item \(A\) : 开环增益
    \item \(\beta\) :反馈系数
    \item \(\beta A\) :环路增益
    \item \(\dfrac{Y}{X}\) :闭环增益
\end{enumerate}

并且满足 \[\dfrac{Y}{X} = \dfrac{A}{1 + \beta A}\]

\subsection{负反馈系统的性质}

\subsubsection{稳定增益}

对于深度负反馈,有 \(\beta A \gg 1\) ,那么

\[\begin{array}{c}
    \frac{Y}{X}=\frac{A}{1+\beta A}=\frac{1}{\beta}\left(1-\frac{1}{1+\beta A}\right) \approx \frac{1}{\beta}\left(1-\frac{1}{\beta A}\right) \approx \frac{1}{\beta} \\
\end{array}\]

闭环增益 \(\dfrac{Y}{X}\) 对于开环增益 \(A\) 的变化敏感程度下降,并且 \(\beta\) 越大增益越稳定,但是同时增益也在减小。
实际中的增益很容易随工艺、温度等发生变化,增益不精确也不稳定,通过负反馈可以实现对增益更精确的控制。

\subsubsection{改变终端阻抗}

开环输入阻抗,即无反馈时的输入阻抗。闭环输入阻抗,即有反馈时的输入阻抗。在之后的小节进行讨论。



\subsubsection{改变带宽}

对于单极点的前馈函数 \(A(s) = \dfrac{A_0}{1 + \dfrac{s}{\omega_0}}\)  增加反馈得到

\[\dfrac{Y(s)}{X(s)} = \dfrac{A(s)}{1 + \beta A(s)} = \dfrac{\dfrac{A_0}{1 + \beta A_0}}{1 + \dfrac{s}{(1+\beta A_0)\omega_0}}\] 

主极点得到改善,但是增益带宽积不变。

\subsubsection{改善线性度}

负反馈放大器的增益稳定性被改善,线性度较好,代价是增益的减小。

\subsubsection{环路增益的求解}

对输入置零,断开环路在反馈点增加测试信号,读取输出端信号,计算与测试信号的比值,取反便可得到环路增益 (loop gain) 。

\qfig{f64.png}{环路增益的求解}

\[\begin{aligned}
    V_t \beta (-1) A &= V_F \\
    \dfrac{V_F}{V_t} = -\beta A
\end{aligned}\]

\section{放大器}

\subsection{基本放大器}

有这样四种基本的放大器,基于这些放大器逐步开展反馈机制的讨论。

\qfig{f65.png}{基本放大器类型}


\begin{longtable}{lllll}
    \caption{基本放大器的阻抗对比} \\
    \toprule \textbf{放大器类型} & \textbf{电压放大器} & \textbf{跨阻放大器} & \textbf{跨导放大器} & \textbf{电流放大器} \\
    \midrule
    \endfirsthead
    
    \toprule \textbf{放大器类型} & \textbf{电压放大器} & \textbf{跨阻放大器} & \textbf{跨导放大器} & \textbf{电流放大器} \\
    \midrule
    \endhead

    \hline
    \multicolumn{5}{c}{见下页}\\   
    \bottomrule
    \endfoot

    \bottomrule
    \endlastfoot

    \textbf{输入阻抗}  & 高              & 低              & 高              & 低              \\
    \textbf{输出阻抗}  & 低              & 低              & 高              & 高              \\
    
\end{longtable}

\qfig{f66.png}{基本放大器的单 MOS 实现}

\qfig{f67.png}{基本放大器的改进实现}

\subsection{检测与反馈机制}

对控制机制进行命名定义:(输出端电学量)-(输入端电学量),“XX—电压反馈”的输入信号必须是电压,
“XX—电流反馈”的输入信号必须是电流。有四种基本反馈机制 
\begin{itemize}
    \item 电压 - 电压反馈
    \item 电压 - 电流反馈
    \item 电流 - 电流反馈
    \item 电流 - 电压反馈
\end{itemize}

图~\ref{fig:ch07:base-feedback}列出了几种基本的反馈方式逐一进行说明:

\begin{itemize}
    \item (a) 电压通过电阻分压反馈到输入电压
    \item (b)(c) 电流通过电阻分压反馈到输入电压
    \item (d) 电压通过电阻反馈到输入电压,并且通过差动对完成电压减法
    \item (e)(f) 电压通过电阻分压反馈到输入电压,通过栅源压降完成减法
    \item (g)(h) 电流通过并联反馈到输入电流
\end{itemize}

\begin{figure}[htb]
    \centering
    \includegraphics[width=0.8\textwidth]{f68.png}
    \caption{基本反馈方式}
    \label{fig:ch07:base-feedback}
\end{figure}

在输出端检测电压应使用并联,检测电流使用串联。

可以注意到这样的规律:电压负反馈的输入量与反馈量在两个节点,
电流负反馈的输入量与反馈量在同一个节点。这个规律可以用来确定反馈类型。

\subsubsection{电压-电压反馈}

前馈网络相当于一个差动输入单端输出的放大器,反馈网络是一个分压器,与输出电压并联分压,反馈串联到输入。理想的反馈网络有无穷大的输入阻抗,接受全部电压,具有零输出阻抗,电压完全反馈。

\qfig{f69.png}{电压-电压反馈}

\paragraph{输出阻抗}

负反馈系统希望得到线性性良好的输入输出关系,当放大器带载时,逐渐减小其阻值。在开环结构中,输出将按 \(R_{ L } /\left(R_{ L }+R_{ out }\right)\) 比例下降。在反馈系统 中,即使 \(R_{ L }\) 减小, \(V_{\text {out }}\) 也始终是 \(V_{\text {ia }}\) 的合理线性复制。也就是说,只要环路增益保持远大于 \(1\) 则增益与负载无关。

\qfig{f70.png}{带载放大器}

检测反馈网络的输出电阻,理想反馈网络的输入电阻无穷大,所以反馈网络返回的电流忽略不计。

\qfig{f71.png}{输出阻抗检测}

可以得到

\[\frac{V_{X}}{I_{X}}=\frac{R_{\text {out }}}{1+\beta A_{0}}\]

可以看出输出阻抗和增益 \(\dfrac{1}{\beta}\) 成比例减小。

整体上,输出阻抗被减小,更接近理想
电压源输出。
输出电压更稳定,不易受负载
电阻影响。

\paragraph{输入阻抗}

\qfig{f72.png}{输入阻抗示意}

\qfig{f73.png}{输入阻抗检测}

进行求解

\[\begin{array}{l}
    V_{e}= V _{X}-V_{F}=V_{X}-\beta A_{0} I_{X} R_{i n} \\
    I_{X} R_{i n}= V _{X}-\beta A_{0} I_{X} R_{i n} \\
    \frac{V_{X}}{I_{X}}=R_{i n}\left(1+\beta A_{0}\right)
\end{array}\]

输入阻抗被增大,更接近理想电压放大器。

\subsubsection{电流-电压反馈}

反馈网络与前馈网络的输出量串联,与输入量串联。
前馈网络必须加负载电阻,以产生电流。
理想反馈网络具有零输入阻抗,将输出电流全部接收。
理想反馈网络具有零输出阻抗,将反馈电压全部送到输入端。

\qfig{f74.png}{电流-电压反馈}

\paragraph{增益}

闭环增益为 

\[\begin{array}{l}
    \frac{I_{o u t}}{V_{i n}}=\frac{G_{m}}{1+G_{m} R_{F}}
\end{array}\]

\qfig{f75.png}{开环增益检测}

开环增益为 \[-\frac{I_{o u t}}{I_{t}}=G_{m} R_{F}\]

\paragraph{输出阻抗}

满足

\[-R_{F} I_{X} G_{m}=I_{X}-\frac{V_{X}}{R_{O u t}} \quad \frac{V_{X}}{I_{X}}=R_{o u t}\left(1+G_{m} R_{F}\right)\]

\qfig{f76.png}{输出阻抗检测}

\paragraph{输入阻抗}

输入阻抗被增大,更接近理想电压放大器。

\[\begin{array}{l}
    I_{X} R_{i n} G_{m}=I_{o u t} \\
    V_{e}=V_{X}-G_{m} R_{F} I_{X} R_{i n} \\
    \frac{V_{X}}{I_{X}}=R_{i n}\left(1+G_{m} R_{F}\right)
\end{array}\]

\qfig{f77.png}{输入阻抗检测}

\subsubsection{电压-电流反馈}

反馈网络与前馈网络的输出量并联,与输入量并联。
理想反馈网络具有无穷大输入阻抗,将输出电压全部接收。
理想反馈网络具有无穷大输出阻抗,将反馈电流全部送到输入端。

\qfig{f78.png}{电压-电流反馈电路}

增益为 
\[\frac{V_{out}}{I_{in}} = \dfrac{R_0}{1 + g_{mF}} R_0\]

\qfig{f79.png}{输入与输出阻抗}

输入阻抗为

\[\begin{aligned}
    I_{F} &=I_{X}-\frac{V_{X}}{R_{i n}} \frac{V_{X} R_{0} g_{m F}}{R_{i n}}=I_{F} \\
    \frac{V_{X}}{I_{X}} &=\frac{R_{i n}}{1+g_{m F} R_{0}}
\end{aligned}\]

输出阻抗为

\[\begin{array}{c}
    \begin{aligned}
        I_{e}&=-I_{F}=-g_{m F} V_{X} \\
        V_{M}&=-R_{0} g_{m F} V_{X} \\
        I_{X}&=\frac{V_{X}-V_{M}}{R_{o u t}}=\frac{V_{X}+g_{m F} R_{0} V_{X}}{R_{o u t}} \\
        \frac{V_{X}}{I_{X}}&=\frac{R_{o u t}}{1+g_{m F} R_{0}}
    \end{aligned}
\end{array}\]

输入阻抗被减小,更接近理想电流放大器。输出阻抗被减小,更接近理想电压源输出。

\subsubsection{电流电流反馈}

\qfig{f80.png}{电流-电流反馈图}

反馈网络与前馈网络的输出量串联,与输入量并联。
理想反馈网络具有零输入阻抗和无穷大输出阻抗。

闭环增益 \[\dfrac{A_1}{1 + \beta A_1}\] 。
输入阻抗减小了\(1 + \beta A_1\),情形类似于“电压—电流反馈”的输入阻抗。
输出阻抗增大了\(1 + \beta A_1\),情形类似于“电流—电压反馈”的输出阻抗。

\subsection{负反馈系统的评估}


\subsubsection{阻抗}

负反馈总是使放大器的输入/输出阻抗变得更加理想。

\begin{longtable}{lll}
    \caption{反馈引起的阻抗变化} \\
    \toprule \text { 反馈类型 } & \text { 输入阻抗 } & \text { 输出阻抗 } \\
    \midrule
    \endfirsthead
    
    \toprule\text { 反馈类型 } & \text { 输入阻抗 } & \text { 输出阻抗 } \\
    \midrule
    \endhead

    \hline
    \multicolumn{3}{c}{见下页}\\   
    \bottomrule
    \endfoot

    \bottomrule
    \endlastfoot
    \text { 电压一电压反馈 } & \text { 增大 } & \text { 减小 } \\
    \text { 电流一电压反馈 } & \text { 增大 } & \text { 增大 } \\
    \text { 电压一电流反馈 } & \text { 减小 } & \text { 减小 } \\
    \text { 电流一电流反馈 } & \text { 减小 } & \text { 增大 } \\
\end{longtable}

\subsubsection{噪声}


负反馈没有消除前馈通路的噪声。实际中,反馈网络也会引入噪声,因此总的噪声性能更差。

\qfig{f81.png}{负反馈对噪声的影响}


\[\begin{array}{l}
    \left(V_{\text {in }}-\beta V_{\text {out }}+V_{n}\right) A_{1}=V_{\text {out }} \\
    V_{\text {out }}=\left(V_{\text {in }}+V_{n}\right) \frac{A_{1}}{1+\beta A_{1}}
\end{array}\]

\section{加载效应}


反馈网络的输入电阻不是“无穷大”,开环增益被减小。
前述理想模型下输出阻抗的计算也不适用。
如果由电容反馈分压,则在低频下几乎不影响开环增益。这都被称为“加载效应”。
反馈网络和前馈网络纠缠,难以清晰分割。反馈回路可能不唯一。

\subsection{线性二端口网络模型与放大器}

二端口网络用来描述线性电路,可以描述加载效应。
过程为
断开加载环路,计算开环增益。
确定反馈系数,得出环路增益。
这是一种近似(环路单向化假设),并非精确解。

\qfig{f82.png}{二端口网络模型}

\[\begin{array}{l}
    V_{1}=Z_{11} I_{1}+Z_{12} I_{2} \\
    V_{2}=Z_{21} I_{1}+Z_{22} I_{2} \\
    \\
    I_{1}=Y_{11} V_{1}+Y_{12} V_{2} \\
    I_{2}=Y_{21} V_{1}+Y_{22} V_{2} \\
    \\
    V_{1}=H_{11} I_{1}+H_{12} V_{2} \\
    I_{2}=H_{21} I_{1}+H_{22} V_{2} \\
    \\
    I_{1}=G_{11} V_{1}+G_{12} I_{2} \\
    V_{2}=G_{21} V_{1}+G_{22} I_{2}
\end{array}\]

\subsection{电压-电压反馈加载效应}

对于电压放大器而言,G模型更直观、更有物理意义。
于普通的电压放大器,内部反馈\(G_{12}I_2\)被忽略,则\(G_{11}\)就是输入
\textbf{导纳},\(G_{22}\)就是输出\textbf{阻抗},\(G_{21}\)就是开路电压增益。

为简化计算,内部反馈被忽略,实现环路单向化近似。

那么 \(G_{11} \rightarrow Z_{in}\), \(G_{22} \rightarrow Z_{out}\) , \(G_{21} \rightarrow A_0\)。若是 \(g_{11} = g_{22} = 0\) 则得到理想模型,没有加载效应。

\qfig{f83.png}{二端口网络表示电压-电压反馈}

进一步确定加载效应的参数,输出短接,可以获得 \(g_{22}\) , 输入开路,可以获得 \(g_(11)\),\(g_{21}\) 则是反馈网络的增益。

\subsection{电流-电压反馈加载效应}

前馈网络Y模型适用;
反馈网络Z模型适用。

\qfig{f84.png}{电流-电压反馈加载效应}

\subsection{电压-电流反馈加载效应}

前馈网络Z模型适用;
反馈网络Y模型适用。

\qfig{f85.png}{电压-电流反馈加载效应}

\subsection{电流-电流反馈加载效应}

前馈网络H模型适用;
反馈网络H模型适用。

\qfig{f86.png}{电流-电流反馈加载效应}

% End Here

\ifx\mainclass\undefined
\end{document}
\fi 