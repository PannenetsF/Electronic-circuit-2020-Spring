\ifx\mainclass\undefined
\documentclass[cn,11pt,chinese,black,simple]{../elegantbook}
\usepackage{array}
\newcommand{\ccr}[1]{\makecell{{\color{#1}\rule{1cm}{1cm}}}}
\newcommand{\qfig}[2]{\begin{figure}[!htb]
    \centering
    \includegraphics[width=0.6\textwidth]{#1}
    \caption{#2}
  \end{figure}}
\begin{document}
\fi 

% Start Here
\chapter{BJT放大电路}

\section{工作区间}

\begin{longtable}{llll}
    \caption{三极管工作区间} \\
    \toprule 区间 & \(U_{BE}\) & \(U_{CE}\) & \(I_C\) \\
    \midrule
    \endfirsthead
    
    \toprule 区间 & \(U_{BE}\) & \(U_CE\) & \(I_C\) \\
    \midrule
    \endhead

    \hline
    \multicolumn{4}{c}{见下页}\\   
    \bottomrule
    \endfoot

    \bottomrule
    \endlastfoot

    截止 & \(< U_{on}\) & \(> U_{BE}\) & \(\leq I_{CEO}\)  \\
    放大 & \(\geq U_{on}\) & \(\geq U_{BE}\) & \(\beta I_{B}\) \\
    饱和 & \(\geq U_{in}\) & \(\leq U_{BE}\) & \(< \beta I_B\)
    
\end{longtable}

\section{直流通路}

直流通路静态工作点的求解过程:

\qfig{f1.png}{一个BJT放大电路}

\begin{solution}
    \begin{enumerate}
        \item 确定基极电势,由于基极电流极小,将之视为开路: \[U_{BB} = \frac{R_{b1}}{R_{b1}+R_{b2} V_{CC}}\]
        \item 确定基极电流,对供电电路进行戴维南等效:\[U_{eq} = U_{BB}\] \[R_{eq} = R_{b1} \| R_{b2}\]
        
        利用基尔霍夫电压定律列方程: \[U_{eq} = I_{BQ}  R_{eq} + U_{BEQ} + I_{BQ}  (1 + \beta)  R_e\]
        其中\(U_{BEQ}\)是基极与射极间固定的电压降
        \item 确定集电极射极的电势差:\[U_{CEQ} = V_{CC} - R_c  \beta I_{BQ} - R_e  (1 + \beta) I_{BQ}\]
        或者 \[U_{CEQ} \approx V_{CC} - R_c * \beta I_{BQ} - R_e   \beta I_{BQ}\] 
    \end{enumerate}
\end{solution}

\qfig{f2.png}{戴维南等效}

\section{交流通路}

交流通路求解较为简单,只需注意一点

\[r_{be} = r_{bb'} + \frac{U_T}{I_{BQ}}\]


\section{三种基本放大器}


\begin{longtable}{llll}
    \caption{BJT放大器} \\
    \toprule \textbf{指标} & \textbf{共射} & \textbf{共集} & \textbf{共基} \\
    \midrule
    \endfirsthead
    
    \toprule \textbf{指标} & \textbf{共射} & \textbf{共集} & \textbf{共基} \\
    \midrule
    \endhead

    \hline
    \multicolumn{4}{c}{见下页}\\   
    \bottomrule
    \endfoot

    \bottomrule
    \endlastfoot

    $A_i$       & 大           & 大                                                         & 小           \\
    $A_v$       & 大           & 小                                                         & 大           \\
    $R_i$       & 中           & 大                                                         & 小           \\
    $R_o$       & 大           & 小                                                         & 大           \\
    \hline 用途          & 低频电压放大器     & \begin{tabular}[c]{@{}l@{}}输入级、\\ 输出级、\\ 缓冲器\end{tabular} & 宽频带放大器     

    
\end{longtable}

\section{失真分析}

失真的发生代表放大器从正常工作的放大区,滑向了截至区或者饱和区。

以共射为例,
Q 点过低,小信号的波动使得放大器发生截止失真;Q 点过高,小信号的波动使得放大器发生饱和失真。又共射放大器为反向放大器,失真波形分别位于顶部与底部。




% End Here

\ifx\mainclass\undefined
\end{document}
\fi 