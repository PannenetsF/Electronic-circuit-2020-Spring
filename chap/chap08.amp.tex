\ifx\mainclass\undefined
\documentclass[cn,11pt,chinese,black,simple]{../elegantbook}
\usepackage{array}
\newcommand{\pp}[1]{\partial #1}
\newcommand{\dd}[1]{\mathrm{d}#1}
\newcommand{\abs}[1]{\left| #1 \right|}
\newcommand{\degr}[1]{#1^{\circ}}
\newcommand{\aint}{\int_{-\infty}^{+\infty} }
% \renewcommand{\dfrac}[2]{\dfrac{#1}{#2}}

\newcommand{\qfig}[2]{\begin{figure}[!htb]
    \centering
    \includegraphics[width=0.6\textwidth]{#1}
    \caption{#2}
\end{figure}}
\usepackage{circuitikz}

\renewcommand\arraystretch{1.5}
\begin{document}
\fi 

% Start Here

\chapter{运算放大器设计基础}

我们粗略地把运放定义为“高增益的差动放大器”。所谓“高”,指的是对应用,增益足够高,通常增益范围在 \( 10^{1} \sim 10^{5} \) 。由于运放一般用来实现一个反馈系统,以提供稳定性、精度等指标,其开环增益的大小根据闭环电路的精度要求来设计。应用范围极广:信号运算、高速放大、滤波、缓冲器。

理想的运放:无限大增益、无限高输入阻抗、零输出阻抗。
实际的运放设计却是多个指标的复杂折中。



\section{一级运放}

一级运放的期望增益为 \(\dfrac{1}{\beta}\) ,那么误差为 \(\dfrac{\dfrac{1}{\beta} - \dfrac{V_{out}}{V_{in}}}{\dfrac{1}{\beta}}\) ,其中实际增益为 \(\dfrac{V_{o u t}}{V_{i n}}=\dfrac{A_{1}}{1+\dfrac{R_{2}}{R_{1}+R_{2}} A_{1}}\)

\qfig{f87.png}{单级放大器}

带有极点的运放处理阶跃信号时, \(V_{in} = \dfrac{a}{s}\) 。
系统函数为

\[ \dfrac{A(s)}{1+\beta A(s)}=\dfrac{A_{0} /\left(1+s / \omega_{0}\right)}{1+\beta A_{0} /\left(1+s / \omega_{0}\right)}=\dfrac{A_{0} /\left(1+\beta A_{0}\right)}{1+s /\left[\left(1+\beta A_{0}\right) \omega_{0}\right]} \approx \dfrac{1 / \beta}{1+s /\left(\beta A_{0} \omega_{0}\right)} \]

输出为 \[
    \begin{array}{l}
        \begin{aligned}
            
            V_{\text {out }}(t) &= L ^{-1}\left[V_{\text {out }}(s)\right]= L ^{-1}\left[V_{\text {in }}(s) \dfrac{A(s)}{1+\beta A(s)}\right] \\
            \approx & a \dfrac{1}{\beta}\left(1-\exp \dfrac{-t}{1 /\left(\beta A_{0} \omega_{0}\right)}\right) u(t) 
        \end{aligned}
    \end{array}    
\]

其中 \[1+\dfrac{R_{1}}{R_{2}}=\dfrac{1}{\beta}\]

可以从时域的输出中计算速度以及对应约束的带宽。单位增益带宽定义为 \(A_0 \omega_0\) 其中 \(\omega_0\) 为主极点。

\qfig{f88.png}{带有极点的单级放大器}


\subsection{套筒式运放}

设计流程相对固定。

\qfig{f89.png}{套筒式共源共栅结构}

\begin{enumerate}
    \item 根据功率分配支路电流;
    \item 根据摆幅分配过驱动电压;
    \item 根据I-V特性计算宽长比;
    \item 检查增益是否符合要求;
    \item 若不符合要求,调整沟道长度。
    \item 确定共模电平和偏置电压
\end{enumerate}

\begin{example}
    Design a fully differential telescopic op amp with the following specifications: \(V_{D D}=3 V\), peak-to-peak differential output swing \(=3 V\), power dissipation \(=10 mW\), voltage gain \(=2000 .\) Assume that \(\mu_{n} C_{o x}=60 \mu A / V ^{2}, \mu_{p} C_{o x}=\) \(30 \mu A / V ^{2}, \lambda_{n}=0.1 V ^{-1}, \lambda_{p}=0.2 V ^{-1}\) (for an effective channel length of \(0.5 \mu m\) ), \(\gamma=0\), and \(V_{T H N}=\left|V_{T H P}\right|=\)
\(0.7 V\)
\end{example}

\begin{solution}
    从功耗与电压得到,总电流为 \(3.33 mA\),分配各自 \(1.5 mA\) 给两路,其余给参考电流源。

    摆幅为 \(3 V\) ,那么 X, Y 均可以摆动 \(1.5 V\),且要保证3,4,5,6管子不进入线性区。同时管子1,7都是共源共栅结构,电压影响较小,可以适当的降低其过驱动电压,使之在线性区边缘。
    
    目前共有 \(1.5 V\) 过驱动电压,那么让电流最大的 9 号的尺寸可接受,使之接受 \(0.5 V\);由于 PMOS 的迁移率低,分配的过驱动电压多一点, \(300 mV\),剩下的均分给 NMOS 1,3 ,各自 \(200 mV\),作为初始猜测。

    计算时,为了简化,暂时忽略沟道调制 \[I_D = \dfrac{1}{2} \mu_n C_{ox} \dfrac{W}{L} (V_{OV})^2\]
    带入求解 得到 \((W/L)_{1-4} = 1250\) , \((W/L)_{5-8} = 1111\) ,\((W/L)_9 = 400\) 。

    进一步验证增益 \[A_{v} \approx g_{m 1}\left[\left(g_{m 3} r_{O 3} r_{O 1}\right) \|\left(g_{m 5} r_{O 5} r_{O 7}\right)\right]\] 

    发现结果为 \(1419\) 比指标低,一般通过调节尺寸来改进,又 \(\lambda \propto 1/L\) ,\(g_m r_o \propto \sqrt{2 \mu C_{ox} (W/L)/I_D} \dfrac{1}{\lambda} \) 那么 \( g_m r_o \propto \sqrt{W L / I_D}\) 。由于 PMOS 对性能影响较小,可以调制其尺寸。

    \((W/L)_{5-8} = 1111\) 保持,但是\(L\) 翻倍时,增益变为 \(4000\) 。同时由于其尺寸变大,可以从尾电流源中分配一些电压。


\end{solution}

其问题也很明显:为获得最大摆幅 M1 和 M7 刚好工作在饱和区边缘,不得不严格控制。如果用套筒式共源共栅OTA运放设计单位增益缓冲器,输出电压范围小于一个阈值电压。并且很难使得输入输出短接。

\qfig{f90.png}{单位增益缓冲器}

\subsection{折叠式运放}

需要两路电流,功耗比套筒式共源共栅大。
输出阻抗小,增益比套筒式共源共栅小,比套筒式低2~3倍。
输出摆幅比套筒式共源共栅大,比套筒式多1个过驱动电压。折叠式比套筒式的极点频率小。

\begin{example}
    Design a folded-cascode op amp with an NMOS input pair (Fig. 9.18) to satisfy the following specifications:
\(V_{D D}=3 V ,\) differential output swing \(=3 V ,\) power dissipation \(=10 mW ,\) and voltage gain \(=2000 \) . Assume that \(\mu_{n} C_{o x}=60 \mu A / V ^{2}, \mu_{p} C_{o x}=\) \(30 \mu A / V ^{2}, \lambda_{n}=0.1 V ^{-1}, \lambda_{p}=0.2 V ^{-1}\) (for an effective channel length of \(0.5 \mu m\) ), \(\gamma=0\), and \(V_{T H N}=\left|V_{T H P}\right|= 0.7 V\)
\end{example}

\begin{solution}
    同样从功耗与摆幅入手。分配输入对 \(1.5 mA\) ,共源共栅支路 \(1.5 mA\) ,剩余的电流给三个电流镜。
    
    摆幅决定了过驱动电压之和为 \(1.5 V\) ,对于过驱动电压,同样由于传导大电流,M5,6分配 \(0.5 V\),由于 PMOS 的迁移率较低,M3,4 分配\(0.4 V\) ,M7,8,9,10 分配\(0.3 V\) 。那么可以得到输出最高电压为 \(3 - 0.5 - 0.4 = 2.1 V\) ,输出的最低电压为 \(0.3 + 0.3 = 0.6 V\) ,那么最合适的共模电平为 \(1.35 V\)。

    对差分输入端进行电压分配,当输入输出短接有 \(V_{GS2} + V_{OD11} = 1.35 V\) ,初始假设 \(V_{OG11} = 0.4 V\) ,那么 \(V_{OD1,2} = 1.35 - 0.4 - 0.7 = 0.25 V\) 。

    目前所有的过驱动电压与电流已知,求解尺寸。M1,2:\(400\) ,M3,4:\(312.5\),M5,6: \(400\),M7,8,9,10: \(277.8\) , M11 \(312.5\) 。

    计算增益,等效跨导为 \[ G_m = g_{m1} \dfrac{r_{o1} \| r_{o5}}{r_{o1} \| r_{o5} + \dfrac{r_{o3}}{1+g_{m3} r_{o3}}}\]

    输出阻抗为 \[R_{out} = [(1 + g_{m7} r_{o7})r_{o9} + r_{o7}] \left|\right| [(1 + g_{m3} r_{o3})(r_{o5} \| r_{o1}) + r_{o3}]\]
    
    首先计算跨导,利用公式 \(g_m = 2 I_D / (V_{GS} - V_{TH})\) 。\(g_{m1,2} = 0.006 A/V\), \(g_{m3,4} = 0.00375 A/V\), \(g_{m5,6} = 0.006 A/V\), \(g_{m7,8,9,10} = 0.005 A/V\) 。计算沟道电阻 \(r_o = \dfrac{1}{\lambda I_D}\),\(r_{o1,7,9} = 13.3 k\Omega\) , \(r_{o5} = 3.33 k\Omega\) , \(r_{o3} = 6.67 k\Omega\) 。
    
    带入得到增益为 \(383\) 

    将 M1 扩充 3 倍,M5 扩充 5 倍, M3 扩充 1.8 倍,增益达到 2019 。


\end{solution}

\qfig{f91.png}{折叠式共源共栅}

\qfig{f92.png}{折叠式共源共栅-输入输出短接}

\subsection{共模反馈}

输出与输入短接电路,普通电阻做负载,输入/输出共模
电平比较稳定。

\qfig{f93.png}{输出与输入短接}


如果用电流源做负载,输入/输出共模电平不稳定,
对环境参数较为敏感。
如果管子参数与设计值稍有偏差,需要X和Y
点付出更大的偏差来平衡电流。
简单的差动反馈无法解决这个问题。
必须采用共模反馈技术来稳定输出共模电平。

\qfig{f94.png}{输出与输入短接-电流源负载}


\section{两级运放}

\subsection{反馈系统的稳定性}

\qfig{f96.png}{潜在的不稳定性}

反馈系统存在潜在的不稳定性,如果 \(\beta H\left(j \omega_{1}\right)=-1,\) 则 系统在 \(\omega_{1}\) 处发生振荡。 振荡时,环路的总相移为\(\degr{360}\),环路增益\(\geq 1\)。

环路增益的幅值为1的频率点称为 “增益交点(GX)”,环路增益的相位为-180度时的频率点称为 “相位交点(PX)”。PX必须发生在GX之后,确保:在-180度相位时,环路增
益已经小于1,不会形成振荡。

如果反馈减弱,之前描述的系统变得更稳定。

\subsubsection{单极点系统}
\[
    \begin{array}{l}
        H(s)=A_{0} /\left(1+s / \omega_{0}\right) \\
        \frac{Y}{X}(s)=\frac{\frac{A_{0}}{1+\beta A_{0}}}{1+\frac{s}{\omega_{0}\left(1+\beta A_{0}\right)}}
    \end{array}\]

相移始终小于 \(\degr{90}\),系统无条件稳定。

\subsubsection{双极点系统}

相位最终趋于-180度,因此,当环路增益的幅值
降至1时,相位尚未相移180度,系统稳定。反馈变弱,系统会更稳定。

\subsubsection{三极点系统}

三极点系统可能不稳定,尤其是PX点比GX点左移更明显。

PX点与GX点重合时,闭环增益的幅值无限大,从而导致
振荡。

相位裕度越大,PX与GX相隔越远。一般保证 \(\degr{60}\) 的相位裕度。

\[P M=180^{\circ}+\angle \beta H\left(\omega=\omega_{G X}\right)\]



\subsection{相位裕度}


GX低于PX不足以保持不出现近似的尖峰或者虽然稳定,但仍有可能产生减幅振荡。。PX与GX相隔足够远,系统更稳定,无减幅振荡。

\subsection{Miller OTA}

已知运放有两个极点和一个右半平面零点,设零点大于\(10 \omega_{GX}\),  为了使相位裕度大于 \(\degr{60}\),第二极点至少高于 \(2.2 \omega_{GX}\)

\qfig{f97.png}{密勒补偿电路}

加入补偿电容后,根据密勒定理:
与\(R_I\)并联的容值增大,该点对应的极点频率减小;
负反馈降低第二级的输出电阻,输出点对应的极点频率增大。拉远两个极点频率的距离,提高相位裕度。产生一个右半平面零点频率,会增加相移和增益,
使相位裕度恶化。应将零点频率推离原点。

\[
\begin{array}{l}
\frac{V_{o u t}}{V_{i n}}(s)=\frac{g_{m I} g_{m I I} R_{I} R_{I I}\left(1-s C_{c} / g_{m I I}\right)}{R_{I} R_{I I} \xi s^{2}+\left[g_{m I I} R_{I} R_{I I} C_{c}+R_{I I}\left(C_{I I}+C_{c}\right)+R_{I}\left(C_{I}+C_{c}\right)\right] s+1} \\
\end{array}
\]


其中
\[\xi=C_{I} C_{I I}+C_{I} C_{c}+C_{I I} C_{c}\]

主极点近似法:
\[
\begin{array}{l}
    \begin{aligned}
        
    \omega_{p 1} &\approx \frac{1}{g_{m I I} R_{I} R_{I I} C_{c}+R_{I I}\left(C_{I I}+C_{c}\right)+R_{I}\left(C_{I}+C_{c}\right)} \approx \frac{1}{g_{m I I} R_{I} R_{I I} C_{c}} \\
    \omega_{p 2} &\approx \frac{g_{m I I} C_{c}}{C_{I} C_{I I}+C_{I} C_{c}+C_{I I} C_{c}} \approx \frac{g_{m I I}}{C_{I I}}
    \end{aligned} 
\end{array}
\]

满足 \[ C_{I I} \gg C_{I}, \quad C_{c} \gg C_{I}\]

增益带宽积为 \[G B W=A_{v 0} \omega_{p 1} \approx g_{m I} R_{I} g_{m I I} R_{I I} \frac{1}{g_{m I I} R_{I} R_{I I} C_{c}}=\frac{g_{m I}}{C_{c}}\]

\qfig{f98.png}{密勒OTA}

通常
给定这些指标
直流增益 \(A_v\)
增益带宽积 \(GBW\)
输入共模范围 \(ICMR\)
负载电容 \(C_L\)
转换速率 \(SR\)
输出电压摆幅
功耗 \(P_{diss}\)

其中 \[A_{v1} = -g_{m1}(r_{o1} \| r_{o4})\]

\[A_{v2} = -g_{m6} (r_{o6 \| r_{o7}})\]

\[SR = \dfrac{I_5}{C_c}\]

\[GBW = \dfrac{g_{mI}}{C_c} = \dfrac{g_{m1}}{C_c} = \omega_{GX}\]

\[\omega_{p2} = \dfrac{g_{mII}}{C_{II}} = \dfrac{g_{m6}}{C_L}\]

\[\omega_z = \dfrac{g_{mII}}{C_c} = \dfrac{g_{m6}}{C_c}\]

输出共模下限 
\[V_{SS} + \sqrt{\dfrac{I_5}{\mu_n C_{ox} (W/L)_1}} + V_{TH1max} + V_{OD5}\]

上限为 \[\begin{array}{l}
    V_{D D}-\left|V_{S G 3}\right|+V_{T H 1_{-} \min } \\
    =V_{D D}-\left(\left|V_{S G 3}\right|-\left|V_{T H 3_{-} \max }\right|\right)-\left|V_{T H 3_{-} \max }\right|+V_{T H 1_{-} \min } \\
    =V_{D D}-\sqrt{\frac{I_{5}}{\mu_{p} C_{o x}(W / L)_{3}}}-\left|V_{T H 3_{3} \max }\right|+V_{T H_{1} \min }
    \end{array}\]

\begin{example}
    \(A_{v}>5000\)
\(G B W>5 M H z\)
输出电压摆幅 \(=\pm 2 V\)
\(V_{D D}=2.5 V\)
\(C_{L}=10 p F\)
\(I C M R=-1 \sim 2 V\)
\(V_{S S}=-2.5 V\)
\(S R>10 V / \mu s\)
\(P_{\text {diss}} \leq 2 m W\)
\(\mu_{n} C_{o x}=110 \mu A / V^{2}, \mu_{p} C_{o x}=50 \mu A / V^{2} \quad\left|V_{T H 3_{-} m a x}\right|=0.85 V\)
\(V_{T H 1} \min =0.55 V\)
\(\lambda_{n}=0.04 V^{-1}, \lambda_{p}=0.05 V^{-1} \quad V_{T H 1_{-} \max }=0.85 V\)
\(L=1 \mu m\)
\end{example}

\begin{solution}
    根据相位裕度要求 \(\omega_z > 10 \omega_{DX}\) 且 \(\omega_{p2} > 2.2 \omega_{GX}\)
    得到 \(\dfrac{g_{m6}}{C_c} > 10 \dfrac{g_{m2}}{C_c}\) 与 \(\dfrac{g_{m6}}{C_L} > 2.2 \dfrac{g_{m2}}{C_c}\) 
    化简 \(g_{m6} > 10 g_{m2}\) \(C_c > \dfrac{2.2 C_2}{10} = 0.22 C_L = 2.2 pF\)
    暂时选定为 \(C_c = 3 pF\)


    根据转换速率 \(SR = \dfrac{I_5}{C_c}\) 解得 \(I_5 = 30 \mu A\)

    根据共模上限确定 \((W/L)_{3,4} = 15\) \[2 V = V_{D D}-\sqrt{\frac{I_{5}}{\mu_{p} C_{o x}(W / L)_{3}}}-\left|V_{T H 3_{3} \max }\right|+V_{T H_{1} \min }\]

    根据增益带宽积确定 \(g_{m1}\) 
    \[g_{m1 = GBW  C_C = 94.25 \mu S}\]

    通过偏置电流计算 \((W/L)_{1,2}\)
    \[(W / L)_{1}=\frac{g_{m 1}^{2}}{2 \mu_{n} C_{o x}\left(I_{5} / 2\right)}=2.79 \approx 3.0=(W / L)_{2}\]

    通过共模下限确定 \(V_{OD5}\) 与 \((W/L)_5\)

    \[\begin{array}{l}
        V_{S S}+\sqrt{\frac{I_{5}}{\mu_{n} C_{o x}(W / L)_{1}}}+V_{T H 1_{-} \max }+V_{O D 5}=-1 V \Rightarrow V_{O D 5}=350 mV \\
        (W / L)_{5}=\frac{2 I_{5}}{\mu_{n} C_{o x} V_{O D 5}^{2}}=4.49 \approx 4.5
    \end{array}\]

    由相位裕度的要求 \(g_{m9} \geq 10 g_{m1,2} = 942.5 \mu S\)

    假设电流镜像确定 \(V_{SG4} = V_{SG6}\) 解得 \((W/L)_6 = (W/L)_4  \dfrac{g_{m6}}{g_{m4}} = 94.25\)
    其中 \(g_{m4}\) 可求。

    那么 \[I_6 = \dfrac{g_{m6}^2}{2 \mu_p C_{ox} (W/L)_6} = 94.5 \mu A\]

    再次通过镜像电流源确定 \((W/L)_7 = (W/L)_5 \dfrac{I_6}{I_5}\)

    检查M6和M7消耗的过驱动电压,符合摆幅要求

    \[V_{O D 6}=\sqrt{\frac{2 I_{6,7}}{\mu_{p} C_{o x}(W / L)_{6}}}=0.201 V \quad V_{O D 7}=\sqrt{\frac{2 I_{6,7}}{\mu_{n} C_{o x}(W / L)_{7}}}=0.351 V\]

    功耗为 \[P_{diss} = 5 V (30 \mu A + 95 \mu A) = 0.625 mW\]

    增益为 \(A_v = A_1 A_2 = 7696\)

\end{solution}




% End Here

\ifx\mainclass\undefined
\end{document}
\fi 