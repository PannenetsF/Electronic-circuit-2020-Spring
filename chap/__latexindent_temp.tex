\ifx\mainclass\undefined
\documentclass[cn,11pt,chinese,black,simple]{../elegantbook}
\usepackage{array}
\newcommand{\pp}[1]{\partial #1}
\newcommand{\dd}[1]{\mathrm{d}#1}
\newcommand{\abs}[1]{\left| #1 \right|}
\newcommand{\degr}[1]{#1^{\circ}}
\newcommand{\aint}{\int_{-\infty}^{+\infty} }
% \renewcommand{\dfrac}[2]{\dfrac{#1}{#2}}

\newcommand{\qfig}[2]{\begin{figure}[!htb]
    \centering
    \includegraphics[width=0.6\textwidth]{#1}
    \caption{#2}
\end{figure}}
\usepackage{circuitikz}

\renewcommand\arraystretch{1.5}
\begin{document}
\fi 

% Start Here
\chapter{噪声}


臊声限制了一个电路能够正确处理的最小信号电平, 现今的模拟电路没计者经常考虑
噪声的问題, 因为噪声与功耗、速度和线性度之间是互相制约的。

\section{噪声的统计特性}

噪声是一个随机过程。
在时域中的瞬时值是不可
预测的。
噪声的平均功率是可以
预测的。
允许我们用统计模型研究噪声。

\subsection{平均功率}

\(R_L\) 负载上消耗的平均功率为

$$\begin{array}{l}
    P_{a v}=\lim _{T \rightarrow \infty} \frac{1}{T} \int_{-T / 2}^{+T / 2} \frac{x^{2}(t)}{R_{L}} d t \\
    
\end{array}$$

简化之后

\[P_{a v}=\lim _{T \rightarrow \infty} \frac{1}{T} \int_{-T / 2}^{+T / 2} x^{2}(t) d t\]

实际功率同样可以由 \(\dfrac{P_{av}}{R_L}\) 得到。

\subsection{功率谱密度}

频谱描述的是噪声的频率分布,也叫功率谱密度(PSD),说明每个频率上信号的功率
噪声波形x(t)的功率谱密度 \(S_{X}(f)\) 定义为:
\(x(t)\) 在 \(f\) 附近的1Hz带宽中携带的平均功率。

与 \(P_{a v}\) 一样, \(S_{X}(f)\) 通常省略 \(R_{L},\) 用 \(V^{2} / H z\)表示,而不是\(W / H z_{\circ}\) 。
对 \(S_{X}(f)\) 求平方根也很常见,用 \(V / \sqrt{H z}\) 表示结果。
常见的噪声频谱的类型是“白噪声" 一在所有频率的值相同。
严格地说,白噪声并不存在,因为噪声所携带的总功率不可能
是无限的。
在一段频带内平坦的噪声频谱通常被称为白噪声。

在系统函数为 \(H(s)\) 的系统中 输入的噪声谱引 \(S_X(f)\) 起的输出噪声谱 \(S_Y(f)\) 满足 

\[S_Y(f) = S_X(f) \abs{H(f)}^2\] 

其中 \(H(f) = H(s)\left|_{s = j\omega = j 2 \pi f}\right.\)

对于实数信号,其噪声谱是频域的偶函数,因此可以分为双边谱与单边谱。

\subsection{噪声的概率分布}

虽然噪声的瞬时值是不可预测的,但是通过足够长时间的观
察噪声波形来构造幅值的“分布”是可能的。

中心极限定理指出,如果加入许多具有任意概率密度函数的
独立随机过程,总和的概率密度函数近似于高斯分布。

本课程中用频谱分析更多。

\section{噪声源特性}

\subsection{相干性}

对于给定的两种噪声 \(x_1(t)\) , \(x_2(t)\) 其噪声之和会体现两者的相干性。

\[\begin{aligned}
    P_{a v}=& \lim _{T \rightarrow \infty} \frac{1}{T} \int_{-T / 2}^{+T / 2}\left[x_{1}(t)+x_{2}(t)\right]^{2} d t \\
    =& \lim _{T \rightarrow \infty} \frac{1}{T} \int_{-T / 2}^{+T / 2} x_{1}^{2}(t) d t+\lim _{T \rightarrow \infty} \frac{1}{T} \int_{-T / 2}^{+T / 2} x_{2}^{2}(t) d t \\
    &+\lim _{T \rightarrow \infty} \frac{1}{T} \int_{-T / 2}^{+T / 2} 2 x_{1}(t) x_{2}(t) d t \\
    =& P_{a v 1}+P_{a v 2}+\lim _{T \rightarrow \infty} \frac{1}{T} \int_{-T / 2}^{+T / 2} 2 x_{1}(t) x_{2}(t) d t
\end{aligned}\]

独立器件产生的噪声波形是不相关的。

\subsection{信噪比}

信噪比 (SNR) 即信号与噪声噪声之比,被噪声干扰的信号,信噪比要足够高才能被解答。

\[SNR = \frac{P_{sig}}{P_{noise}}\]

根据噪声频谱密度 

\[P_{noise} = \aint S_{noise}(f) \dd{f} \]

频谱越宽,总的噪声功率越大,因此对于宽带放大器,特定频带的信号就会被整个频带的噪声破坏。所以电路的带宽必须限制在最小可接受值,以使总集成噪声功率
最小化。

\section{噪声分析}

整体流程可以分为记录器件的输入噪声频谱,分析到输出的传递函数,利用 \(S_Y(f) = S_X(f) \abs{H(f)}^2\) 分析输出噪声频谱,对输出求和得到总的输出频谱,对频率积分获得总噪声功率。

\subsection{电阻}

导体中电子的随机运动尽管平均电流为零, 但是它会引起导体两端电压的波动。电阻的热噪声可以用单边谱密度的串联电压源来模拟

\[S_v(f) = \bar{V_R^2 }= 4 k T R\]

并联电流源 

\[\bar{I_n^2} = \dfrac{4 k T}{R}\]

\(k\) 是玻尔兹曼常数。

\qfig{f49.png}{电阻噪声模型-电压源}

\qfig{f50.png}{电阻噪声模型-电流源}

\subsection{MOS 管}

\subsubsection{沟道噪声}

MOS 晶体管热噪声,主要是沟道噪声。对于\textbf{饱和}工作的\textbf{长沟道}MOS器件,沟道噪声可以通过一个连
接在漏极和源极之间的电流源来模拟

\[\bar{I_n ^2} = 4 k T \gamma g_m \]

此处的 \(\gamma\) 不是体效应系数, 对于长沟道晶体管是 \(\dfrac{2}{3}\) ,对于
亚微米MOS管更高。

\qfig{f51.png}{MOS 管噪声模型}

沟道电阻 \(r_o\) 不产生噪声,因为它不是一个实际的电阻。

\subsubsection{栅极电阻噪声}

对于宽晶体管,源极和漏极电阻可忽略不计,而栅极分布电阻较为显著。

\[\bar{V^2_{nR_G}} = 4 k T \dfrac{R_G}{3}\]

\qfig{f52.png}{栅极电阻噪声}

\qfig{f53.png}{栅极分布电阻噪声模型}

为了使得栅极电阻噪声可以忽略不计,其影响应远小于沟道噪声

\[4 k T \frac{R_{G}}{3}\left(g_{m} r_{O}\right)^{2} \ll\left(4 k T \gamma g_{m}\right) r_{O}^{2} \Rightarrow \frac{R_{G}}{3} \ll \frac{\gamma}{g_{m}}\]

上式通常可以成立。

\subsubsection{闪烁噪声}

在栅极氧化物和硅衬底之间的界面上出现许多“悬空”键,
产生额外的能态。
在界面上移动的载流子被随机捕获,随后被这种能量态释放,
在漏电流中引入“闪烁”噪声,模拟成与栅极串联的电压源,在饱和区大致为

\[\bar{V_n^2} = \dfrac{K}{C_{ox} WL} \dfrac{1}{f}\]

\(K\)是一个与工艺相关的常数,其数量级为 \(10^{-25} V^{2} F\)。可以看出噪声谱密度与频率成反比,即捕获和释放现象多发生在低频,也称为“ \(1/f\) ”噪声。 一般来说,PMOS器件比NMOS晶体管的 \(1/f\) 噪声小。



% End Here

\ifx\mainclass\undefined
\end{document}
\fi 