\ifx\mainclass\undefined
\documentclass[cn,11pt,chinese,black,simple]{../elegantbook}
\newcommand{\pp}[1]{\partial #1}
\newcommand{\dd}[1]{\mathrm{d}#1}
\newcommand{\abs}[1]{\left| #1 \right|}
\newcommand{\degr}[1]{#1^{\circ}}
\newcommand{\aint}{\int_{-\infty}^{+\infty} }
% \renewcommand{\dfrac}[2]{\dfrac{#1}{#2}}

\newcommand{\qfig}[2]{\begin{figure}[!htb]
    \centering
    \includegraphics[width=0.6\textwidth]{#1}
    \caption{#2}
\end{figure}}
\usepackage{circuitikz}

\renewcommand\arraystretch{1.5}
\begin{document}
\fi 

% Start Here
\renewcommand\arraystretch{1.5}

\chapter{MOS-FET基础}

\section{工作区间}

\begin{longtable}{llll}
    \caption{MOS管工作区间\protect\footnotemark} \\
    \toprule 区间 & \(U_{GS}\) & \(U_{GD}\) & \(I_D\) \\
    \midrule
    \endfirsthead
    
    \toprule 区间 & \(U_{GS}\) & \(U_{GD}\) & \(I_D\) \\
    \midrule
    \endhead

    \hline
    \multicolumn{4}{c}{见下页}\\   
    \bottomrule
    \endfoot

    \bottomrule
    \endlastfoot

    \footnotetext{\(V_{TH} = V_{TH,0} + \gamma (\sqrt{2 \Phi_F + V_{SB}} - \sqrt{2 \Phi_F})\)}
    截止 & \(< V_{TH}\) &  & \(0\)  \\
    线性 & \(> V_{TH}\) & \(\geq V_{TH}\) & \( \mu_n C_{ox} \frac{W}{L} \left[(V_{GS}-V_{TH})V_{DS}-\frac{1}{2}V_{DS}^2\right] \) \\
    饱和 & \(> V_{TH}\) & \(\leq V_{TH}\) & \(\frac{1}{2}\mu_n C_{ox} \frac{W}{L} \left[(V_{GS}-V_{TH})V_{DS}-\frac{1}{2}V_{DS}^2\right](1+\lambda V_{DS})\)
    
\end{longtable}

\section{放大器基础}

\subsection{辅助定理}

在线性电路中,电压增益满足\[A_v = - G_m R_{out}\] ,其中 \(G_m\) 为输出接地的跨导, \(R_{out}\) 为输入置零时的输出电阻。


\subsection{阻抗变换}

对于放大器常用阻抗变换来计算某一端的等效电阻,图中是代表进行偏置的 MOS 管。

\qfig{f3.png}{阻抗变换}



对于漏极:\[
R_{eq} = (1 + (g_m + g_{mb})r_o)R_s + r_o = (1 + (g_m + g_{mb})R_s)r_o + R_s
\]

对于源极:
\[
R_{eq} = \frac{R_D + r_o}{1 + (g_m + g_{mb}) r_o}    
\]

特殊的,当无负载时,对于漏极:\[
R_{eq} = r_o
\]

对于源极:
\[
R_{eq} = r_o \| \frac{1}{g_m} \| \frac{1}{g_{mb}} 
\]

\subsection{分流原理}

可以看作是负载电阻和沟道电阻对电流源进行分流,那么 

\[I_{out} = \frac{r_o}{R_s + R_o} g_m V_{in}\]


\begin{figure}[htb]
    \centering
    \begin{circuitikz}
        \draw (2,4) node[ground]{} node[above]{Out} to (1,4) 
        node[currarrow, xscale=-1]{} node[above]{$I_{out}$}
        to (0,4) to[R={$R_s$}] (0,2) 
            to[cI={\  }, l=$g_m V_{in}$, -]
            (0,0) node[ground](GND){};
        \draw (0,2) node[circ]{}  to (2,2) 
        to[R=$r_o$] (2,0) 
        node[ground]{};
    \end{circuitikz}
    \caption{基本分流结构}
\end{figure}

\section{典型放大器}


\begin{longtable}{ll}
    \caption{MOS单级放大器} \\
    \toprule \textbf{指标} & \textbf{特点} \\
    \midrule
    \endfirsthead
    
    \toprule \textbf{指标} & \textbf{特点} \\
    \midrule
    \endhead

    \hline
    \multicolumn{2}{c}{见下页}\\   
    \bottomrule
    \endfoot

    \bottomrule
    \endlastfoot
    共源级         & 输入电阻大,反向放大,有增益,适合基本放大。 \\
    源随器         & 增益近似为1,输出电阻小,适合做缓冲级。   \\
    共栅级         & 有输入电阻,而且较低,能接收电流。      \\
    共源共栅级       & 超高的增益,但牺牲摆幅。    
    
\end{longtable}
% End Here

\ifx\mainclass\undefined
\end{document}
\fi 