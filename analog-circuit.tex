\documentclass[cn,11pt,chinese,black,simple]{elegantbook}

\def\mainclass{main}

\title{电子电路}
% \subtitle{数字设计初步}

\author{Pannenets.F}
% \institute{微电子学院}
\date{\today}
% \version{4}
\bioinfo{分类}{笔记}

\extrainfo{Je reviendrai et je serai des millions. —— <<Spartacus>>}
\setcounter{tocdepth}{4}
\newcommand{\dollar}{\mbox{\textdollar}}
\lstset{
  mathescape = false}
% \logo{logo-blue.png}
\cover{covernew.jpg}

% 本文档命令
\usepackage{array}
\newcommand{\ccr}[1]{\makecell{{\color{#1}\rule{1cm}{1cm}}}}
\newcommand{\pp}[1]{\partial #1}
\newcommand{\dd}[1]{\mathrm{d}#1}
\newcommand{\abs}[1]{\left| #1 \right|}
\newcommand{\degr}[1]{#1^{\circ}}
\newcommand{\aint}{\int_{-\infty}^{+\infty} }
% \renewcommand{\dfrac}[2]{\dfrac{#1}{#2}}

\newcommand{\qfig}[2]{\begin{figure}[!htb]
    \centering
    \includegraphics[width=0.6\textwidth]{#1}
    \caption{#2}
\end{figure}}
\usepackage{circuitikz}

\renewcommand\arraystretch{1.5}

\begin{document}

\maketitle
\frontmatter

\chapter*{特别声明}
\markboth{Introduction}{前言}

本手册记录了\textbf{电子电路 2020 春}的部分知识点,主要是依据考试范围进行整理。

本笔记采用了 Elegant \LaTeX{} 项目的 \href{https://github.com/ElegantLaTeX/ElegantBook}{Elegant Book} 模板,封面等图片来自无版权图片网站,全部插图来自课程的课件。受限于老师的课件版权问题,这部分没有上传。除此之外为了迅速笔记,源文件中采取的许多不是很美观的方式进行书写。如果能对以后的同学提供帮助的话不胜荣幸,同时对于文中的错误欢迎不吝赐教,进行修订。

\vskip 1.5cm


\begin{flushright}
Pannenets.F
\today
\end{flushright}

\tableofcontents
%\listofchanges

\mainmatter

\part{BJT基础}

\ifx\mainclass\undefined
\documentclass[cn,11pt,chinese,black,simple]{../elegantbook}
\usepackage{array}
\newcommand{\ccr}[1]{\makecell{{\color{#1}\rule{1cm}{1cm}}}}
\newcommand{\qfig}[2]{\begin{figure}[!htb]
    \centering
    \includegraphics[width=0.6\textwidth]{#1}
    \caption{#2}
  \end{figure}}
\begin{document}
\fi 

% Start Here
\chapter{BJT放大电路}

\section{工作区间}

\begin{longtable}{llll}
    \caption{三极管工作区间} \\
    \toprule 区间 & \(U_{BE}\) & \(U_{CE}\) & \(I_C\) \\
    \midrule
    \endfirsthead
    
    \toprule 区间 & \(U_{BE}\) & \(U_CE\) & \(I_C\) \\
    \midrule
    \endhead

    \hline
    \multicolumn{4}{c}{见下页}\\   
    \bottomrule
    \endfoot

    \bottomrule
    \endlastfoot

    截止 & \(< U_{on}\) & \(> U_{BE}\) & \(\leq I_{CEO}\)  \\
    放大 & \(\geq U_{on}\) & \(\geq U_{BE}\) & \(\beta I_{B}\) \\
    饱和 & \(\geq U_{in}\) & \(\leq U_{BE}\) & \(< \beta I_B\)
    
\end{longtable}

\section{直流通路}

直流通路静态工作点的求解过程:

\qfig{f1.png}{一个BJT放大电路}

\begin{solution}
    \begin{enumerate}
        \item 确定基极电势,由于基极电流极小,将之视为开路: \[U_{BB} = \frac{R_{b1}}{R_{b1}+R_{b2} V_{CC}}\]
        \item 确定基极电流,对供电电路进行戴维南等效:\[U_{eq} = U_{BB}\] \[R_{eq} = R_{b1} \| R_{b2}\]
        
        利用基尔霍夫电压定律列方程: \[U_{eq} = I_{BQ}  R_{eq} + U_{BEQ} + I_{BQ}  (1 + \beta)  R_e\]
        其中\(U_{BEQ}\)是基极与射极间固定的电压降
        \item 确定集电极射极的电势差:\[U_{CEQ} = V_{CC} - R_c  \beta I_{BQ} - R_e  (1 + \beta) I_{BQ}\]
        或者 \[U_{CEQ} \approx V_{CC} - R_c * \beta I_{BQ} - R_e   \beta I_{BQ}\] 
    \end{enumerate}
\end{solution}

\qfig{f2.png}{戴维南等效}

\section{交流通路}

交流通路求解较为简单,只需注意一点

\[r_{be} = r_{bb'} + \frac{U_T}{I_{BQ}}\]


\section{三种基本放大器}


\begin{longtable}{llll}
    \caption{BJT放大器} \\
    \toprule \textbf{指标} & \textbf{共射} & \textbf{共集} & \textbf{共基} \\
    \midrule
    \endfirsthead
    
    \toprule \textbf{指标} & \textbf{共射} & \textbf{共集} & \textbf{共基} \\
    \midrule
    \endhead

    \hline
    \multicolumn{4}{c}{见下页}\\   
    \bottomrule
    \endfoot

    \bottomrule
    \endlastfoot

    $A_i$       & 大           & 大                                                         & 小           \\
    $A_v$       & 大           & 小                                                         & 大           \\
    $R_i$       & 中           & 大                                                         & 小           \\
    $R_o$       & 大           & 小                                                         & 大           \\
    \hline 用途          & 低频电压放大器     & \begin{tabular}[c]{@{}l@{}}输入级、\\ 输出级、\\ 缓冲器\end{tabular} & 宽频带放大器     

    
\end{longtable}

\section{失真分析}

失真的发生代表放大器从正常工作的放大区,滑向了截至区或者饱和区。

以共射为例,
Q 点过低,小信号的波动使得放大器发生截止失真;Q 点过高,小信号的波动使得放大器发生饱和失真。又共射放大器为反向放大器,失真波形分别位于顶部与底部。




% End Here

\ifx\mainclass\undefined
\end{document}
\fi 

\part{MOS 放大器}

\ifx\mainclass\undefined
\documentclass[cn,11pt,chinese,black,simple]{../elegantbook}
\newcommand{\pp}[1]{\partial #1}
\newcommand{\dd}[1]{\mathrm{d}#1}
\newcommand{\abs}[1]{\left| #1 \right|}
\newcommand{\degr}[1]{#1^{\circ}}
\newcommand{\aint}{\int_{-\infty}^{+\infty} }
% \renewcommand{\dfrac}[2]{\dfrac{#1}{#2}}

\newcommand{\qfig}[2]{\begin{figure}[!htb]
    \centering
    \includegraphics[width=0.6\textwidth]{#1}
    \caption{#2}
\end{figure}}
\usepackage{circuitikz}

\renewcommand\arraystretch{1.5}
\begin{document}
\fi 

% Start Here
\renewcommand\arraystretch{1.5}

\chapter{MOS-FET基础}

\section{工作区间}

\begin{longtable}{llll}
    \caption{MOS管工作区间\protect\footnotemark} \\
    \toprule 区间 & \(U_{GS}\) & \(U_{GD}\) & \(I_D\) \\
    \midrule
    \endfirsthead
    
    \toprule 区间 & \(U_{GS}\) & \(U_{GD}\) & \(I_D\) \\
    \midrule
    \endhead

    \hline
    \multicolumn{4}{c}{见下页}\\   
    \bottomrule
    \endfoot

    \bottomrule
    \endlastfoot

    \footnotetext{\(V_{TH} = V_{TH,0} + \gamma (\sqrt{2 \Phi_F + V_{SB}} - \sqrt{2 \Phi_F})\)}
    截止 & \(< V_{TH}\) &  & \(0\)  \\
    线性 & \(> V_{TH}\) & \(\geq V_{TH}\) & \( \mu_n C_{ox} \frac{W}{L} \left[(V_{GS}-V_{TH})V_{DS}-\frac{1}{2}V_{DS}^2\right] \) \\
    饱和 & \(> V_{TH}\) & \(\leq V_{TH}\) & \(\frac{1}{2}\mu_n C_{ox} \frac{W}{L} \left[(V_{GS}-V_{TH})V_{DS}-\frac{1}{2}V_{DS}^2\right](1+\lambda V_{DS})\)
    
\end{longtable}

\section{放大器基础}

\subsection{辅助定理}

在线性电路中,电压增益满足\[A_v = - G_m R_{out}\] ,其中 \(G_m\) 为输出接地的跨导, \(R_{out}\) 为输入置零时的输出电阻。


\subsection{阻抗变换}

对于放大器常用阻抗变换来计算某一端的等效电阻,图中是代表进行偏置的 MOS 管。

\qfig{f3.png}{阻抗变换}



对于漏极:\[
R_{eq} = (1 + (g_m + g_{mb})r_o)R_s + r_o = (1 + (g_m + g_{mb})R_s)r_o + R_s
\]

对于源极:
\[
R_{eq} = \frac{R_D + r_o}{1 + (g_m + g_{mb}) r_o}    
\]

特殊的,当无负载时,对于漏极:\[
R_{eq} = r_o
\]

对于源极:
\[
R_{eq} = r_o \| \frac{1}{g_m} \| \frac{1}{g_{mb}} 
\]

\subsection{分流原理}

可以看作是负载电阻和沟道电阻对电流源进行分流,那么 

\[I_{out} = \frac{r_o}{R_s + R_o} g_m V_{in}\]


\begin{figure}[htb]
    \centering
    \begin{circuitikz}
        \draw (2,4) node[ground]{} node[above]{Out} to (1,4) 
        node[currarrow, xscale=-1]{} node[above]{$I_{out}$}
        to (0,4) to[R={$R_s$}] (0,2) 
            to[cI={\  }, l=$g_m V_{in}$, -]
            (0,0) node[ground](GND){};
        \draw (0,2) node[circ]{}  to (2,2) 
        to[R=$r_o$] (2,0) 
        node[ground]{};
    \end{circuitikz}
    \caption{基本分流结构}
\end{figure}

\section{典型放大器}


\begin{longtable}{ll}
    \caption{MOS单级放大器} \\
    \toprule \textbf{指标} & \textbf{特点} \\
    \midrule
    \endfirsthead
    
    \toprule \textbf{指标} & \textbf{特点} \\
    \midrule
    \endhead

    \hline
    \multicolumn{2}{c}{见下页}\\   
    \bottomrule
    \endfoot

    \bottomrule
    \endlastfoot
    共源级         & 输入电阻大,反向放大,有增益,适合基本放大。 \\
    源随器         & 增益近似为1,输出电阻小,适合做缓冲级。   \\
    共栅级         & 有输入电阻,而且较低,能接收电流。      \\
    共源共栅级       & 超高的增益,但牺牲摆幅。    
    
\end{longtable}
% End Here

\ifx\mainclass\undefined
\end{document}
\fi 

\ifx\mainclass\undefined
\documentclass[cn,11pt,chinese,black,simple]{../elegantbook}
\usepackage{array}
\newcommand{\ccr}[1]{\makecell{{\color{#1}\rule{1cm}{1cm}}}}
\newcommand{\pp}[1]{\partial #1}
\newcommand{\dd}[1]{\mathrm{d}#1}
\newcommand{\abs}[1]{\left| #1 \right|}
\newcommand{\degr}[1]{#1^{\circ}}
\newcommand{\aint}{\int_{-\infty}^{+\infty} }
% \renewcommand{\dfrac}[2]{\dfrac{#1}{#2}}

\newcommand{\qfig}[2]{\begin{figure}[!htb]
    \centering
    \includegraphics[width=0.6\textwidth]{#1}
    \caption{#2}
\end{figure}}
\usepackage{circuitikz}

\renewcommand\arraystretch{1.5}
\begin{document}
\fi 

% Start Here
\chapter{差动放大器}



\section{基本概念}

\textbf{共模电平}:差动信号的中间电平。

\section{直流分析}

\subsection{定量电压电流分析}

处于饱和区的晶体管,从输入端到电流源的过驱动电压满足:

\[V_{override} = V_{GS} - V_{TH} = \sqrt{\frac{2 I_D}{\mu_n C_{ox}(W/L)}}\]

并且我们还要建立两路输入电压的不同引起的输出电压不同,为此,从电流入手:

\[V_{in1}-V_{in2} = \sqrt{\frac{2 I_{D1}}{\mu_n C_{ox} (W/L)}} - \sqrt{\frac{2 I_{D2}}{\mu_n C_{ox} (W/L)}}\]

最终化简为

\begin{equation}\label{eq:diffvdiifi}
    \begin{aligned}
        \Delta I_D &= I_{D1} - I_{D2} \\
        &= \frac{1}{2} \mu_n C_{ox} \frac{W}{L} (V_{in1} - V_{in2}) \sqrt{\frac{4 I_{SS}}{\mu_n C_{ox} (W/L)} - (V_{in1} - V_{in2})^2}\\
        &= \frac{1}{2} \mu_n C_{ox} \frac{W}{L} (\Delta V_{in}) \sqrt{\frac{4 I_{SS}}{\mu_n C_{ox} (W/L)} - (\Delta V_{in})^2}
    \end{aligned}
\end{equation}


\qfig{f4.png}{理想电流源的基本差动对}

\subsection{差动信号增益}

根据式\ref{eq:diffvdiifi}求斜率表达式

\begin{equation}
    \begin{aligned}
        G_m = \frac{\pp \Delta I_D}{\pp \Delta V_{in}} = \frac{1}{2} \mu_n C_{ox} \frac{W}{L} \displaystyle\frac{\displaystyle\frac{4 I_{SS}}{\mu_n C_{ox} W/L} - 2\Delta V_{in}^2}{\sqrt{\displaystyle\frac{4 I_{SS}}{\mu_n C_{ox} W/L}-\Delta V_{in}^2}}
    \end{aligned}
\end{equation}


根据其图像,差动信号在差动信号直流电平相等时,可以得到最大的小信号增益。在非理想电流源中,直流电位差别较大的时候,存在电流完全被一路抽取的情况。

并且在 \(\Delta V_{in} = \sqrt{2 I_{SS} / (\mu_n C_{ox} W/L)}\) 增益下降为 \(0\) 。


\subsection{共模电平选取}

差动放大器的共模电平应保证放大管以及电流源均工作在饱和区,以获得最大且最稳定的小信号增益。

\qfig{f5.png}{非理想电流源的基本差动对}

\section{交流分析}

\subsection{叠加原理}

差动输入分别考虑为两个,通过线性电路狄叠加原理分别得到 \((V_{o1}-V_{o2}) = K V_{in1}\) , \((V_{o2}-V_{o1}) = K V_{in2} = K (- V_{in1})\) 之后

\[\frac{(V_{o1}-V_{o2})_{total}}{V_{in1}-V_{in2}} = K \]

\subsection{半边电路法}

电路完全对称,D1和D2代表任何三端有源器件。 假设 \(Vin1\) 和
\(Vin2\) 差动变化,如果电路保持线性,则 \(V_p\) 不变,在交流通路中接地。

\qfig{f13.png}{辅助定理}




\section{共模分析}

研究共模 (Common Mode) 响应的原因:
实际电路中共模信号的位置会影响整个电路的静态工作点,进而改变小信号的增益与电路摆幅。并且电路往往不会完全对称,共模输入也会自然地导致差动地输出。

差动对的共模响应取决于\textbf{尾电流源的输出阻抗}和\textbf{电路的不对称性},并表现为两个方面的影响: 对称电路的输出共模电平变化以及输入
共模电压变化在输出端产生差模分量。在模拟电路
中, 后者的影响要比前者严重得多。有鉴于此, 研究
共模响应时通常应当考虑不匹配。

响应可以通过 MOS 管并联求解,并联视为宽度加倍。增益为

\[A_{v,CM} = - \frac{R_D/2}{1/(2 g_m)+R_SS}\]

\qfig{f7.png}{共模响应- MOS 并联}

\section{失配分析}

电路的不对称既来自负载电阻也来自输入晶体管,通常后者产生的失配要大
得多。

仅考虑晶体管失配,即\(g_{m1}\) ,\(g_{m2}\) 不相等。即\textbf{共模输入}由于电路不对称引起的\textbf{差动输出},增益为

\[A_{CM-DM} = \frac{\Delta V_X - \Delta V_Y}{\Delta V_{in,CM}} = - \frac{(g_{m1}-g_{m2})R_D}{1+(g_{m1}+g_{m2})R_{SS}}\]


\qfig{f8.png}{失配分析的电路样例}

\section{共模响应总结}

无失配情况:尾电流源的有限阻抗引起共模电平变化。

失配情况:共模信号会产生差动输出。

高频情况:尾电流源的寄生电容阻抗的降低会引起共模电平的剧烈变化。

\qfig{f9.png}{高频情况下尾电流源的寄生电容}

\section{差动放大器评估}

为了合理地比较各种差动电路, 由共模变化而产生的不期望的差动成分必须用放大后所
需要的差动输出归一化。定义共模抑制比 CMRR (common-mode rejection ratio) ,可以理解为放大效果被失配稀释的程度。

\[CMRR = \left|\frac{A_{DM}}{A_{CM-DM}} \right|\]




\section{典型差动对}

\subsection{带源极负反馈的差动对}

增加了线性放大区输入的范围,但是源极负载消耗电压裕度,摆幅减小。

\qfig{f6.png}{带源极负反馈的差动对}

\subsection{带有负载的差动对}

差动对的负载往往是二极管接法或电流源接法的 MOS 管。

二极管接法:

\begin{equation*}
    \begin{aligned}
        A_v &= - g_{m1} \left(r_{o1} \| r_{o3} \| \frac{1}{g_{m3}}\right)
        \\& \approx - \frac{g_{m1}}{g_{m3}} = - \sqrt{\frac{\mu_n(W/L)_N}{\mu_p(W/L)_P}}
    \end{aligned}
\end{equation*}

电流源接法:

\[A_v = -g_{m,N} (r_{o,N} \| r_{o,P}) \]

\qfig{f10.png}{二极管接法的差动对负载}


\qfig{f11.png}{电流源接法的差动对负载}


\subsection{一道典型增益的求解}

在这里,可以将横跨电阻 \(R_1\) 看作是两个 \( 0.5R_1\) 的串联,便于将其使用半边电路对称处理。在直流条件下,两个 \( 0.5R_1\) 连接点的电平是固定的,因此在半边电路小信号中, 将之当作大小为 \( 0.5R_1\) 的接地电阻即可。

由辅助定理,尾电流源接地。


\qfig{f12.png}{带有横跨电阻的差动放大器}

\subsection{一道典型共模抑制比的求解}

求解本题目的方法具有普适性。

\begin{exercise}
    Due to a manufacturing defect, a large parasitic resistance has appeared between the drain and source terminals of M1. Assuming \(\lambda\) = \(\gamma\) = 0, calculate the small-signal gain, common-mode gain, and CMRR. 
    \qfig{f14}{题图}
\end{exercise}

\begin{solution}
    设电流源的漏极电平为 \(V_X\) ,输出端两端分别为 \(V_{out,1}\), \(V_{out,2}\) 。

    由于无沟道调制与体效应,小信号中总电流为 \(0\),即输出端两端反向,那么 \(V_{out,1} = - V_{out,2} = \displaystyle\frac{1}{2} V_o\) 。

    求解 \(A_{CM-DM}\) 时, 设\(V_{in,1} = V_{in,2} = V_{in}\)。

    求解 \(A_{DM}\) 时,设\(V_{in,1} = - V_{in,2} = \displaystyle \frac{1}{2} V_{in}\)。

    联立两条支路的基尔霍夫方程即可,不再赘述。
\end{solution}




% End Here

\ifx\mainclass\undefined
\end{document}
\fi 

\ifx\mainclass\undefined
\documentclass[cn,11pt,chinese,black,simple]{../elegantbook}
\usepackage{array}
\newcommand{\pp}[1]{\partial #1}
\newcommand{\dd}[1]{\mathrm{d}#1}
\newcommand{\abs}[1]{\left| #1 \right|}
\newcommand{\degr}[1]{#1^{\circ}}
\newcommand{\aint}{\int_{-\infty}^{+\infty} }
% \renewcommand{\dfrac}[2]{\dfrac{#1}{#2}}

\newcommand{\qfig}[2]{\begin{figure}[!htb]
    \centering
    \includegraphics[width=0.6\textwidth]{#1}
    \caption{#2}
\end{figure}}
\usepackage{circuitikz}

\renewcommand\arraystretch{1.5}
\begin{document}
\fi 

% Start Here

\chapter{无源与有源电流镜}

电流镜在模拟电路中可以作为电路的偏置单元提供标准的电流,也可以通过有源负载形成所谓的五管 OTA ,对差动信号进行处理,转换为单端信号。

\section{基本电流镜}

以一个通过电阻分压调控的电流源为例,其 \(\mu_n , V_{TH}\) 等受温度影响很大,并且过驱动电压的小波动就会引起很大的电流输出偏差。因此使用电压控制无法获得准确的电流偏置。

\qfig{f15.png}{电阻分压电流源}

因此模拟电路中采用的普遍方法是对\textbf{基准电流}进行复制,将基准电流偏置的 MOS 管的电压进行复制,即可获得“镜像”,也就是电流镜。

\qfig{f16.png}{基本电流镜}

对于基本电流镜满足

\begin{equation}\label{eq:c4:1}
    \begin{aligned}
        I_{REF} &= \frac{1}{2} \mu_n C_{OX} \left(\frac{W}{L}\right)_1(V_{GS} - V_{TH})^2 \\
        I_{out} &= \frac{1}{2} \mu_n C_{OX} \left(\frac{W}{L}\right)_2(V_{GS} - V_{TH})^2 \\
        \frac{I_{out}}{I_{REF}} &= \frac{(W/L)_2}{(W/L)_1}
    \end{aligned}
\end{equation}


这样电流的复制就不受工艺与温度的影响,仅由器件尺寸的比值决定。需要注意的是,尺寸的调节不可以调节沟道长度, 因为源漏区的扩散效应影响 \(L\) ,需要通过晶体管并联的方式调节栅极的宽度。

电流镜作为直流偏置的器件可以处理大信号,同时也可以处理小信号。

\section{共源共栅电流镜}

到目前为止, 我们有关电流镜的讨论中, 都忽略了沟道长
度调制。在实际中, 这一效应使得镜像的电流产生了极大的误差, 尤其是当使用最小长度晶体管以便通过减小宽度来减小电流源输出电容时。

由图~\ref{eq:c4:1}加入沟道调制效应之后

\begin{equation}\label{eq:c4:2}
    \begin{aligned}
        I_{REF} &= \frac{1}{2} \mu_n C_{OX} \left(\frac{W}{L}\right)_1(V_{GS} - V_{TH})^2 (1 + \lambda V_{DS1}) \\
        I_{out} &= \frac{1}{2} \mu_n C_{OX} \left(\frac{W}{L}\right)_2(V_{GS} - V_{TH})^2 (1 + \lambda V_{DS2})  \\
        \frac{I_{out}}{I_{REF}} &= \frac{(W/L)_2 (1 + \lambda V_{DS2})}{(W/L)_1 (1 + \lambda V_{DS1})} 
    \end{aligned}
\end{equation}

为了减少沟道调制效应的影响,采用共源共栅电流源,在保持\(V_X \approx V_Y\)的前提下,两路电流可以认为满足电流镜的条件,在体效应情况下仍成立,尺寸需要满足

\[\frac{(W/L)_3}{(W/L)_0} = \frac{(W/L)_2}{(W/L)_0}\]

值得注意的是精度的提升的代价是 \(M_3\) 的一个过驱动电压,P 点的最低电平提高了。并且底部晶体管受到 \(V_P\) 的影响被屏蔽了,\(\Delta V_Y\) 满足

\[\Delta V_Y = \Delta V_P \frac{1}{(g_{m3} + g_{mb3}) r_{o3}}\]

\qfig{f29.png}{共源共栅电流源}

\qfig{f17.png}{共源共栅电流镜}


\section{低压共源共栅电流镜}

\qfig{f18.png}{低压共源共栅电流镜}

为了保证 \(M_0, M_1\) 工作在饱和区。

\[M_0: V_b - V_{TH0} \leq V_X\]

\[M_1: V_{GS1} - V_{TH1} \leq V_A = V_b - V_{GS0} \]

联立得到

\[V_{GS0} - V_{TH0} \leq V_{TH1}\]

实际上就是保证 \(M_0\) 的过驱动电压远低于 \(V_{TH1}\),实际使用时调节 \(V_b\) 使得\(M_1, M_2\) 工作在线性区边缘可以消耗最小的电压裕度。

通过两种方式确定偏置的 \(V_b\)

\begin{equation}
    \begin{aligned}
        (a) V_b &= V_{GS5} + V_{GS6) - R_b I_1} \\
        (b) V_b &= V_{GS6} - V_{TH6} + V_{GS7}
    \end{aligned}
\end{equation}

\qfig{f19.png}{两种电压偏置}


\section{五管 OTA (Operational Transconductance Amplifier) }

五管 OTA (运算跨导放大器)是指的有源电流镜做负载的差动对,可以将差动信号转换为单端输出。

\subsection{无源电流镜负载}

\qfig{f20.png}{无源电流镜负载}

按照辅助定理求解,输出短路时电路完全对称。

\[G_m =  \frac{g_m}{2}\] 
\[R_{out} = r_{o4} \| (r_{o1} + r_{o2})\]

因此

\[\left|A_v\right| = \frac{g_m}{2} \left(2 r_{o1} \| r_[o4]\right)\]

通过分析发现,\(M_1\) 管子的电流直接流向了交流地,没有起到任何放大作用,被浪费,因此为了利用这股电流增加增益,使用五管 OTA 。

\subsection{有源电流镜负载}

\qfig{f21.png}{有源电流镜负载}

\subsubsection{大信号分析}

输出电平与输入电平差值的关系可以看出大信号在近似共模时可以获得高增益。输出应保证\(M_2\) 饱和, 那么 \[V_{OUT} \geq V_{in,CM} - V_{TH}\] ,摆幅与输入共模电平直接相关。

\qfig{f22.png}{输出电平与输入电平差值的关系}

\subsubsection{小信号分析}

严格求解不可以使用半边电路法,因为电路不完全对称,\(F, X\) 摆幅差异很大,\(P\) 不能视为虚地。

\qfig{f23.png}{OTA 的不对称摆幅}

但是在实际近似求解\textbf{跨导}时,可以近似。

由于 \(r_{o1} \gg \left[(1/g_{m3}) \| r_{o3}\right]\) ,\(r_{o1}\) 几乎没有电流经过,可以视为断路,而由电路的对称性知道,漏极此时接地,那么 \(P\) 此时必定接地。

\qfig{f24.png}{等效跨导}

此时 \(M_4\) 复制了 \(M+3\) 的电流 
\[I_{out} = - I_3 + I_2 = -g_{m1} V_{in}\]


\qfig{f25.png}{等效输出电阻}
 
求解输出电阻,此时只有 \(M_4\) 为受控管子,并且受控线上没有电流。

\[R_{out} = \left[(2 r_{o1,2}) + (\frac{1}{g_{m3}} \| r_{o3} ) \right] \| \dfrac{\dfrac{1}{g_{m3}} \| r_{o3}}{(2 r_{o1,2}) + (\frac{1}{g_{m3}} \| r_{o3} ) } g_{m_4} \| r_{o4}   \] 

近似表示为
\[R_{out} \approx r_{o2} \| r_{o4}\]

那么 \[ |A_v| =  g_{m1} (r_{o1} \| r_{o4}) \frac{2 g_{m4} r_{o4} + 1}{2 (g_{m4} r_{o4} + 1)} \approx g_{m1} (r_{o1} \| r_{o4}) \]

\qfig{f26.png}{等效电路}

\subsection{OTA 评估}

OTA 的电压裕度消耗较大,可以使用电流抽取方式使得漏极电位提高。

\qfig{f27.png}{抽取电流设置}

共模增益的简化求解:\[A_{CM} \approx - \frac{\dfrac{1}{}2 g_{m3,4} || \dfrac{r_{o3,4}}{2}}{\dfrac{1}{2 g_{m1,2} +R_{SS} }} = -\frac{1}{1 + 2 g_{m1,2} R_{SS}} \frac{g_{m1,2}}{g_{m3,4}}\]

\qfig{f28.png}{简化的共模电路}

那么共模抑制比为 

\[CMRR = \abs{\frac{A_{DM}}{A_{CM}}} = g_{m3,4} (1 + 2 g_{m1,2} R_{SS})(r_{o1,2} || r_{o3,4})\]

即使电路无失配,OTA也存在有限的共模抑制比,这是其缺点。

五管OTA的非差分输出的结构导致电源噪声抑制能力较差, \(V_F\)几乎以单位增益
随 \(V_{DD}\)变化。

\section{一道典型例题}

题目求解电流源的漏极电压变化引起的电流变化,往往是通过设出未知数联立一侧的电流进行求解。

通过标准电流等条件求解出左侧各点的电势,设 B 点的电压由于栅极变化 \(\Delta V\) 

\[I_{out} = \frac{1}{2} k_n (W/L)_2 (V_x - V_{TH})^2 (1 + \lambda (V_B + \Delta V)) \]

\[I_{out} = \frac{1}{2} k_n (W/L)_3 (V_b - (V_B + \Delta V) - V_{TH})^2 (1 + \lambda (V_P + \Delta V_P - (V_B) + \Delta V))\]

仅有 \(\Delta V\) 一个未知数,完成求解。

\qfig{f30.png}{共源共栅电流源做负载}

% End Here

\ifx\mainclass\undefined
\end{document}
\fi 

\part{深入理解放大器}

\ifx\mainclass\undefined
\documentclass[cn,11pt,chinese,black,simple]{../elegantbook}
\usepackage{array}

\newcommand{\pp}[1]{\partial #1}
\newcommand{\dd}[1]{\mathrm{d}#1}
\newcommand{\abs}[1]{\left| #1 \right|}
\newcommand{\degr}[1]{#1^{\circ}}
\newcommand{\aint}{\int_{-\infty}^{+\infty} }
% \renewcommand{\dfrac}[2]{\dfrac{#1}{#2}}

\newcommand{\qfig}[2]{\begin{figure}[!htb]
    \centering
    \includegraphics[width=0.6\textwidth]{#1}
    \caption{#2}
\end{figure}}
\usepackage{circuitikz}

\renewcommand\arraystretch{1.5}
\begin{document}
\fi 

% Start Here
\chapter{频率响应}

本章研究的核心问题是放大器对不同频率的信号的放大效果,电容的存在使得增益等指标成为频率的函数,这就是\textbf{频率响应}。


\qfig{f48.png}{MOS 管完整等效电路}

\section{波特图}

横轴使用对数刻度 \(\lg f\) 或者 \(\lg \omega\),幅频纵轴为 \(20 \lg \abs{A_v} \) 单位为 \(dB\) ,相频纵轴仍为角度。并且曲线折线化,容忍误差。

以低通电路为例,有

\[\dot{U}_o = \frac{1 / (j \omega C)}{R + (1 + j \omega C)} U_i = \frac{1}{1 + j \omega R C} \dot{U}_i\]

那么 \[\abs{A_v} = \sqrt{\dfrac{1}{1 + \omega^2R^2C^2}} = \sqrt{\dfrac{1}{1 + (f / f_p)^2}}\] 
\[\varphi = -\arctan(\omega RC) = -\arctan(\dfrac{f}{f_p}) \] 

折线化的来源 



\begin{equation*}
    20 \lg \abs{A_v} \approx \left\{
    \begin{aligned}
        0, &f \ll f_p\\
        -20\lg \dfrac{f}{f_p}, &f \gg f_p
    \end{aligned}
    \right.
\end{equation*}


\begin{equation*}
    -\arctan \frac{f}{f_p} \approx \left\{
    \begin{aligned}
        &\degr{0}, &f \ll 0.1 f_p\\
        &\degr{-45} - \lg \frac{f}{f_p}, &0.1  f_p <f < 10 f_p\\
        &\degr{-90}, &f \gg 10 f_p
    \end{aligned}
    \right.
\end{equation*}

波特图的方式幅频特性的横轴均用对数刻度,拓宽视野;纵轴用对数,将乘除运算换算为加减运算,方便处理级联等问题;误差较小,近似程度可以接受。

本课程中采用拉普拉斯变换,需满足以下\textbf{标准形式}

\[A_v = K \frac{1 + \dfrac{s}{\omega_z}}{\left(1 + \dfrac{s}{\omega_{p1}}\right)\left(1 + \dfrac{s}{\omega_{p2}}\right)}\]

转化为实频域


\[A_v = K \frac{1 + \dfrac{j \omega}{\omega_z}}{\left(1 + \dfrac{j \omega}{\omega_{p1}}\right)\left(1 + \dfrac{j \omega}{\omega_{p2}}\right)}\]

之后转化为对数形式的加减法

\[20 \lg \abs{A_v} = 20 \lg K + 20 \lg \sqrt{1 + \left(\dfrac{j \omega}{\omega_z}\right)^2} - 20 \lg \sqrt{1 + \left(\dfrac{j \omega}{\omega_{p1}}\right)^2} - 20 \lg \sqrt{1 + \left(\dfrac{j \omega}{\omega_{p2}}\right)^2}\] 

\[\varphi = \arctan\dfrac{\omega}{\omega_z} - \arctan\dfrac{\omega}{\omega_{p1}} - \arctan\dfrac{\omega}{\omega_{p2}}  \]


\begin{definition}[频响零极点]
    定义以上的标准形式中的 \(\omega_z\) 为零点频率,\(\omega_p\) 为极点频率
\end{definition}

每经过一个极点,增益下降的速度增加 20 dB/10倍频,
相位在\(\pm 10\)倍频的范围内下降。

每经过一个零点,增益上升的速度增加 20 dB/10倍频,
相位在\(\pm 10\)倍频的范围内上升 \(\degr{90}\) 。

\section{密勒定理}

\begin{theorem}[密勒定理]
    如果一个电路可以完成如图~\ref{fig:ch5:1} 的变换,那么有
    \[\frac{V_X-V_Y}{Z} = \frac{V_X}{Z_1} \Rightarrow Z_1 = \frac{Z}{1 - \dfrac{V_Y}{V_X}} = \frac{Z}{1 - A_v}\]

    \[\frac{V_Y-V_X}{Z} = \frac{V_Y}{Z_2} \Rightarrow Z_2 = \frac{Z}{1 - \dfrac{V_X}{V_Y}} = \frac{Z}{1 - A_v^{-1}}\]
\end{theorem}

\begin{figure}[htb]
    \centering
    \includegraphics[width=0.6\textwidth]{f31.png}
    \caption{密勒等效变换}
    \label{fig:ch5:1}
\end{figure}

但是遗憾的是,密勒定理没有给出成立条件,在单信号通路时,通常不成立,阻抗与信号主通路并联时,通常成立。

\qfig{f32.png}{阻抗与主信号通路并联}

\section{密勒近似的评估}

严格求解各个频率下的系统特性使代数式变得十分复杂。然而, 在许多情况下, 我们采用低频的放大性能便能深人了解
电路的特性。密勒近似就是用低频的增益替代实际的增益,来近似了解电路的特性。实际上密勒近似的结果与实际存在一定的偏差,但是极大简化了运算过程。


\section{结点与极点}

放大器进行级联时,对于每一级的输入输出关系都有 

\[V_{out} = V_{in} \frac{\dfrac{1}{sC}}{R + \dfrac{1}{sC}} A_v = \frac{A_v}{1 + R s C}V_{in} \]

那么对整个放大器链有

\[\frac{V_{out}}{V_{in}}(s) =  \frac{A_1}{1 + R_S s C_{in}} \frac{A_2}{1 + R_1 s C_N}  \frac{1}{1 + R_2 s C_P} \]

那么可以说\textbf{一个结点贡献一个极点},极点频率为 \(\dfrac{1}{RC}\)虽在存在耦合时计算会十分复杂且不满足这个结论,但是仍然可以进行一定的估算,并且极大简化问题。值得注意的是,在实际使用时每一级放大器后面跟随的是其输出电阻,而不一定需要保持串联的形式,常见的输出电阻如沟道电阻就是和输出并联的。



\qfig{f33.png}{放大器的级联}

\qfig{f34.png}{一种耦合的例子}


\section{特征频率}

MOS 管的特征频率指的是源极漏极接地时,输入电流等于输出电流时的频率。
输入电流 \(I_{in} = I_G = \dfrac{V_{in}}{\dfrac{1}{s (C_{GS} + C_{GD})}}\) 。
输出电流 \(I_{out} = g_m V_{in}\) 。

解得 \[f_T = \frac{g_m}{2 \pi (C_{GS} + C_{GD})}\]

\qfig{f35.png}{MOS 小信号电路}


\section{共源级频响}

\subsection{密勒近似}

增益为 \(A_v = - g_m R_D\) 

输入极点 \(\omega_{in} = \frac{1}{R_S [C_{GS} + (1 - A_v) C_{GD}]}\)

输出极点 \(\omega_{out} = \frac{1}{R_D [C_{DB} + (1 - A_v^{-1}) C_{GD}]}\)

估算传输函数 \[\frac{V_{out}}{V_{in}}(s) = \frac{-g_m R_D}{(1 + \dfrac{s}{\omega_{in}}) (1 + \dfrac{s}{\omega_{out}})}\]

\qfig{f36.png}{共源级密勒近似}

输入阻抗

\[Z_{in} = \frac{1}{s C_{GS}} \| \frac{1/ (s C_{GD})}{1 + g_m R_D}\]

\subsection{精确解}

回路方程的解

\[\frac{V_{out}}{V_{in}}(s) = \frac{(s C_{GD} - g_m)R_D}{R_S R_D \xi^2 + [R_S(1 + g_m R_D) C_{GD} + R_S C_{GS} +R_D(C_{GD} + C_{DB} )]s + 1}\]

\[\xi = C_{DS} C_{GD} + C_{GS} C_{DB} + C_{GD} C_{DB}\]

采用\textbf{主极点近似法},当分母很复杂时,靠直觉假定,两个极点相距很远。

分母改写为 

\[D = (1 + \frac{s}{\omega_{p1}})(1 + \frac{s}{\omega_{p2}})\]

极点间隔很大,一次项系数为 \(\dfrac{1}{\omega_{p1}}\) 。再通过二次项系数解出另一个极点。

零点为 \(s_z = \dfrac{g_m}{C_{GD}}\)

增益在高频区间(零点之后)有反弹 \(C_{GD}\) 提供前馈通路,传导高频信号到输出端。

\qfig{f37.png}{共源级频响波特图}

输入阻抗

\[Z_{in} = \frac{1 + R_D (C_{GD} + C_{DB })s}{s C_{GD} (1 + g_m R_D + s C_{DB} R_D)} \| \frac{1}{s C_{GS}}\]



\subsection{对比评估}

估算无法估出零点,估算得到的极点也不准确。

主极点:

\[\omega_{p1} = \frac{1}{R_S(1 + g_m R_D) C_{GD} + R_S C_{GS} + R_D(C_{GD} + C_{DB})}\]

\[\omega_{in} = \frac{1}{R_S(1 + g_m R_D) C_{GD} + R_S C_{GS}}\]

只要 \( R_D(C_{GD} + C_{DB})\) 足够小,误差就很小。

次极点

\[\omega_{p2} = \frac{R_S(1 + g_m R_D)C_{GD} + R_S C_{DS} + R_D(C_{GD} + C_{DB})}{R_S R_D (C_{GS} C_{GD} + C_{GS} C_{DB} + C_{GD} C_{DB})}\]

\[\omega_{out} = \frac{1}{R_D(C_{GD} + C_{DB})}\]

只要 \(C_{GD}\) 够大,误差就很小。

在低频情况下,输入阻抗无穷大,均符合预期。

\section{源随器频响}

\subsection{精确解}

X点与Y点的相互作用极强,不宜用密勒近似。实际上,若 \(\lambda = \gamma = 0\),则X到Y的增益为 \(1\)。 密勒近似出现分母为 \(0\) 的情况。

\qfig{f38.png}{源随器等效电路}

\[\frac{V_{out}}{V_{in}} = \frac{s C_{DS} + g_m}{R_S (C_L C_{DS} + C_L C_{GD} + C_{GS} C_{GD})s^2 + (g_m R_S C_{GD} + C_L + C_{GS}) s + g_m }\]

主极点近似

\[\omega_{p1} = \frac{g_m}{g_m R_S C_{GD} + C_L +C_{DS}}\]

\[\omega_{p2} = \frac{1}{\omega_{p1}} \frac{g_m}{R_S (C_L C_{GS} + C_L C_{GD} + C_{GS} C_{GD})}\]

若 \(R_S = 0\) 那么输出极点消失,主极点近似得到的是精确解。

\[s_z = -\frac{-g_m}{C_{GS}}\]

\subsection{输入阻抗}

\[Z_{in} = (\frac{1}{s C_{DS}} + (1 + \frac{g_m}{s C_{GS}})\frac{1}{g_{mb} + s C_L}) \| \frac{1}{s C_{GD}} \]

若 \(C_L = 0\) 且 \(g_{mb} = 0\) 那么 \(C_{DS}\)  被自举,输出端感受不到输入的变化,输入阻抗无穷大。

\subsection{输出阻抗}

\[Z_{out} = \frac{1 + s R_s C_{GS}}{g_m + s C_{GS}} \| \frac{1}{s (C_L + C_{SB}) } = R_2 + R_1 \| (s L)\]

(仅第一部分)低频近似为 \(\frac{1}{g_m}\) 

(仅第一部分)高频近似为 \(R_S\)

其中  \(R_2 = \dfrac{1}{g_m}\) \(R_1 + R_2 = R_S\)

\[L = \frac{C_{GS}}{g_m} (R_S - \frac{1}{g_m})\]


在较高频率下,呈现负电阻效应。输出阻抗包含电
感成分。

\section{共栅级频响}

\subsection{无沟道调制效应}

此时晶体管两端无阻抗耦合,无需使用密勒定理。

\[\omega_{in} = \dfrac{1}{C_S (R_S \| \dfrac{1}{g_m + g_{mb}})}\]

\[\omega_{out} = \frac{1}{C_D R_D}\]

增益为 

\[\frac{V_{out}}{V_{in}}(s) = \frac{(g_m + g_{mb}) R_D}{1 + (g_m + g_{mb} R_S)} \dfrac{1}{(1 + \dfrac{s}{\omega_{in}}) (1 + \dfrac{s}{\omega_{out}})} \]

电容无密勒乘积项,可达到宽带。输入阻抗较低,级联时,容易引起电压信号损失。

输入阻抗 
\[Z_{in} = R_S + (\frac{1}{g_m +g_{mb}}) \| \frac{1}{s C_s}\]

\qfig{f39.png}{共栅级等效电路 \(\lambda = 0\)}

\subsection{沟道调制效应}

为简化计算,负载电阻用电流源代替。

\[\frac{V_{out}}{V_{in}} = \frac{1 + g_m r_o}{r_o C_L C_{in} R_s s^2 + [r_o C_l + R_S C_{in} + R_S C_L (1 + g_m r_o)]s + 1}\]

\[Z_{in} =  R_S +  \frac{1/(s C_L) + r_o}{1 + g_m r_o} \| \frac{1}{s C_{in}}\]

输入阻抗较低,且随频率升高而降低。

\qfig{f40.png}{共栅级等效电路 \(\lambda \neq 0\)}

\subsection{栅极电阻}

从源极看过去的阻抗为 \[Z = \frac{1 + s C_{GS}R_G}{g_m + s C_{GS}}\]

易知 \(I_{out}\) 和右侧的电流成正比,那么存在一个系数 \(\dfrac{Z}{Z + R_s}\)

那么 \[\omega_p = \frac{1+ g_m R_S}{(R_G + R_S)C_{GS}}\]

为获得宽带,应减小栅极电阻。

\qfig{f41.png}{共栅级等效电路 栅极电阻)}

\section{共源共栅级频响}

\subsection{极点}

输入的密勒电容是 \[C_X = (1 + \frac{g_{m1}}{g_{m2} + g_{mb2}})C_{GD1} + C_{DB1} + C_{SB2} + C_{GS2}\]

那么 \(A\) 节点 \[\omega_{p,A} = \frac{1}{R_S\left[C_{GS1}+(1 + \dfrac{g_{m1}}{g_{m2} + g_{mb2}})C_{GD1}\right]}\]

\(X\) 节点

\[\omega_{p.X} = \frac{g_{m2} + g_{mb2}}{2 C_{GD1} + C_{DB1} + C_{SB2} + C_{GS2}}\]


\(Y\) 节点无密勒乘积项

\[\omega_{p,Y} = \frac{1}{R_D (C_{DB2} + C_L + C_{GD2})}\]

一般情况下选取 \(\omega_{p,X}\) 离原点最远。

\qfig{f42.png}{共源共栅级等效电路}

\subsection{简化模型:电流源负载}

\[\frac{V_{out}}{I_{in}} = -\frac{1 + g_{m2} r_{o2}}{s C_X} \frac{1}{1 + (1 + g_{m2} r_{o2} ) \dfrac{C_Y}{C_X}+ s C_Y r_{o2} } \approx - \frac{g_[m2]}{s C_X} \frac{1}{g_{m2} \dfrac{C_Y}{C_X} + s C_Y}  \]

那么结点 \(X\) 的极点为 \(\dfrac{g_m2}{C_X}\) 与之前的结果一致。

暂时忽略 \(C_Y\) ,
输出阻抗为 \[Z_{out} = r_{o2} + (1 + g_{m2} r_{02}) \left(r_{o1} \| \dfrac{1}{s C_X}\right)\]

在高频时,输出阻抗会下降,即使忽略 \(C_Y\) 。


\qfig{f43.png}{简化模型}

\section{差动对频响}

对于差分信号,其频率响应与共源放大器一致(半边电路法)

\[V_{DM-OUT}(s) = \frac{V_{in}(s)}{2} H(s) - \left[-\frac{V_{in}(s)}{2}\right] H(s) = V_{in} (s) H(s)\]

\qfig{f44.png}{差动对等效电路}

共模增益带入之前的增益即可 

\begin{equation*}
    \begin{aligned}
        A_{CM-DM} &= \frac{\Delta g_m (R_D \| \dfrac{1}{s C_L}) }{1 + (g_{m1} + g_{m2})\left(r_{o3} \| \dfrac{1}{s C_P}\right)} \\
        &= \frac{\Delta g_m R_D}{1 + (g_{m1} + g_{m2})r_{o3}} \frac{1 + s C_P r_{o3}}{(1 + s R_D C_L)\left[1 + s \dfrac{C_P r_{o3}}{1 + (g_{m1} + g_{m2})r_{o3}}\right]}
    \end{aligned}
\end{equation*}

共模抑制比同样带入

\begin{equation*}
    \begin{aligned}
        CMRR &= \frac{g_{m1} + g_{m2} + 4 g_{m1} g_{m2} (r_{o3} \| \dfrac{1}{s C_P})}{2 \Delta g_m} \\
        &\approx \frac{g_m}{\Delta g_m} \frac{1 + 2 g_m r_{o3} + s C_P r_{o3}}{1 + s C_P r_[o3]} \\
        &\approx \frac{2 g_m^2 r_{o3}}{\Delta g_m } \frac{1 + s \dfrac{C_P}{2 g_m}}{1 + s C_[ r_{o3}]} 
    \end{aligned}
\end{equation*}

共模抑制比在高频下降,模抑制比的零点实际是 \(A_{Dm}\) 的零点


\qfig{f45.png}{共模等效电路}

\section{OTA 频响}

\qfig{f46.png}{OTA 等效电路}

系统函数为

\[\frac{g_{m N} r_{O N}\left(2 g_{m} p+C_{E} s\right) r_{O P}}{2 r_{O P} r_{O N} C_{E} C_{L} s^{2}+\left[\left(2 r_{O N}+r_{O P}\right) C_{E}+r_{O P}\left(1+2 g_{m P} r_{O N}\right) C_{L}\right] s+2 g_{m P}\left(r_{O N}+r_{O P}\right)}\]

主极点为 

\[\omega_{p 1} \approx \frac{2 g_{m P}\left(r_{O N}+r_{O P}\right)}{\left(2 r_{O N}+r_{O P}\right) C_{E}+r_{O P}\left(1+2 g_{m} P r_{O N}\right) C_{L}}\]

忽略第一项分母并且假定增益极大 \(2 g_{m} p r_{O N} \gg 1\)

\[\omega_{p 1} \approx \frac{1}{\left(r_{O N} \| r_{O P}\right) C_{L}}\]

那么标准形式为 
\[\begin{aligned}
    \frac{V_{o u t}}{V_{i n}} &=\frac{A_{0}}{1+s / \omega_{p 1}}\left(\frac{1}{1+s / \omega_{p 2}}+1\right) \\
    &=\frac{A_{0}\left(2+s / \omega_{p 2}\right)}{\left(1+s / \omega_{p 1}\right)\left(1+s / \omega_{p 2}\right)}
    \end{aligned}\]

\section{增益带宽积}

增益与带宽通常存在折中。
通常用增益带宽积(GBW)来综合考量。

\subsection{单极点电路}

极点和 \(-3 dB\) 带宽可以认为相等。

\[GBW = \abs{A_{v0}} \omega_{p,1}\]

\subsection{多极点电路}



对于两级结构有 

\[\frac{V_{o u t}}{V_{i n}}=\frac{A_{0}^{2}}{\left(1+\frac{s}{\omega_{p}}\right)^{2}}\]

即 

\[\frac{A_{0}^{2}}{1+\frac{\omega_{-3 d B}^{2}}{\omega_{p}^{2}}}=\frac{A_{0}^{2}}{\sqrt{2}}\]

解得

\[\begin{aligned}
    \omega_{-3 d B} &=\sqrt{\sqrt{2}-1 \omega_{p}} \\
    & \approx 0.64 \omega_{p}
    \end{aligned}\]

因此

\[\mathrm{GBW}=\sqrt{\sqrt{2}-1} A_{0}^{2} \omega_{p}\]

对于 N 级联使 \(-3 dB\) 带宽减小,使增益带宽积增大

\[\omega_{-3 d B}=\sqrt{\sqrt[N]{2}-1 \omega_{p}}\]

\[\mathrm{GBW} = \omega_{-3 d B} A_0^N\]



\qfig{f47.png}{多级共源共栅结构}

% End Here

\ifx\mainclass\undefined
\end{document}
\fi 

\ifx\mainclass\undefined
\documentclass[cn,11pt,chinese,black,simple]{../elegantbook}
\usepackage{array}
\newcommand{\pp}[1]{\partial #1}
\newcommand{\dd}[1]{\mathrm{d}#1}
\newcommand{\abs}[1]{\left| #1 \right|}
\newcommand{\degr}[1]{#1^{\circ}}
\newcommand{\aint}{\int_{-\infty}^{+\infty} }
% \renewcommand{\dfrac}[2]{\dfrac{#1}{#2}}

\newcommand{\qfig}[2]{\begin{figure}[!htb]
    \centering
    \includegraphics[width=0.6\textwidth]{#1}
    \caption{#2}
\end{figure}}
\usepackage{circuitikz}

\renewcommand\arraystretch{1.5}
\begin{document}
\fi 

% Start Here
\chapter{噪声}


臊声限制了一个电路能够正确处理的最小信号电平, 现今的模拟电路没计者经常考虑
噪声的问題, 因为噪声与功耗、速度和线性度之间是互相制约的。

\section{噪声的统计特性}

噪声是一个随机过程。
在时域中的瞬时值是不可
预测的。
噪声的平均功率是可以
预测的。
允许我们用统计模型研究噪声。

\subsection{平均功率}

\(R_L\) 负载上消耗的平均功率为

$$\begin{array}{l}
    P_{a v}=\lim _{T \rightarrow \infty} \dfrac{1}{T} \int_{-T / 2}^{+T / 2} \dfrac{x^{2}(t)}{R_{L}} d t \\
    
\end{array}$$

简化之后

\[P_{a v}=\lim _{T \rightarrow \infty} \dfrac{1}{T} \int_{-T / 2}^{+T / 2} x^{2}(t) d t\]

实际功率同样可以由 \(\dfrac{P_{av}}{R_L}\) 得到。

\subsection{功率谱密度}

频谱描述的是噪声的频率分布,也叫功率谱密度(PSD),说明每个频率上信号的功率
噪声波形x(t)的功率谱密度 \(S_{X}(f)\) 定义为:
\(x(t)\) 在 \(f\) 附近的1Hz带宽中携带的平均功率。

与 \(P_{a v}\) 一样, \(S_{X}(f)\) 通常省略 \(R_{L},\) 用 \(V^{2} / H z\)表示,而不是\(W / H z_{\circ}\) 。
对 \(S_{X}(f)\) 求平方根也很常见,用 \(V / \sqrt{H z}\) 表示结果。
常见的噪声频谱的类型是“白噪声" 一在所有频率的值相同。
严格地说,白噪声并不存在,因为噪声所携带的总功率不可能
是无限的。
在一段频带内平坦的噪声频谱通常被称为白噪声。

在系统函数为 \(H(s)\) 的系统中 输入的噪声谱引 \(S_X(f)\) 起的输出噪声谱 \(S_Y(f)\) 满足 

\[S_Y(f) = S_X(f) \abs{H(f)}^2\] 

其中 \(H(f) = H(s)\left|_{s = j\omega = j 2 \pi f}\right.\)

对于实数信号,其噪声谱是频域的偶函数,因此可以分为双边谱与单边谱。

\subsection{噪声的概率分布}

虽然噪声的瞬时值是不可预测的,但是通过足够长时间的观
察噪声波形来构造幅值的“分布”是可能的。

中心极限定理指出,如果加入许多具有任意概率密度函数的
独立随机过程,总和的概率密度函数近似于高斯分布。

本课程中用频谱分析更多。

\section{噪声源特性}

\subsection{相干性}

对于给定的两种噪声 \(x_1(t)\) , \(x_2(t)\) 其噪声之和会体现两者的相干性。

\[\begin{aligned}
    P_{a v}=& \lim _{T \rightarrow \infty} \dfrac{1}{T} \int_{-T / 2}^{+T / 2}\left[x_{1}(t)+x_{2}(t)\right]^{2} d t \\
    =& \lim _{T \rightarrow \infty} \dfrac{1}{T} \int_{-T / 2}^{+T / 2} x_{1}^{2}(t) d t+\lim _{T \rightarrow \infty} \dfrac{1}{T} \int_{-T / 2}^{+T / 2} x_{2}^{2}(t) d t \\
    &+\lim _{T \rightarrow \infty} \dfrac{1}{T} \int_{-T / 2}^{+T / 2} 2 x_{1}(t) x_{2}(t) d t \\
    =& P_{a v 1}+P_{a v 2}+\lim _{T \rightarrow \infty} \dfrac{1}{T} \int_{-T / 2}^{+T / 2} 2 x_{1}(t) x_{2}(t) d t
\end{aligned}\]

独立器件产生的噪声波形是不相关的。

\subsection{信噪比}

信噪比 (SNR) 即信号与噪声噪声之比,被噪声干扰的信号,信噪比要足够高才能被解答。

\[SNR = \dfrac{P_{sig}}{P_{noise}}\]

根据噪声频谱密度 

\[P_{noise} = \aint S_{noise}(f) \dd{f} \]

频谱越宽,总的噪声功率越大,因此对于宽带放大器,特定频带的信号就会被整个频带的噪声破坏。所以电路的带宽必须限制在最小可接受值,以使总集成噪声功率
最小化。

\section{器件噪声分析}

整体流程可以分为记录器件的输入噪声频谱,分析到输出的传递函数,利用 \(S_Y(f) = S_X(f) \abs{H(f)}^2\) 分析输出噪声频谱,对输出求和得到总的输出频谱,对频率积分获得总噪声功率。

\subsection{电阻}

导体中电子的随机运动尽管平均电流为零, 但是它会引起导体两端电压的波动。电阻的热噪声可以用单边谱密度的串联电压源来模拟

\[S_v(f) = \overline{V_R^2 }= 4 k T R\]

并联电流源 

\[\overline{I_n^2} = \dfrac{4 k T}{R}\]

\(k\) 是玻尔兹曼常数。

\qfig{f49.png}{电阻噪声模型-电压源}

\qfig{f50.png}{电阻噪声模型-电流源}

\subsection{MOS 管}

\subsubsection{沟道噪声}

MOS 晶体管热噪声,主要是沟道噪声。对于\textbf{饱和}工作的\textbf{长沟道}MOS器件,沟道噪声可以通过一个连
接在漏极和源极之间的电流源来模拟

\[\overline{I_n ^2} = 4 k T \gamma g_m \]

此处的 \(\gamma\) 不是体效应系数, 对于长沟道晶体管是 \(\dfrac{2}{3}\) ,对于
亚微米MOS管更高。

\qfig{f51.png}{MOS 管噪声模型}

沟道电阻 \(r_o\) 不产生噪声,因为它不是一个实际的电阻。

\subsubsection{栅极电阻噪声}

对于宽晶体管,源极和漏极电阻可忽略不计,而栅极分布电阻较为显著。

\[\overline{V^2_{nR_G}} = 4 k T \dfrac{R_G}{3}\]

\qfig{f52.png}{栅极电阻噪声}

\qfig{f53.png}{栅极分布电阻噪声模型}

为了使得栅极电阻噪声可以忽略不计,其影响应远小于沟道噪声

\[4 k T \dfrac{R_{G}}{3}\left(g_{m} r_{O}\right)^{2} \ll\left(4 k T \gamma g_{m}\right) r_{O}^{2} \Rightarrow \dfrac{R_{G}}{3} \ll \dfrac{\gamma}{g_{m}}\]

上式通常可以成立。

\subsubsection{闪烁噪声}

在栅极氧化物和硅衬底之间的界面上出现许多“悬空”键,
产生额外的能态。
在界面上移动的载流子被随机捕获,随后被这种能量态释放,
在漏电流中引入“闪烁”噪声,模拟成与栅极串联的电压源,在饱和区大致为

\[\overline{V_n^2} = \dfrac{K}{C_{ox} WL} \dfrac{1}{f}\]

\(K\)是一个与工艺相关的常数,其数量级为 \(10^{-25} V^{2} F\)。可以看出噪声谱密度与频率成反比,即捕获和释放现象多发生在低频,也称为“ \(1/f\) ”噪声。 一般来说,PMOS器件比NMOS晶体管的 \(1/f\) 噪声小。

在低频,闪烁噪声功率接近无穷大,但是这个周期会相当的长,这么慢的速度无法和热漂移以及老化区分。

沟道热噪声与闪烁噪声谱密度的交点称为转角频率

\[\begin{array}{c}
    \begin{aligned}
        
    4 k T \gamma g_{m}&=\dfrac{K}{C_{o x} W L} \cdot \dfrac{1}{f_{C}} \cdot g_{m}^{2} \\
    f_{C}&=\dfrac{K}{\gamma C_{o x} W L} g_{m} \dfrac{1}{4 k T}
    \end{aligned}
\end{array}\]

闪烁噪声可以通过栅极的跨导变换转换为一个并联在源漏之间的电流源,并且栅极接地。

\section{电路噪声分析}

为了计算出输出噪声,将输入设为零,并根据电路中所
有的噪声源计算出输出的总噪声。

\subsection{输入参考噪声}

对噪声进行归一化,研究在特定增益下的噪声。

\[\overline{V_{n,out}^2} = A_v^2 \overline{V_{n,in}^2}\]

等效电路如图

\qfig{f54.png}{噪声处理的等效电路}

\subsection{有限输入阻抗的参考噪声}

在存在有限的输入阻抗时,使用单一电压源模拟会出现输出噪声随阻抗上升而下降的现象。理论可以证明再引入一个电流源是必须的,并且可以表示任何线性二端口网络的噪声。对于同样的输出噪声,分别计算输入参考电压与电流即可。

当输入阻抗很小时,输入参考噪声电流\(\overline{I_{n,in}}\)的作用才比较重要。

\qfig{f55.png}{完备的噪声模型}

\subsection{辅助定理}

低频时等效的条件

\[\overline{V_n^2} = \dfrac{\overline{I_n^2}}{g_m^2}\]

\qfig{f56.png}{等效电路}

\section{单级放大器的噪声}

\subsection{共源级}


\subsubsection{简单共源级}



\qfig{f57.png}{简单共源级}


\[\begin{aligned}
    \overline{V_{n, i n}^{2}} &=\dfrac{\overline{V_{n, o u t}^{2}}}{A_{v}^{2}} \\
    &=\left(4 k T \gamma g_{m}+\dfrac{K}{C_{o x} W L} \cdot \dfrac{1}{f} \cdot g_{m}^{2}+\dfrac{4 k T}{R_{D}}\right) R_{D}^{2} \dfrac{1}{g_{m}^{2} R_{D}^{2}} \\
    &=4 k T \dfrac{\gamma}{g_{m}}+\dfrac{K}{C_{o x} W L} \cdot \dfrac{1}{f}+\dfrac{4 k T}{g_{m}^{2} R_{D}}
\end{aligned}\]

\[\overline{I_{n,in}^2} = (4 k T \dfrac{\gamma}{g_{m}}+\dfrac{K}{C_{o x} W L} \cdot \dfrac{1}{f}+\dfrac{4 k T}{g_{m}^{2} R_{D}}) g_m^2\]

若是作为放大器,使得输出噪声最小,\(g_m\) 尽量大;若是作为电流源,\(g_m\) 尽量小。

\subsubsection{电流源负载的共源级热噪声}

\[\overline{V_{n,  {out}}^{2}}=4 k T\left(\gamma g_{m 1}+\gamma g_{m 2}\right)\left(r_{O 1} \| r_{O 2}\right)^{2}\]

那么为了作为放大器,\(g_{m1}\) 最大化, \(g_{m2}\) 最小化。

\[\begin{aligned}
    \overline{V_{n, in}^{2}} &=4 k T\left(\gamma g_{m 1}+\gamma g_{m 2}\right) \dfrac{1}{g_{m 1}^{2}} \\
    &=4 k T \gamma\left(\dfrac{1}{g_{m 1}}+\dfrac{g_{m 2}}{g_{m 1}^{2}}\right)
\end{aligned}\]

总输出噪声的功率为

\[
    \begin{aligned}
\overline{V_{n, \text {out}, \text {tot}}^{2}}&=\int_{0}^{\infty} 4 k T \gamma\left(g_{m 1}+g_{m 2}\right)\left(r_{O 1} \| r_{O 2}\right)^{2} \dfrac{d f}{1+\left(r_{O 1} \| r_{O 2}\right)^{2} C_{L}^{2}(2 \pi f)^{2}}\\
 &= \gamma\left(g_{m 1}+g_{m 2}\right)\left(r_{O 1} \| r_{O 2}\right) \dfrac{k T}{C_{L}}
    \end{aligned}
\]

\qfig{f58.png}{电流源负载的共源级热噪声}

当输入的信号摆幅为 \(V_m\) 时, 信噪比为

\[\begin{aligned}
    SNR _{\text {out }} &=\left[\dfrac{g_{m 1}\left(r_{O 1} \| r_{O 2}\right) V_{m}}{\sqrt{2}}\right]^{2} \cdot \dfrac{1}{\gamma\left(g_{m 1}+g_{m 2}\right)\left(r_{O 1} \| r_{O 2}\right)\left(k T / C_{L}\right)} \\
    &=\dfrac{C_{L}}{2 \gamma k T} \cdot \dfrac{g_{m 1}^{2}\left(r_{O 1} \| r_{O 2}\right)}{g_{m 1}+g_{m 2}} V_{m}^{2}
\end{aligned}\]

为了降低噪声影响需要增大电容,但是会导致带宽下降。

\subsection{共栅级热噪声}

以下讨论忽略沟道调制效应。

显然输出噪声仅由 \(R_D\) 提供,

\[\overline{V_{n,out}^2} = 4 k T R_D\]

那么 

\[\overline{I_{n,in}^2} = \dfrac{4 k T}{R_D}\]


\qfig{f59.png}{共栅级电路-输入参考电流}

显然

\[\left(4 k T \gamma g_{m}+\dfrac{4 k T}{R_{D}}\right) R_{D}^{2}=\overline{V_{n, i n}^{2}}\left(g_{m}+g_{m b}\right)^{2} R_{D}^{2}\]

\[\overline{V_{n, i n}^{2}}=\dfrac{4 k T\left(\gamma g_{m}+1 / R_{D}\right)}{\left(g_{m}+g_{m b}\right)^{2}}\]

\qfig{f60.png}{共栅级电路-输入参考电压}


\subsection{源随器热噪声}

由于源跟随器的输入阻抗比较高,在大小适中的驱动源阻抗
下,可以忽略输入参考噪声电流。


将 \(M_1\) 使用定理变换到输入,仅求解 \(M_2\)

\[\left.\overline{V_{n, \text {out}}^{2}}\right|_{M 2}=\overline{I_{n 2}^{2}}\left(\dfrac{1}{g_{m 1}}\left\|\dfrac{1}{g_{m b 1}}\right\| r_{O 1} \| r_{O 2}\right)^{2}\]

增益为

\[A_{v}=\dfrac{\dfrac{1}{g_{m b 1}}\left\|r_{O 1}\right\| r_{O 2}}{\dfrac{1}{g_{m b 1}}\left\|r_{O 1}\right\| r_{O 2}+\dfrac{1}{g_{m 1}}}\]

得到

\[\begin{aligned}
    \overline{V_{n, i n}^{2}} &=\overline{V_{n 1}^{2}}+\dfrac{\left.\overline{V_{n, o u t}^{2}}\right|_{M 2}}{A_{v}^{2}} \\
    &=4 k T \gamma\left(\dfrac{1}{g_{m 1}}+\dfrac{g_{m 2}}{g_{m 1}^{2}}\right)
\end{aligned}\]

\qfig{f61.png}{源随器的噪声}


源跟随器为输入信号添加噪声,并提供小于1的电压增益,
信噪比性能较差。
在低噪放大器中通常不使用源跟随器。

% End Here

\ifx\mainclass\undefined
\end{document}
\fi 

\ifx\mainclass\undefined
\documentclass[cn,11pt,chinese,black,simple]{../elegantbook}
\usepackage{array}
\newcommand{\pp}[1]{\partial #1}
\newcommand{\dd}[1]{\mathrm{d}#1}
\newcommand{\abs}[1]{\left| #1 \right|}
\newcommand{\degr}[1]{#1^{\circ}}
\newcommand{\aint}{\int_{-\infty}^{+\infty} }
% \renewcommand{\dfrac}[2]{\dfrac{#1}{#2}}

\newcommand{\qfig}[2]{\begin{figure}[!htb]
    \centering
    \includegraphics[width=0.6\textwidth]{#1}
    \caption{#2}
\end{figure}}
\usepackage{circuitikz}

\renewcommand\arraystretch{1.5}
\begin{document}
\fi 

% Start Here
\chapter{负反馈}

\section{负反馈系统}

\subsection{系统组成}

一个典型的负反馈系统作用时使净输入量减小,或者使输出量
的变化减弱,通常包括以下部分,


\begin{enumerate}
    \item \(H(s)\) :前馈网络,通常是一个放大器。
    \item \(G(s)\) :反馈网络,通常是一个与频率无关的函数。
    \item \(X(s) - G(s) Y(s)\) :反馈误差
    \item \(\dfrac{Y(s)}{X(s)}\) :闭环传输函数
    \item \(H(s)\) :开环传输函数
\end{enumerate}

满足方程

\[\begin{array}{c}
    \begin{aligned}
    Y(s)&=H(s)[X(s)-G(s) Y(s)] \\
    \frac{Y(s)}{X(s)}&=\frac{H(s)}{1+G(s) H(s)}
    \end{aligned}
\end{array}\]

\qfig{f62.png}{典型负反馈系统-复频域}

设计时通常满足以下原则 
\begin{itemize}
    \item 反馈网络 \(G(s)\) 与频率无关,不移相
    \item 深度负反馈 \begin{itemize}
        \item 反馈误差也就是前馈网络的输入极小,可以认为虚地
        \item \(X(s)\) 与 \(Y(s)\) 的比例几乎只与反馈有关
        \item \(G(s) H(s) \gg 1\) 
    \end{itemize}
\end{itemize}

以简单的放大器构成负反馈系统,那么有

\qfig{f63.png}{简单负反馈放大器}

参照以上的讨论,有

\begin{enumerate}
    \item \(A\) : 开环增益
    \item \(\beta\) :反馈系数
    \item \(\beta A\) :环路增益
    \item \(\dfrac{Y}{X}\) :闭环增益
\end{enumerate}

并且满足 \[\dfrac{Y}{X} = \dfrac{A}{1 + \beta A}\]

\subsection{负反馈系统的性质}

\subsubsection{稳定增益}

对于深度负反馈,有 \(\beta A \gg 1\) ,那么

\[\begin{array}{c}
    \frac{Y}{X}=\frac{A}{1+\beta A}=\frac{1}{\beta}\left(1-\frac{1}{1+\beta A}\right) \approx \frac{1}{\beta}\left(1-\frac{1}{\beta A}\right) \approx \frac{1}{\beta} \\
\end{array}\]

闭环增益 \(\dfrac{Y}{X}\) 对于开环增益 \(A\) 的变化敏感程度下降,并且 \(\beta\) 越大增益越稳定,但是同时增益也在减小。
实际中的增益很容易随工艺、温度等发生变化,增益不精确也不稳定,通过负反馈可以实现对增益更精确的控制。

\subsubsection{改变终端阻抗}

开环输入阻抗,即无反馈时的输入阻抗。闭环输入阻抗,即有反馈时的输入阻抗。在之后的小节进行讨论。



\subsubsection{改变带宽}

对于单极点的前馈函数 \(A(s) = \dfrac{A_0}{1 + \dfrac{s}{\omega_0}}\)  增加反馈得到

\[\dfrac{Y(s)}{X(s)} = \dfrac{A(s)}{1 + \beta A(s)} = \dfrac{\dfrac{A_0}{1 + \beta A_0}}{1 + \dfrac{s}{(1+\beta A_0)\omega_0}}\] 

主极点得到改善,但是增益带宽积不变。

\subsubsection{改善线性度}

负反馈放大器的增益稳定性被改善,线性度较好,代价是增益的减小。

\subsubsection{环路增益的求解}

对输入置零,断开环路在反馈点增加测试信号,读取输出端信号,计算与测试信号的比值,取反便可得到环路增益 (loop gain) 。

\qfig{f64.png}{环路增益的求解}

\[\begin{aligned}
    V_t \beta (-1) A &= V_F \\
    \dfrac{V_F}{V_t} = -\beta A
\end{aligned}\]

\section{放大器}

\subsection{基本放大器}

有这样四种基本的放大器,基于这些放大器逐步开展反馈机制的讨论。

\qfig{f65.png}{基本放大器类型}


\begin{longtable}{lllll}
    \caption{基本放大器的阻抗对比} \\
    \toprule \textbf{放大器类型} & \textbf{电压放大器} & \textbf{跨阻放大器} & \textbf{跨导放大器} & \textbf{电流放大器} \\
    \midrule
    \endfirsthead
    
    \toprule \textbf{放大器类型} & \textbf{电压放大器} & \textbf{跨阻放大器} & \textbf{跨导放大器} & \textbf{电流放大器} \\
    \midrule
    \endhead

    \hline
    \multicolumn{5}{c}{见下页}\\   
    \bottomrule
    \endfoot

    \bottomrule
    \endlastfoot

    \textbf{输入阻抗}  & 高              & 低              & 高              & 低              \\
    \textbf{输出阻抗}  & 低              & 低              & 高              & 高              \\
    
\end{longtable}

\qfig{f66.png}{基本放大器的单 MOS 实现}

\qfig{f67.png}{基本放大器的改进实现}

\subsection{检测与反馈机制}

对控制机制进行命名定义:(输出端电学量)-(输入端电学量),“XX—电压反馈”的输入信号必须是电压,
“XX—电流反馈”的输入信号必须是电流。有四种基本反馈机制 
\begin{itemize}
    \item 电压 - 电压反馈
    \item 电压 - 电流反馈
    \item 电流 - 电流反馈
    \item 电流 - 电压反馈
\end{itemize}

图~\ref{fig:ch07:base-feedback}列出了几种基本的反馈方式逐一进行说明:

\begin{itemize}
    \item (a) 电压通过电阻分压反馈到输入电压
    \item (b)(c) 电流通过电阻分压反馈到输入电压
    \item (d) 电压通过电阻反馈到输入电压,并且通过差动对完成电压减法
    \item (e)(f) 电压通过电阻分压反馈到输入电压,通过栅源压降完成减法
    \item (g)(h) 电流通过并联反馈到输入电流
\end{itemize}

\begin{figure}[htb]
    \centering
    \includegraphics[width=0.8\textwidth]{f68.png}
    \caption{基本反馈方式}
    \label{fig:ch07:base-feedback}
\end{figure}

在输出端检测电压应使用并联,检测电流使用串联。

可以注意到这样的规律:电压负反馈的输入量与反馈量在两个节点,
电流负反馈的输入量与反馈量在同一个节点。这个规律可以用来确定反馈类型。

\subsubsection{电压-电压反馈}

前馈网络相当于一个差动输入单端输出的放大器,反馈网络是一个分压器,与输出电压并联分压,反馈串联到输入。理想的反馈网络有无穷大的输入阻抗,接受全部电压,具有零输出阻抗,电压完全反馈。

\qfig{f69.png}{电压-电压反馈}

\paragraph{输出阻抗}

负反馈系统希望得到线性性良好的输入输出关系,当放大器带载时,逐渐减小其阻值。在开环结构中,输出将按 \(R_{ L } /\left(R_{ L }+R_{ out }\right)\) 比例下降。在反馈系统 中,即使 \(R_{ L }\) 减小, \(V_{\text {out }}\) 也始终是 \(V_{\text {ia }}\) 的合理线性复制。也就是说,只要环路增益保持远大于 \(1\) 则增益与负载无关。

\qfig{f70.png}{带载放大器}

检测反馈网络的输出电阻,理想反馈网络的输入电阻无穷大,所以反馈网络返回的电流忽略不计。

\qfig{f71.png}{输出阻抗检测}

可以得到

\[\frac{V_{X}}{I_{X}}=\frac{R_{\text {out }}}{1+\beta A_{0}}\]

可以看出输出阻抗和增益 \(\dfrac{1}{\beta}\) 成比例减小。

整体上,输出阻抗被减小,更接近理想
电压源输出。
输出电压更稳定,不易受负载
电阻影响。

\paragraph{输入阻抗}

\qfig{f72.png}{输入阻抗示意}

\qfig{f73.png}{输入阻抗检测}

进行求解

\[\begin{array}{l}
    V_{e}= V _{X}-V_{F}=V_{X}-\beta A_{0} I_{X} R_{i n} \\
    I_{X} R_{i n}= V _{X}-\beta A_{0} I_{X} R_{i n} \\
    \frac{V_{X}}{I_{X}}=R_{i n}\left(1+\beta A_{0}\right)
\end{array}\]

输入阻抗被增大,更接近理想电压放大器。

\subsubsection{电流-电压反馈}

反馈网络与前馈网络的输出量串联,与输入量串联。
前馈网络必须加负载电阻,以产生电流。
理想反馈网络具有零输入阻抗,将输出电流全部接收。
理想反馈网络具有零输出阻抗,将反馈电压全部送到输入端。

\qfig{f74.png}{电流-电压反馈}

\paragraph{增益}

闭环增益为 

\[\begin{array}{l}
    \frac{I_{o u t}}{V_{i n}}=\frac{G_{m}}{1+G_{m} R_{F}}
\end{array}\]

\qfig{f75.png}{开环增益检测}

开环增益为 \[-\frac{I_{o u t}}{I_{t}}=G_{m} R_{F}\]

\paragraph{输出阻抗}

满足

\[-R_{F} I_{X} G_{m}=I_{X}-\frac{V_{X}}{R_{O u t}} \quad \frac{V_{X}}{I_{X}}=R_{o u t}\left(1+G_{m} R_{F}\right)\]

\qfig{f76.png}{输出阻抗检测}

\paragraph{输入阻抗}

输入阻抗被增大,更接近理想电压放大器。

\[\begin{array}{l}
    I_{X} R_{i n} G_{m}=I_{o u t} \\
    V_{e}=V_{X}-G_{m} R_{F} I_{X} R_{i n} \\
    \frac{V_{X}}{I_{X}}=R_{i n}\left(1+G_{m} R_{F}\right)
\end{array}\]

\qfig{f77.png}{输入阻抗检测}

\subsubsection{电压-电流反馈}

反馈网络与前馈网络的输出量并联,与输入量并联。
理想反馈网络具有无穷大输入阻抗,将输出电压全部接收。
理想反馈网络具有无穷大输出阻抗,将反馈电流全部送到输入端。

\qfig{f78.png}{电压-电流反馈电路}

增益为 
\[\frac{V_{out}}{I_{in}} = \dfrac{R_0}{1 + g_{mF}} R_0\]

\qfig{f79.png}{输入与输出阻抗}

输入阻抗为

\[\begin{aligned}
    I_{F} &=I_{X}-\frac{V_{X}}{R_{i n}} \frac{V_{X} R_{0} g_{m F}}{R_{i n}}=I_{F} \\
    \frac{V_{X}}{I_{X}} &=\frac{R_{i n}}{1+g_{m F} R_{0}}
\end{aligned}\]

输出阻抗为

\[\begin{array}{c}
    \begin{aligned}
        I_{e}&=-I_{F}=-g_{m F} V_{X} \\
        V_{M}&=-R_{0} g_{m F} V_{X} \\
        I_{X}&=\frac{V_{X}-V_{M}}{R_{o u t}}=\frac{V_{X}+g_{m F} R_{0} V_{X}}{R_{o u t}} \\
        \frac{V_{X}}{I_{X}}&=\frac{R_{o u t}}{1+g_{m F} R_{0}}
    \end{aligned}
\end{array}\]

输入阻抗被减小,更接近理想电流放大器。输出阻抗被减小,更接近理想电压源输出。

\subsubsection{电流电流反馈}

\qfig{f80.png}{电流-电流反馈图}

反馈网络与前馈网络的输出量串联,与输入量并联。
理想反馈网络具有零输入阻抗和无穷大输出阻抗。

闭环增益 \[\dfrac{A_1}{1 + \beta A_1}\] 。
输入阻抗减小了\(1 + \beta A_1\),情形类似于“电压—电流反馈”的输入阻抗。
输出阻抗增大了\(1 + \beta A_1\),情形类似于“电流—电压反馈”的输出阻抗。

\subsection{负反馈系统的评估}


\subsubsection{阻抗}

负反馈总是使放大器的输入/输出阻抗变得更加理想。

\begin{longtable}{lll}
    \caption{反馈引起的阻抗变化} \\
    \toprule \text { 反馈类型 } & \text { 输入阻抗 } & \text { 输出阻抗 } \\
    \midrule
    \endfirsthead
    
    \toprule\text { 反馈类型 } & \text { 输入阻抗 } & \text { 输出阻抗 } \\
    \midrule
    \endhead

    \hline
    \multicolumn{3}{c}{见下页}\\   
    \bottomrule
    \endfoot

    \bottomrule
    \endlastfoot
    \text { 电压一电压反馈 } & \text { 增大 } & \text { 减小 } \\
    \text { 电流一电压反馈 } & \text { 增大 } & \text { 增大 } \\
    \text { 电压一电流反馈 } & \text { 减小 } & \text { 减小 } \\
    \text { 电流一电流反馈 } & \text { 减小 } & \text { 增大 } \\
\end{longtable}

\subsubsection{噪声}


负反馈没有消除前馈通路的噪声。实际中,反馈网络也会引入噪声,因此总的噪声性能更差。

\qfig{f81.png}{负反馈对噪声的影响}


\[\begin{array}{l}
    \left(V_{\text {in }}-\beta V_{\text {out }}+V_{n}\right) A_{1}=V_{\text {out }} \\
    V_{\text {out }}=\left(V_{\text {in }}+V_{n}\right) \frac{A_{1}}{1+\beta A_{1}}
\end{array}\]

\section{加载效应}


反馈网络的输入电阻不是“无穷大”,开环增益被减小。
前述理想模型下输出阻抗的计算也不适用。
如果由电容反馈分压,则在低频下几乎不影响开环增益。这都被称为“加载效应”。
反馈网络和前馈网络纠缠,难以清晰分割。反馈回路可能不唯一。

\subsection{线性二端口网络模型与放大器}

二端口网络用来描述线性电路,可以描述加载效应。
过程为
断开加载环路,计算开环增益。
确定反馈系数,得出环路增益。
这是一种近似(环路单向化假设),并非精确解。

\qfig{f82.png}{二端口网络模型}

\[\begin{array}{l}
    V_{1}=Z_{11} I_{1}+Z_{12} I_{2} \\
    V_{2}=Z_{21} I_{1}+Z_{22} I_{2} \\
    \\
    I_{1}=Y_{11} V_{1}+Y_{12} V_{2} \\
    I_{2}=Y_{21} V_{1}+Y_{22} V_{2} \\
    \\
    V_{1}=H_{11} I_{1}+H_{12} V_{2} \\
    I_{2}=H_{21} I_{1}+H_{22} V_{2} \\
    \\
    I_{1}=G_{11} V_{1}+G_{12} I_{2} \\
    V_{2}=G_{21} V_{1}+G_{22} I_{2}
\end{array}\]

\subsection{电压-电压反馈加载效应}

对于电压放大器而言,G模型更直观、更有物理意义。
于普通的电压放大器,内部反馈\(G_{12}I_2\)被忽略,则\(G_{11}\)就是输入
\textbf{导纳},\(G_{22}\)就是输出\textbf{阻抗},\(G_{21}\)就是开路电压增益。

为简化计算,内部反馈被忽略,实现环路单向化近似。

那么 \(G_{11} \rightarrow Z_{in}\), \(G_{22} \rightarrow Z_{out}\) , \(G_{21} \rightarrow A_0\)。若是 \(g_{11} = g_{22} = 0\) 则得到理想模型,没有加载效应。

\qfig{f83.png}{二端口网络表示电压-电压反馈}

进一步确定加载效应的参数,输出短接,可以获得 \(g_{22}\) , 输入开路,可以获得 \(g_(11)\),\(g_{21}\) 则是反馈网络的增益。

\subsection{电流-电压反馈加载效应}

前馈网络Y模型适用;
反馈网络Z模型适用。

\qfig{f84.png}{电流-电压反馈加载效应}

\subsection{电压-电流反馈加载效应}

前馈网络Z模型适用;
反馈网络Y模型适用。

\qfig{f85.png}{电压-电流反馈加载效应}

\subsection{电流-电流反馈加载效应}

前馈网络H模型适用;
反馈网络H模型适用。

\qfig{f86.png}{电流-电流反馈加载效应}

% End Here

\ifx\mainclass\undefined
\end{document}
\fi 

\part{运算放大器}

\ifx\mainclass\undefined
\documentclass[cn,11pt,chinese,black,simple]{../elegantbook}
\usepackage{array}
\newcommand{\pp}[1]{\partial #1}
\newcommand{\dd}[1]{\mathrm{d}#1}
\newcommand{\abs}[1]{\left| #1 \right|}
\newcommand{\degr}[1]{#1^{\circ}}
\newcommand{\aint}{\int_{-\infty}^{+\infty} }
% \renewcommand{\dfrac}[2]{\dfrac{#1}{#2}}

\newcommand{\qfig}[2]{\begin{figure}[!htb]
    \centering
    \includegraphics[width=0.6\textwidth]{#1}
    \caption{#2}
\end{figure}}
\usepackage{circuitikz}

\renewcommand\arraystretch{1.5}
\begin{document}
\fi 

% Start Here

\chapter{运算放大器设计基础}

我们粗略地把运放定义为“高增益的差动放大器”。所谓“高”,指的是对应用,增益足够高,通常增益范围在 \( 10^{1} \sim 10^{5} \) 。由于运放一般用来实现一个反馈系统,以提供稳定性、精度等指标,其开环增益的大小根据闭环电路的精度要求来设计。应用范围极广:信号运算、高速放大、滤波、缓冲器。

理想的运放:无限大增益、无限高输入阻抗、零输出阻抗。
实际的运放设计却是多个指标的复杂折中。



\section{一级运放}

一级运放的期望增益为 \(\dfrac{1}{\beta}\) ,那么误差为 \(\dfrac{\dfrac{1}{\beta} - \dfrac{V_{out}}{V_{in}}}{\dfrac{1}{\beta}}\) ,其中实际增益为 \(\dfrac{V_{o u t}}{V_{i n}}=\dfrac{A_{1}}{1+\dfrac{R_{2}}{R_{1}+R_{2}} A_{1}}\)

\qfig{f87.png}{单级放大器}

带有极点的运放处理阶跃信号时, \(V_{in} = \dfrac{a}{s}\) 。
系统函数为

\[ \dfrac{A(s)}{1+\beta A(s)}=\dfrac{A_{0} /\left(1+s / \omega_{0}\right)}{1+\beta A_{0} /\left(1+s / \omega_{0}\right)}=\dfrac{A_{0} /\left(1+\beta A_{0}\right)}{1+s /\left[\left(1+\beta A_{0}\right) \omega_{0}\right]} \approx \dfrac{1 / \beta}{1+s /\left(\beta A_{0} \omega_{0}\right)} \]

输出为 \[
    \begin{array}{l}
        \begin{aligned}
            
            V_{\text {out }}(t) &= L ^{-1}\left[V_{\text {out }}(s)\right]= L ^{-1}\left[V_{\text {in }}(s) \dfrac{A(s)}{1+\beta A(s)}\right] \\
            \approx & a \dfrac{1}{\beta}\left(1-\exp \dfrac{-t}{1 /\left(\beta A_{0} \omega_{0}\right)}\right) u(t) 
        \end{aligned}
    \end{array}    
\]

其中 \[1+\dfrac{R_{1}}{R_{2}}=\dfrac{1}{\beta}\]

可以从时域的输出中计算速度以及对应约束的带宽。单位增益带宽定义为 \(A_0 \omega_0\) 其中 \(\omega_0\) 为主极点。

\qfig{f88.png}{带有极点的单级放大器}


\subsection{套筒式运放}

设计流程相对固定。

\qfig{f89.png}{套筒式共源共栅结构}

\begin{enumerate}
    \item 根据功率分配支路电流;
    \item 根据摆幅分配过驱动电压;
    \item 根据I-V特性计算宽长比;
    \item 检查增益是否符合要求;
    \item 若不符合要求,调整沟道长度。
    \item 确定共模电平和偏置电压
\end{enumerate}

\begin{example}
    Design a fully differential telescopic op amp with the following specifications: \(V_{D D}=3 V\), peak-to-peak differential output swing \(=3 V\), power dissipation \(=10 mW\), voltage gain \(=2000 .\) Assume that \(\mu_{n} C_{o x}=60 \mu A / V ^{2}, \mu_{p} C_{o x}=\) \(30 \mu A / V ^{2}, \lambda_{n}=0.1 V ^{-1}, \lambda_{p}=0.2 V ^{-1}\) (for an effective channel length of \(0.5 \mu m\) ), \(\gamma=0\), and \(V_{T H N}=\left|V_{T H P}\right|=\)
\(0.7 V\)
\end{example}

\begin{solution}
    从功耗与电压得到,总电流为 \(3.33 mA\),分配各自 \(1.5 mA\) 给两路,其余给参考电流源。

    摆幅为 \(3 V\) ,那么 X, Y 均可以摆动 \(1.5 V\),且要保证3,4,5,6管子不进入线性区。同时管子1,7都是共源共栅结构,电压影响较小,可以适当的降低其过驱动电压,使之在线性区边缘。
    
    目前共有 \(1.5 V\) 过驱动电压,那么让电流最大的 9 号的尺寸可接受,使之接受 \(0.5 V\);由于 PMOS 的迁移率低,分配的过驱动电压多一点, \(300 mV\),剩下的均分给 NMOS 1,3 ,各自 \(200 mV\),作为初始猜测。

    计算时,为了简化,暂时忽略沟道调制 \[I_D = \dfrac{1}{2} \mu_n C_{ox} \dfrac{W}{L} (V_{OV})^2\]
    带入求解 得到 \((W/L)_{1-4} = 1250\) , \((W/L)_{5-8} = 1111\) ,\((W/L)_9 = 400\) 。

    进一步验证增益 \[A_{v} \approx g_{m 1}\left[\left(g_{m 3} r_{O 3} r_{O 1}\right) \|\left(g_{m 5} r_{O 5} r_{O 7}\right)\right]\] 

    发现结果为 \(1419\) 比指标低,一般通过调节尺寸来改进,又 \(\lambda \propto 1/L\) ,\(g_m r_o \propto \sqrt{2 \mu C_{ox} (W/L)/I_D} \dfrac{1}{\lambda} \) 那么 \( g_m r_o \propto \sqrt{W L / I_D}\) 。由于 PMOS 对性能影响较小,可以调制其尺寸。

    \((W/L)_{5-8} = 1111\) 保持,但是\(L\) 翻倍时,增益变为 \(4000\) 。同时由于其尺寸变大,可以从尾电流源中分配一些电压。


\end{solution}

其问题也很明显:为获得最大摆幅 M1 和 M7 刚好工作在饱和区边缘,不得不严格控制。如果用套筒式共源共栅OTA运放设计单位增益缓冲器,输出电压范围小于一个阈值电压。并且很难使得输入输出短接。

\qfig{f90.png}{单位增益缓冲器}

\subsection{折叠式运放}

需要两路电流,功耗比套筒式共源共栅大。
输出阻抗小,增益比套筒式共源共栅小,比套筒式低2~3倍。
输出摆幅比套筒式共源共栅大,比套筒式多1个过驱动电压。折叠式比套筒式的极点频率小。

\begin{example}
    Design a folded-cascode op amp with an NMOS input pair (Fig. 9.18) to satisfy the following specifications:
\(V_{D D}=3 V ,\) differential output swing \(=3 V ,\) power dissipation \(=10 mW ,\) and voltage gain \(=2000 \) . Assume that \(\mu_{n} C_{o x}=60 \mu A / V ^{2}, \mu_{p} C_{o x}=\) \(30 \mu A / V ^{2}, \lambda_{n}=0.1 V ^{-1}, \lambda_{p}=0.2 V ^{-1}\) (for an effective channel length of \(0.5 \mu m\) ), \(\gamma=0\), and \(V_{T H N}=\left|V_{T H P}\right|= 0.7 V\)
\end{example}

\begin{solution}
    同样从功耗与摆幅入手。分配输入对 \(1.5 mA\) ,共源共栅支路 \(1.5 mA\) ,剩余的电流给三个电流镜。
    
    摆幅决定了过驱动电压之和为 \(1.5 V\) ,对于过驱动电压,同样由于传导大电流,M5,6分配 \(0.5 V\),由于 PMOS 的迁移率较低,M3,4 分配\(0.4 V\) ,M7,8,9,10 分配\(0.3 V\) 。那么可以得到输出最高电压为 \(3 - 0.5 - 0.4 = 2.1 V\) ,输出的最低电压为 \(0.3 + 0.3 = 0.6 V\) ,那么最合适的共模电平为 \(1.35 V\)。

    对差分输入端进行电压分配,当输入输出短接有 \(V_{GS2} + V_{OD11} = 1.35 V\) ,初始假设 \(V_{OG11} = 0.4 V\) ,那么 \(V_{OD1,2} = 1.35 - 0.4 - 0.7 = 0.25 V\) 。

    目前所有的过驱动电压与电流已知,求解尺寸。M1,2:\(400\) ,M3,4:\(312.5\),M5,6: \(400\),M7,8,9,10: \(277.8\) , M11 \(312.5\) 。

    计算增益,等效跨导为 \[ G_m = g_{m1} \dfrac{r_{o1} \| r_{o5}}{r_{o1} \| r_{o5} + \dfrac{r_{o3}}{1+g_{m3} r_{o3}}}\]

    输出阻抗为 \[R_{out} = [(1 + g_{m7} r_{o7})r_{o9} + r_{o7}] \left|\right| [(1 + g_{m3} r_{o3})(r_{o5} \| r_{o1}) + r_{o3}]\]
    
    首先计算跨导,利用公式 \(g_m = 2 I_D / (V_{GS} - V_{TH})\) 。\(g_{m1,2} = 0.006 A/V\), \(g_{m3,4} = 0.00375 A/V\), \(g_{m5,6} = 0.006 A/V\), \(g_{m7,8,9,10} = 0.005 A/V\) 。计算沟道电阻 \(r_o = \dfrac{1}{\lambda I_D}\),\(r_{o1,7,9} = 13.3 k\Omega\) , \(r_{o5} = 3.33 k\Omega\) , \(r_{o3} = 6.67 k\Omega\) 。
    
    带入得到增益为 \(383\) 

    将 M1 扩充 3 倍,M5 扩充 5 倍, M3 扩充 1.8 倍,增益达到 2019 。


\end{solution}

\qfig{f91.png}{折叠式共源共栅}

\qfig{f92.png}{折叠式共源共栅-输入输出短接}

\subsection{共模反馈}

输出与输入短接电路,普通电阻做负载,输入/输出共模
电平比较稳定。

\qfig{f93.png}{输出与输入短接}


如果用电流源做负载,输入/输出共模电平不稳定,
对环境参数较为敏感。
如果管子参数与设计值稍有偏差,需要X和Y
点付出更大的偏差来平衡电流。
简单的差动反馈无法解决这个问题。
必须采用共模反馈技术来稳定输出共模电平。

\qfig{f94.png}{输出与输入短接-电流源负载}


\section{两级运放}

\subsection{反馈系统的稳定性}

\qfig{f96.png}{潜在的不稳定性}

反馈系统存在潜在的不稳定性,如果 \(\beta H\left(j \omega_{1}\right)=-1,\) 则 系统在 \(\omega_{1}\) 处发生振荡。 振荡时,环路的总相移为\(\degr{360}\),环路增益\(\geq 1\)。

环路增益的幅值为1的频率点称为 “增益交点(GX)”,环路增益的相位为-180度时的频率点称为 “相位交点(PX)”。PX必须发生在GX之后,确保:在-180度相位时,环路增
益已经小于1,不会形成振荡。

如果反馈减弱,之前描述的系统变得更稳定。

\subsubsection{单极点系统}
\[
    \begin{array}{l}
        H(s)=A_{0} /\left(1+s / \omega_{0}\right) \\
        \frac{Y}{X}(s)=\frac{\frac{A_{0}}{1+\beta A_{0}}}{1+\frac{s}{\omega_{0}\left(1+\beta A_{0}\right)}}
    \end{array}\]

相移始终小于 \(\degr{90}\),系统无条件稳定。

\subsubsection{双极点系统}

相位最终趋于-180度,因此,当环路增益的幅值
降至1时,相位尚未相移180度,系统稳定。反馈变弱,系统会更稳定。

\subsubsection{三极点系统}

三极点系统可能不稳定,尤其是PX点比GX点左移更明显。

PX点与GX点重合时,闭环增益的幅值无限大,从而导致
振荡。

相位裕度越大,PX与GX相隔越远。一般保证 \(\degr{60}\) 的相位裕度。

\[P M=180^{\circ}+\angle \beta H\left(\omega=\omega_{G X}\right)\]



\subsection{相位裕度}


GX低于PX不足以保持不出现近似的尖峰或者虽然稳定,但仍有可能产生减幅振荡。。PX与GX相隔足够远,系统更稳定,无减幅振荡。

\subsection{Miller OTA}

已知运放有两个极点和一个右半平面零点,设零点大于\(10 \omega_{GX}\),  为了使相位裕度大于 \(\degr{60}\),第二极点至少高于 \(2.2 \omega_{GX}\)

\qfig{f97.png}{密勒补偿电路}

加入补偿电容后,根据密勒定理:
与\(R_I\)并联的容值增大,该点对应的极点频率减小;
负反馈降低第二级的输出电阻,输出点对应的极点频率增大。拉远两个极点频率的距离,提高相位裕度。产生一个右半平面零点频率,会增加相移和增益,
使相位裕度恶化。应将零点频率推离原点。

\[
\begin{array}{l}
\frac{V_{o u t}}{V_{i n}}(s)=\frac{g_{m I} g_{m I I} R_{I} R_{I I}\left(1-s C_{c} / g_{m I I}\right)}{R_{I} R_{I I} \xi s^{2}+\left[g_{m I I} R_{I} R_{I I} C_{c}+R_{I I}\left(C_{I I}+C_{c}\right)+R_{I}\left(C_{I}+C_{c}\right)\right] s+1} \\
\end{array}
\]


其中
\[\xi=C_{I} C_{I I}+C_{I} C_{c}+C_{I I} C_{c}\]

主极点近似法:
\[
\begin{array}{l}
    \begin{aligned}
        
    \omega_{p 1} &\approx \frac{1}{g_{m I I} R_{I} R_{I I} C_{c}+R_{I I}\left(C_{I I}+C_{c}\right)+R_{I}\left(C_{I}+C_{c}\right)} \approx \frac{1}{g_{m I I} R_{I} R_{I I} C_{c}} \\
    \omega_{p 2} &\approx \frac{g_{m I I} C_{c}}{C_{I} C_{I I}+C_{I} C_{c}+C_{I I} C_{c}} \approx \frac{g_{m I I}}{C_{I I}}
    \end{aligned} 
\end{array}
\]

满足 \[ C_{I I} \gg C_{I}, \quad C_{c} \gg C_{I}\]

增益带宽积为 \[G B W=A_{v 0} \omega_{p 1} \approx g_{m I} R_{I} g_{m I I} R_{I I} \frac{1}{g_{m I I} R_{I} R_{I I} C_{c}}=\frac{g_{m I}}{C_{c}}\]

\qfig{f98.png}{密勒OTA}

通常
给定这些指标
直流增益 \(A_v\)
增益带宽积 \(GBW\)
输入共模范围 \(ICMR\)
负载电容 \(C_L\)
转换速率 \(SR\)
输出电压摆幅
功耗 \(P_{diss}\)

其中 \[A_{v1} = -g_{m1}(r_{o1} \| r_{o4})\]

\[A_{v2} = -g_{m6} (r_{o6 \| r_{o7}})\]

\[SR = \dfrac{I_5}{C_c}\]

\[GBW = \dfrac{g_{mI}}{C_c} = \dfrac{g_{m1}}{C_c} = \omega_{GX}\]

\[\omega_{p2} = \dfrac{g_{mII}}{C_{II}} = \dfrac{g_{m6}}{C_L}\]

\[\omega_z = \dfrac{g_{mII}}{C_c} = \dfrac{g_{m6}}{C_c}\]

输出共模下限 
\[V_{SS} + \sqrt{\dfrac{I_5}{\mu_n C_{ox} (W/L)_1}} + V_{TH1max} + V_{OD5}\]

上限为 \[\begin{array}{l}
    V_{D D}-\left|V_{S G 3}\right|+V_{T H 1_{-} \min } \\
    =V_{D D}-\left(\left|V_{S G 3}\right|-\left|V_{T H 3_{-} \max }\right|\right)-\left|V_{T H 3_{-} \max }\right|+V_{T H 1_{-} \min } \\
    =V_{D D}-\sqrt{\frac{I_{5}}{\mu_{p} C_{o x}(W / L)_{3}}}-\left|V_{T H 3_{3} \max }\right|+V_{T H_{1} \min }
    \end{array}\]

\begin{example}
    \(A_{v}>5000\)
\(G B W>5 M H z\)
输出电压摆幅 \(=\pm 2 V\)
\(V_{D D}=2.5 V\)
\(C_{L}=10 p F\)
\(I C M R=-1 \sim 2 V\)
\(V_{S S}=-2.5 V\)
\(S R>10 V / \mu s\)
\(P_{\text {diss}} \leq 2 m W\)
\(\mu_{n} C_{o x}=110 \mu A / V^{2}, \mu_{p} C_{o x}=50 \mu A / V^{2} \quad\left|V_{T H 3_{-} m a x}\right|=0.85 V\)
\(V_{T H 1} \min =0.55 V\)
\(\lambda_{n}=0.04 V^{-1}, \lambda_{p}=0.05 V^{-1} \quad V_{T H 1_{-} \max }=0.85 V\)
\(L=1 \mu m\)
\end{example}

\begin{solution}
    根据相位裕度要求 \(\omega_z > 10 \omega_{DX}\) 且 \(\omega_{p2} > 2.2 \omega_{GX}\)
    得到 \(\dfrac{g_{m6}}{C_c} > 10 \dfrac{g_{m2}}{C_c}\) 与 \(\dfrac{g_{m6}}{C_L} > 2.2 \dfrac{g_{m2}}{C_c}\) 
    化简 \(g_{m6} > 10 g_{m2}\) \(C_c > \dfrac{2.2 C_2}{10} = 0.22 C_L = 2.2 pF\)
    暂时选定为 \(C_c = 3 pF\)


    根据转换速率 \(SR = \dfrac{I_5}{C_c}\) 解得 \(I_5 = 30 \mu A\)

    根据共模上限确定 \((W/L)_{3,4} = 15\) \[2 V = V_{D D}-\sqrt{\frac{I_{5}}{\mu_{p} C_{o x}(W / L)_{3}}}-\left|V_{T H 3_{3} \max }\right|+V_{T H_{1} \min }\]

    根据增益带宽积确定 \(g_{m1}\) 
    \[g_{m1 = GBW  C_C = 94.25 \mu S}\]

    通过偏置电流计算 \((W/L)_{1,2}\)
    \[(W / L)_{1}=\frac{g_{m 1}^{2}}{2 \mu_{n} C_{o x}\left(I_{5} / 2\right)}=2.79 \approx 3.0=(W / L)_{2}\]

    通过共模下限确定 \(V_{OD5}\) 与 \((W/L)_5\)

    \[\begin{array}{l}
        V_{S S}+\sqrt{\frac{I_{5}}{\mu_{n} C_{o x}(W / L)_{1}}}+V_{T H 1_{-} \max }+V_{O D 5}=-1 V \Rightarrow V_{O D 5}=350 mV \\
        (W / L)_{5}=\frac{2 I_{5}}{\mu_{n} C_{o x} V_{O D 5}^{2}}=4.49 \approx 4.5
    \end{array}\]

    由相位裕度的要求 \(g_{m9} \geq 10 g_{m1,2} = 942.5 \mu S\)

    假设电流镜像确定 \(V_{SG4} = V_{SG6}\) 解得 \((W/L)_6 = (W/L)_4  \dfrac{g_{m6}}{g_{m4}} = 94.25\)
    其中 \(g_{m4}\) 可求。

    那么 \[I_6 = \dfrac{g_{m6}^2}{2 \mu_p C_{ox} (W/L)_6} = 94.5 \mu A\]

    再次通过镜像电流源确定 \((W/L)_7 = (W/L)_5 \dfrac{I_6}{I_5}\)

    检查M6和M7消耗的过驱动电压,符合摆幅要求

    \[V_{O D 6}=\sqrt{\frac{2 I_{6,7}}{\mu_{p} C_{o x}(W / L)_{6}}}=0.201 V \quad V_{O D 7}=\sqrt{\frac{2 I_{6,7}}{\mu_{n} C_{o x}(W / L)_{7}}}=0.351 V\]

    功耗为 \[P_{diss} = 5 V (30 \mu A + 95 \mu A) = 0.625 mW\]

    增益为 \(A_v = A_1 A_2 = 7696\)

\end{solution}




% End Here

\ifx\mainclass\undefined
\end{document}
\fi 

\ifx\mainclass\undefined
\documentclass[cn,11pt,chinese,black,simple]{../elegantbook}
\usepackage{array}
\newcommand{\pp}[1]{\partial #1}
\newcommand{\dd}[1]{\mathrm{d}#1}
\newcommand{\abs}[1]{\left| #1 \right|}
\newcommand{\degr}[1]{#1^{\circ}}
\newcommand{\aint}{\int_{-\infty}^{+\infty} }
% \renewcommand{\dfrac}[2]{\dfrac{#1}{#2}}

\newcommand{\qfig}[2]{\begin{figure}[!htb]
    \centering
    \includegraphics[width=0.6\textwidth]{#1}
    \caption{#2}
\end{figure}}
\usepackage{circuitikz}

\renewcommand\arraystretch{1.5}
\begin{document}
\fi 

% Start Here

\chapter{理想运放}

\section{概念}

开环电压增益\(A_{od}\)无穷大,
差模输入电阻无穷大,
带宽无穷大,
共模抑制比无穷大,
输出电阻为零,
噪声为零。

运放分为线性工作区和非线性工作区 (正向/负向饱和区)。用来设计运算电路时,处于线性工作区。用来设计比较器时,处于非线性工作区。

线性区有 \(u_o = A_{od} (u_p - u_n)\)。

理想运放有虚短和虚断。

\qfig{f99.png}{理想运放}

好的,复习不完了,这里就咕咕咕了。

% End Here

\ifx\mainclass\undefined
\end{document}
\fi 

\appendix

\addappheadtotoc

\ifx\mainclass\undefined
\documentclass[cn,11pt,chinese,black,simple]{../elegantbook}
\usepackage{array}
\newcommand{\pp}[1]{\partial #1}
\newcommand{\dd}[1]{\mathrm{d}#1}
\newcommand{\abs}[1]{\left| #1 \right|}
\newcommand{\degr}[1]{#1^{\circ}}
\newcommand{\aint}{\int_{-\infty}^{+\infty} }
% \renewcommand{\dfrac}[2]{\dfrac{#1}{#2}}

\newcommand{\qfig}[2]{\begin{figure}[!htb]
    \centering
    \includegraphics[width=0.6\textwidth]{#1}
    \caption{#2}
\end{figure}}
\usepackage{circuitikz}

\renewcommand\arraystretch{1.5}
\begin{document}
\fi 

% Start Here

\chapter{SPICE 模型参数}

注意,\(\epsilon_{ox} = 3.51511 \times  10^{-11} \) F/m

\begin{longtable}{lll}
    \caption{SPICE \(0.5 \mu m\) 模型参数} \\
    \toprule \textbf{NMOS Model} & \textbf{PMOS Model} & \textbf{Para}                                                                                                            \\
    \midrule
    \endfirsthead
    
    \toprule \textbf{NMOS Model} & \textbf{PMOS Model} & \textbf{Para}                                                                                                            \\
    \midrule
    \endhead

    \hline
    \multicolumn{3}{c}{见下页}\\   
    \bottomrule
    \endfoot

    \bottomrule
    \endlastfoot
    LEVEL = 1           & LEVEL = 1           &                                                                                                                             \\
    VTO = 0.7           & VTO = −0.8          & \begin{tabular}[c]{@{}l@{}}VTO: threshold voltage with zero VSB \\ (unit: V)\end{tabular}                                   \\
    GAMMA = 0.45        & GAMMA = 0.4         & \begin{tabular}[c]{@{}l@{}}GAMMA: body-effect coefficient \\ (unit: V\(^{1/2}\))\end{tabular}                                      \\
    PHI = 0.9           & PHI = 0.8           & PHI: 2$\Phi$F (unit: V)                                                                                                     \\
    NSUB = 9e+14        & NSUB = 5e+14        & NSUB: substrate doping (unit: cm\(^{−3}\))                                                                                         \\
    LD = 0.08e-6       & LD = 0.09e-6       & LD: source/drain side diffusion (unit: m)                                                                                   \\
    UO = 350e-4           & UO = 100e-4            & UO: channel mobility (unit: m\(^2\)/V/s)                                                                                        \\
    LAMBDA = 0.1        & LAMBDA = 0.2        & \begin{tabular}[c]{@{}l@{}}LAMBDA: channel-length modulation coefficient \\ (unit: V\(^{-1}\))\end{tabular}                        \\
    TOX = 9e-9         & TOX = 9e-9         & TOX: gate-oxide thickness (unit: m)                                                                                         \\
    PB = 0.9            & PB = 0.9            & \begin{tabular}[c]{@{}l@{}}PB: source/drain junction built-in potential \\ (unit: V)\end{tabular}                           \\
    CJ = 0.56e-3       & CJ = 0.94e-3       & \begin{tabular}[c]{@{}l@{}}CJ: source/drain bottom-plate \\ junction capacitance per unit area \\ (unit: F/m\(^2\))\end{tabular} \\
    CJSW = 0.35e-11    & CJSW = 0.32e-11    & \begin{tabular}[c]{@{}l@{}}CJSW: source/drain sidewall \\ junction capacitance per unit length \\ (unit: F/m)\end{tabular}  \\
    MJ = 0.45           & MJ = 0.5            & MJ: exponent in CJ equation (unitless)                                                                                      \\
    MJSW = 0.2          & MJSW = 0.3          & MJSW: exponent in CJSW equation (unitless)                                                                                  \\
    CGDO = 0.4e-9      & CGDO = 0.3e-9      & \begin{tabular}[c]{@{}l@{}}CGDO: gate-drain overlap capacitance \\ per unit width (unit: F/m)\end{tabular}                  \\
    JS = 1.0e-8        & JS = 0.5e-8        & \begin{tabular}[c]{@{}l@{}}JS: source/drain leakage current\\  per unit area (unit: A/m\(^2\))\end{tabular}                      \\
                        &                     & \begin{tabular}[c]{@{}l@{}}CGSO: gate-source overlap capacitance \\ per unit width (unit: F/m)\end{tabular}                 \\
     
\end{longtable}
% End Here

\ifx\mainclass\undefined
\end{document}
\fi 

\end{document}
